% Options for packages loaded elsewhere
\PassOptionsToPackage{unicode}{hyperref}
\PassOptionsToPackage{hyphens}{url}
%
\documentclass[
]{article}
\usepackage{amsmath,amssymb}
\usepackage{iftex}
\ifPDFTeX
  \usepackage[T1]{fontenc}
  \usepackage[utf8]{inputenc}
  \usepackage{textcomp} % provide euro and other symbols
\else % if luatex or xetex
  \usepackage{unicode-math} % this also loads fontspec
  \defaultfontfeatures{Scale=MatchLowercase}
  \defaultfontfeatures[\rmfamily]{Ligatures=TeX,Scale=1}
\fi
\usepackage{lmodern}
\ifPDFTeX\else
  % xetex/luatex font selection
\fi
% Use upquote if available, for straight quotes in verbatim environments
\IfFileExists{upquote.sty}{\usepackage{upquote}}{}
\IfFileExists{microtype.sty}{% use microtype if available
  \usepackage[]{microtype}
  \UseMicrotypeSet[protrusion]{basicmath} % disable protrusion for tt fonts
}{}
\makeatletter
\@ifundefined{KOMAClassName}{% if non-KOMA class
  \IfFileExists{parskip.sty}{%
    \usepackage{parskip}
  }{% else
    \setlength{\parindent}{0pt}
    \setlength{\parskip}{6pt plus 2pt minus 1pt}}
}{% if KOMA class
  \KOMAoptions{parskip=half}}
\makeatother
\usepackage{xcolor}
\setlength{\emergencystretch}{3em} % prevent overfull lines
\providecommand{\tightlist}{%
  \setlength{\itemsep}{0pt}\setlength{\parskip}{0pt}}
\setcounter{secnumdepth}{-\maxdimen} % remove section numbering
\ifLuaTeX
\usepackage[bidi=basic]{babel}
\else
\usepackage[bidi=default]{babel}
\fi
\babelprovide[main,import]{italian}
% get rid of language-specific shorthands (see #6817):
\let\LanguageShortHands\languageshorthands
\def\languageshorthands#1{}
\usepackage{cancel}
\usepackage{steinmetz}
\usepackage{derivative}
\usepackage{mathtools}
\usepackage{siunitx}
\DeclareMathOperator{\sinc}{sinc}
\DeclareMathOperator{\rect}{rect}
\DeclareMathOperator{\tfs}{TFS}
\newcommand{\dft}{\operatorname{DFT}}
\newcommand{\tf}[1]{\text{T}\Big[ #1 \Big]}
\newcommand{\Abs}[1]{\Big| #1 \Big|}
\newcommand{\ov}[1]{\overline{#1}}
\usepackage{geometry}
    \geometry{
        a4paper,
        total={170mm,257mm},
        left=20mm,
        top=20mm,
    }
\ifLuaTeX
  \usepackage{selnolig}  % disable illegal ligatures
\fi
\usepackage{bookmark}
\IfFileExists{xurl.sty}{\usepackage{xurl}}{} % add URL line breaks if available
\urlstyle{same}
\hypersetup{
  pdflang={it},
  hidelinks,
  pdfcreator={LaTeX via pandoc}}

\author{}
\date{}

\begin{document}

\section{Lista teoremi Teoria dei
Segnali}\label{lista-teoremi-teoria-dei-segnali}

\subsection{Introduzione}\label{introduzione}

\begin{enumerate}
\def\labelenumi{\arabic{enumi}.}
\item
  \(x(t)\) segnale periodico di periodo \(T_0\) e ha potenza media su un
  intervallo finita, allora ha \(\bar{P}\) finita e calcolabile sul
  periodo:

  Per definizione
  \(\bar{P}= \frac{1}{T} \int_{-\frac{T}{2}}^{\frac{T}{2}} |x(t)|^{2} \,dt\).
  Scegliamo come periodo \(NT_0\), in quanto se \(N\to \infty\) vale
  come \(T\to \infty\).

  Quindi
  \(\displaystyle \lim_{N \to \infty} \frac{1}{NT_0} \int_{-\frac{NT_0}{2}}^{\frac{NT_0}{2}} |x(t)|^{2} \,dt\)
  equivale a \(N\) integrali cui si aggiunge un \(T_0\) ogni volta:

  \[
   \displaystyle \lim_{N \to \infty} \frac{1}{\cancel{N}T_0} \cdot \cancel{N} \Big \{ \int_{-\frac{T_0}{2}}^{\frac{T_0}{2}} |x(t)|^{2} \,dt \} = \bar{P} = \frac{1}{T_0} \int_{-\frac{T_0}{2}}^{\frac{T_0}{2}} |x(t)|^{2} \,dt
   \]
\end{enumerate}

\subsection{Segnali periodici a tempo
continuo}\label{segnali-periodici-a-tempo-continuo}

\subsubsection{Serie di Fourier}\label{serie-di-fourier}

\begin{enumerate}
\def\labelenumi{\arabic{enumi}.}
\setcounter{enumi}{1}
\item
  Da forma polare a complessa (o rettangolare)

  la forma polare della serie di Fourier è data da: \[
   x(t) = A_0 + 2 \sum_{k=1}^{\infty} A_{k} \cos (2\pi kf_{0}t + \theta_{K})
   \] \[
   = A_0 + \cancel{2} \sum_{k=1}^{\infty} A_{k} \frac{e^{j(2\pi kf_{0}t + \theta_{K})}+e^{-j(2\pi kf_{0}t + \theta_{K})}}{\cancel{2}} \to \text{ uso formula di Eulero per il coseno}
   \] \[
   = A_0 + \sum_{k=1}^{\infty} A_{k} e^{j(2\pi kf_{0}t + \theta_{k})}  + \sum_{k=1}^{\infty} A_{k} e^{-j(2\pi kf_{0}t + \theta_{K})} \to \text{ separo le due esponenziali}
   \] \[
   = x_0 + \sum_{k=1}^{\infty} A_{k} e^{j\theta_{k}} e^{j2\pi kf_{0}t}  + \sum_{k=1}^{\infty} A_{k} e^{-j\theta_{k}} e^{-j2\pi kf_{0}t}  \to \text{ raggruppo le sommatorie e} k \text{ diventa }-k
   \] \[
   = x_0 + \sum_{k=1}^{\infty} X_k e^{j2\pi kf_{0}t}  + \sum_{k=1}^{\infty} X_k e^{-j2\pi kf_{0}t}
   \] \[
   \Rightarrow x(t) = \sum_{k=-\infty}^{\infty} X_k \ e^{j2\pi kf_{0}t} \text{ forma complessa della serie di Fourier}
   \]
\item
  Come si calcolano i coefficienti \(X_n\)?

  Partendo dalla forma complessa, moltiplico a destra e a sinistra per
  \(e^{-j2\pi kf_{0}t}\), integrando sul periodo \(T_0\). \[
   \int_{-\frac{T_0}{2}}^{\frac{T_0}{2}} x(t)e^{-j2\pi kf_{0}t} \,dt = \int_{-\frac{T_0}{2}}^{\frac{T_0}{2}} \sum_{k=-\infty}^{\infty} X_k \ e^{j2\pi kf_{0}t}  e^{-j2\pi nf_{0}t} \,dt
   \] Porto fuori la sommatoria e raccolgo \(e\): per ipotesi la serie
  converge. \[
   \int_{-\frac{T_0}{2}}^{\frac{T_0}{2}} x(t)e^{-j2\pi kf_{0}t} \,dt = \sum_{k=-\infty}^{\infty} X_k  \int_{-\frac{T_0}{2}}^{\frac{T_0}{2}} e^{j2\pi (k-n)f_{0}t} \,dt
   \] L'integrale al secondo membro viene calcolato per \(k\neq n\) \[
   \to \int_{-\frac{T_0}{2}}^{\frac{T_0}{2}} e^{j2\pi (k-n)f_{0}t} = \frac{e^{j2\pi (k-n)f_{0}t}}{j2\pi (k-n)f_{0}} \Big|_{-\frac{T_0}{2}}^{\frac{T_0}{2}}=
   \] \[
   \frac{e^{j\cancel{2}\pi (k-n)\cancel{f_{0}}\frac{\cancel{T_0}}{\cancel{2}}} - e^{-j\cancel{2}\pi (k-n)\cancel{f_{0}}\frac{\cancel{T_0}}{\cancel{2}}}}{2j \cdot \pi (k-n)f_{0}} \to \text{ uso formula di Eulero per il seno}
   \] \[
   \frac{\sin (\pi (k-n))}{\pi (k-n)f_{0}} = \left\{ \begin{array}{cl}
   k=n\to T_0  \\
   k\neq n \to 0
   \end{array} \right.
   \to \text{ sostituiamo questo risultato}
   \] \[
   \int_{-\frac{T_0}{2}}^{\frac{T_0}{2}} x(t)e^{-j2\pi nf_{0}t} \,dt = X_{n}T_{0} \Rightarrow X_{n} = \frac{1}{T_0}\int_{-\frac{T_0}{2}}^{\frac{T_0}{2}} x(t)e^{-j2\pi nf_{0}t} \,dt 
   \]

  N-esimo termine della serie di Fourier
\item
  Forma rettangolare dalla forma polare \[
   x(t) = A_{0} + 2 \sum_{k=1}^{\infty} A_{k} \cos(2\pi k f_{0}t + \theta_{k})
   \] usiamo la formula di addizione
  \(\cos(\alpha + \beta) = \cos\alpha\cos\beta - \sin\alpha\sin\beta\)
  \[
   x(t) = A_{0} + 2 \sum_{k=1}^{\infty} (A_{k}(\cos(2\pi k f_{0})\cos({\theta_k}) - \sin(2\pi k f_{0})\sin({\theta_k})))
   \] \[
   x(t) = a_{0} + 2 \sum_{k=1}^{\infty} [a_{k} \cos(2\pi k f_{0}) - b_{k}\sin(2\pi k f_{0})]
   \] sapendo che
  \(a_{0} = A_{0}, \ a_{k} = A_{k}\cos(\theta_k), \ b_{k} = B_{k}\sin(\theta_k)\).

  Abbiamo quindi ottenuto la forma rettangolare della serie di Fourier,
  dove si nota che un segnale \emph{periodico} \(x(t)\) può essere
  espresso tramite una \textbf{somma di seni e coseni}.

  Il coefficiente \(X_n\) può essere espresso anche come: \[
   X_k = \frac{1}{T_0} \int_{[T_0]} x(t) (\cos(2\pi kf_{0}t) - j\sin(2\pi kf_{0}t))
   \] \[
   X_k = a_k +jb_k = A_k \cos(\theta_k) + jA_k\sin(\theta_k) = A_k e^{j\theta_k}
   \]
\item
  Criterio di Dirichlet (per \(x(t)\) periodico):

  È una serie di condizioni che se incontrate sono sufficienti per poter
  sviluppare un dato segnale \(x(t)\) in serie di Fourier:

  \begin{itemize}
  \tightlist
  \item
    \(x(t)\) deve essere \emph{assolutamente integrabile sul periodo}:
    ovvero \((\int_{[T_0]}|x(t)| \,dt < \infty)\)
  \item
    \(x(t)\) deve essere \emph{continua} (o avere un numero
    \emph{finito} di discontinuità di prima specie)
  \item
    \(x(t)\) deve essere \emph{derivabile sul periodo} \(T_0\), escluso
    al più un numero finito di punti, dove comunque esiste
    \textbf{finita} sia la derivata destra che la derivata sinistra

    \begin{itemize}
    \tightlist
    \item
      quest'ultima ipotesi è equivalente a: \(x(t)\) presenta un numero
      finito di massimi e minimi nel periodo La serie \textbf{converge}
      al valore assunto da \(x(t)\) dove \emph{continua} e alla
      semisomma dei limiti sinistro e destro se discontinua.
    \end{itemize}
  \end{itemize}
\end{enumerate}

\subsubsection{Spettro di un segnale periodico e
reale}\label{spettro-di-un-segnale-periodico-e-reale}

\paragraph{Proprietà}\label{proprietuxe0}

\begin{enumerate}
\def\labelenumi{\arabic{enumi}.}
\setcounter{enumi}{5}
\item
  Simmetria Hermitiana dello spettro reale:

  I coefficienti \(X_k\) sono generalmente quantità complesse del tipo
  \begin{gather*}
   X_k = |X_{k}|e^{j\phase{X_{k}}}
   \end{gather*} \(X_{k}\)può essere rappresentata tramite spettro di
  ampiezza e spettro di fase, discreti (esiste solo in corrispondenza
  delle armoniche \(k\)) \[
   X_k = \frac{1}{T_0} \int_{-\frac{T_0}{2}}^{\frac{T_0}{2}} x(t) e^{-j2\pi kf_{0}t} \,dt
   \] Analizziamone il coniugato \(X^{*}_{k}\): \[
   X_k^{*} = \Big(\frac{1}{T_0} \int_{-\frac{T_0}{2}}^{\frac{T_0}{2}} x(t) e^{-j2\pi kf_{0}t} \,dt \Big)^{*} = \frac{1}{T_0} \int_{-\frac{T_0}{2}}^{\frac{T_0}{2}} x(t)^{*} e^{+j2\pi kf_{0}t} \,dt = \frac{1}{T_0} \int_{-\frac{T_0}{2}}^{\frac{T_0}{2}} x(t) e^{-j2\pi (-k)f_{0}t} \,dt
   \] È da notare come \(x(t)^{*}=x(t)\), dal momento che il segnale
  \(x(t)\) è reale.

  Quindi \(X_{k}^{*}=X_{-k}\): i coefficienti \(X_k\) di un segnale
  \emph{reale} \textbf{godono di simmetria hermitiana}, ossia hanno
  \emph{lo stesso modulo e fase opposta} \[
   X_{-k}=X_{k}^{*} \Longleftrightarrow \left\{ \begin{array}{cl}
   |X_k|=|X_{-k}| \ \text{ stesso modulo}  \\
   \phase{X_{k}} = -\phase{X_{-k}} \ \text{ fase opposta}
   \end{array} \right.
   \] In definitiva per un segnale reale:

  \begin{itemize}
  \tightlist
  \item
    lo spettro d'ampiezza è \textbf{simmetrico} rispetto a \(k \to\)
    pari
  \item
    lo spettro di fase è \textbf{antisimmetrico} rispetto a \(k \to\)
    dispari
  \end{itemize}
\item
  Linearità dello spettro reale:

  Se \(x(t)\) e \(y(t)\) sono due segnali con periodo \(T_0\)
  \emph{reali} allora vale: \[
   z(t)=ax(t)+by(t) \Longleftrightarrow Z_{k} = aX_{k} + bY_{k}
   \] Somma di \emph{oscillazioni} alle (o con?) le stesse frequenze dei
  segnali \(x(t)\) e \(y(t)\). \[
   Z_{k} = \frac{1}{T_0} \int_{-\frac{T_0}{2}}^{\frac{T_0}{2}} z(t) e^{-j2\pi kf_{0}t} \,dt = \frac{1}{T_0} \int_{-\frac{T_0}{2}}^{\frac{T_0}{2}} (ax(t)+by(t)) e^{-j2\pi kf_{0}t} \,dt 
   \] \[
   = \frac{a}{T_0} \int_{-\frac{T_0}{2}}^{\frac{T_0}{2}} x(t) e^{-j2\pi kf_{0}t} \,dt + \frac{b}{T_0} \int_{-\frac{T_0}{2}}^{\frac{T_0}{2}} y(t) e^{-j2\pi kf_{0}t} \,dt = aX_{k}+bY_{k}
   \]
\item
  Parità e disparità del segnale

  \begin{itemize}
  \item
    Se \(x(t)\) è \textbf{pari}, allora il coefficiente
    \(X_{k} = X_{-k}\); se il segnale è anche \textbf{reale} vale
    \(X_{k}=X_{-k}=X^{*}_{k} \Longleftrightarrow X_k \in \mathbb{R}\).

    \(X_k = X_{-k}\) (con un cambio di variabile
    \(\alpha=-t \to \,dt=-\,d\alpha\)). \[
      X_{k} = \frac{1}{T_0} \int_{-\frac{T_0}{2}}^{\frac{T_0}{2}} x(t) e^{-j2\pi kf_{0}t} \,dt \Longleftrightarrow X_{-k} = \frac{1}{T_0} \int_{-\frac{T_0}{2}}^{\frac{T_0}{2}} x(t) e^{-j2\pi (-k)f_{0}t} \,dt 
      \] Utilizziamo il cambio di variabile \begin{gather*}
      X_{-k} = \frac{1}{T_0} \int_{\frac{T_0}{2}}^{-\frac{T_0}{2}} x(-\alpha) e^{-j2\pi (\cancel{-}(\cancel{-}\alpha)) kf_{0}} -\,d\alpha = -\frac{1}{T_0} \int_{\frac{T_0}{2}}^{-\frac{T_0}{2}} x(\alpha) e^{-j2\pi\alpha kf_{0}} \,d\alpha =
      \\ 
      \frac{1}{T_0} \int_{-\frac{T_0}{2}}^{\frac{T_0}{2}} x(\alpha) e^{-j2\pi\alpha kf_{0}} \,d\alpha = X_k
      \end{gather*} dato che il segnale \(\in \mathbb{R}\) lo possiamo
    rappresentare come (perché essendo reale ha fase nulla?): \[
      x(t) = X_{0} + 2\sum_{k=1}^{\infty}X_{k}\cos(2\pi kf_{0}t)
      \] Dimostrazione: \[
      x(t) = \sum_{k=-\infty}^{\infty} X_{k} e^{j2\pi kf_{0}t} = X_{0} + \sum_{k=1}^{\infty} X_{k} e^{j2\pi kf_{0}t} + \sum_{k=-\infty}^{-1} X_{k} e^{j2\pi kf_{0}t} =
      \] \[
      = X_{0} + \sum_{k=1}^{\infty} X_{k} e^{j2\pi kf_{0}t} + \sum_{k=1}^{\infty} X_{-k} e^{-j2\pi kf_{0}t} =
      \] \[
      = X_{0} + \sum_{k=1}^{\infty} X_{k} e^{j2\pi kf_{0}t} + \sum_{k=1}^{\infty} X_{k} e^{-j2\pi kf_{0}t} =
      \] \[
      = X_{0} + 2 \sum_{k=1}^{\infty} X_{k} \frac{e^{j2\pi kf_{0}t}+ e^{-j2\pi kf_{0}t}}{2} = X_{0} + 2 \sum_{k=1}^{\infty} X_{k} \cos{(2\pi k f_{0}t)}
      \] Da ciò deduco che un segnale reale e pari è esprimibile in
    serie di soli \emph{coseni} (i quali sono a loro volta pari).

    Possiamo inoltre scrivere i coefficienti \(X_k\) in modo
    semplificato, data la \emph{parità} del segnale: \begin{gather*}
      X_k = \frac{1}{T_0}\int_{-\frac{T_0}{2}}^{\frac{T_0}{2}} x(t) e^{-j2\pi kf_{0}t} \,dt = \\
      \frac{1}{T_0}\int_{-\frac{T_0}{2}}^{\frac{T_0}{2}} \underbrace{x(t)}_{pari}\cdot\underbrace{\cos{(2\pi kf_{0}t)}}_{pari} \,dt - \frac{j}{T_0}\int_{-\frac{T_0}{2}}^{\frac{T_0}{2}} \underbrace{x(t)}_{pari}\cdot\underbrace{\sin{(2\pi kf_{0}t)}}_{dispari}\,dt=
      \\
      \frac{2}{T_0} \int_{0}^{\frac{T_0}{2}} x(t)\cdot\cos{(2\pi kf_{0}t)} \,dt - 0
      \end{gather*} Integrale di una funzione pari su un intervallo
    simmetrico.
  \item
    se \(x(t)\) è \textbf{dispari}, allora anche i coefficienti \(X_k\)
    saranno dispari. Inoltre, dato che \(x(t)\in\mathbb{R}\), \(X_k\)
    sarà un \textbf{immaginario puro}, ed \[
      x(t)=2j \sum_{k=1}^{\infty} X_{k} \sin{(2\pi k f_{0}t)} \text{ e } X_k=-\frac{2j}{T_0}\int_{0}^{\frac{T_0}{2}}x(t)\sin(2\pi kf_0 t) \,dt
      \] Dimostrazione:

    \begin{itemize}
    \tightlist
    \item
      Dato che \(x(t)\in\mathbb{R}\), \(X_k\), allora vale
      \(X_{-k} = -X_{k} = X_{k}^{*} \Rightarrow X_{k}^{*}=-X_k\), quindi
      è un immaginario puro!
    \item
      Per \(X_k\): \[
        X_k = \frac{1}{T_0}\int_{-\frac{T_0}{2}}^{\frac{T_0}{2}} x(t) e^{-j2\pi kf_{0}t} \,dt =
        \] \begin{gather*}
        \frac{1}{T_0}\int_{-\frac{T_0}{2}}^{\frac{T_0}{2}} \underbrace{x(t)}_{\text{dispari}}\cdot\underbrace{\cos(2\pi kf_{0}t)}_{\text{pari}} \,dt - \frac{j}{T_0}\int_{-\frac{T_0}{2}}^{\frac{T_0}{2}} \underbrace{x(t)}_{\text{dispari}}\cdot\underbrace{\sin(2\pi kf_{0}t)}_{\text{dispari}}\,dt=
        \\
        -\frac{2j}{T_0} \int_{0}^{\frac{T_0}{2}} x(t)\cdot\sin{(2\pi kf_{0}t)} \,dt 
        \end{gather*}
    \end{itemize}
  \item
    \textbf{Note varie}

    \begin{itemize}
    \tightlist
    \item
      Se \(x(t)\) è pari i suoi coefficienti \(X_k\) sono reali e lo
      spettro di fase vale \(0\) o \(\pm \pi\); mentre se \(x(t)\) è
      dispari i suoi coefficienti \(X_k\) sono immaginari puri e lo
      spettro di ampiezza non viene toccato: un segnale dispari è solo
      ``spostato'' nel tempo.
    \item
      È da notare come la diversa velocità di un segnale dipenda dal suo
      andamento temporale: le variazioni brusche comportano la presenza
      di \textbf{armoniche}{[}\^{}1{]} con \(k\) più elevato per
      rappresentare la velocimento alta(?):

      \begin{itemize}
      \tightlist
      \item
        più il segnale è regolare meno armoniche sono necessarie per
        ``ricreare'' il segnale

        \begin{itemize}
        \tightlist
        \item
          \(\frac{1}{k} \to\) funzioni discontinue: dente di sega
          ideale, onda quadra, onda quadra ``antisimmetrica'', rect
        \item
          \(\frac{1}{k^2} \to\) funzioni continue a derivata
          discontinua: onda triangolare. {[}\^{}1{]}: TODO: definire
          meglio armoniche
        \end{itemize}
      \end{itemize}
    \end{itemize}
  \end{itemize}
\end{enumerate}

\subsection{Segnali aperiodici a tempo
continuo}\label{segnali-aperiodici-a-tempo-continuo}

\subsubsection{Trasformata continua di
Fourier}\label{trasformata-continua-di-fourier}

\begin{quote}
Una funzione non periodica, definita tra \(-\infty\) e \(\infty\), può
essere rappresentata come \textbf{somma} di \textbf{infinite armoniche
semplici} di ampiezza \emph{infinitesima} e di frequenza variabile con
continuità tra \(-\infty\) e \(\infty\)
\end{quote}

\begin{enumerate}
\def\labelenumi{\arabic{enumi}.}
\setcounter{enumi}{8}
\item
  Dal segnale periodico al segnale aperiodico\ldots{}

  Partiamo dall'impulso rettangolare \emph{aperiodico}
  \(\mathop{\mathrm{rect}}{\frac{t}{T}}\): \begin{gather*}
   x(t) = \mathop{\mathrm{rect}}{\frac{t}{T}} \to x_{p}(t) = \sum \mathop{\mathrm{rect}}(\frac{t-nT_0}{T}) \text{ treno di impulsi rettangolari}
   \end{gather*} possiamo vedere \(x(t)\) come caso limite di \(x_p(t)\)
  con periodo \(T_0 \to \infty\) \[
   x(t) = \lim_{T_0 \to \infty} x_{p}(t)
   \]

  \begin{enumerate}
  \def\labelenumii{\arabic{enumii}.}
  \tightlist
  \item
    la frequenza diventa infinitesima \((f_0 = \frac{1}{T_0})\)
  \item
    si riduce la \emph{distanza tra le armoniche}, ossia si
    \textbf{infittisce} lo spettro;
  \item
    \(X_k=\frac{1}{T_0}\int_{-\frac{T_0}{2}}^{\frac{T_0}{2}} x_p(t) \ e^{-j2\pi kf_0 t}\,dt\),
    l'ampiezza assume valori \emph{sempre più piccoli}
  \end{enumerate}

  Usiamo il coefficiente \emph{modificato} \(X(f_0 k) = T_0 X_k\) per
  ovviare il problema. Riscriviamo \(x_p(t)\) e \(X_k\)\\
  \begin{gather*}
   x_{p}(t) = \sum_{k=-\infty}^{\infty}X(kf_0)\ e^{j2\pi kf_0 t} \cdot f_0 \to x(t) = \underbrace{\int_{-\infty}^{\infty}X(f)\ e^{j2\pi ft}\,df}_{\text{integrale di Fourier}}
   \end{gather*} Le armoniche si \emph{infittiscono talmente tanto} da
  non essere più distinte ma \textbf{continue}.

  \begin{gather*}
   X(kf_0) = T_0 \ X_k = \int_{-\frac{T_0}{2}}^{\frac{T_0}{2}}  x_{p}(t) \ e^{-j2\pi kf_0 t}\,dt \to \underbrace{X(f) = \int_{-\infty}^{\infty}x(t)\ e^{j2\pi ft}\,dt}_{\text{trasformata continua di Fourier}}
   \end{gather*}

  \(X(f)\) è una \textbf{funzione complessa della variabile continua
  \(f\)}, quindi è di spettro continuo.

  \begin{itemize}
  \tightlist
  \item
    Nota: \textbf{differenze tra segnali continui periodici e
    aperiodici}:

    \begin{itemize}
    \tightlist
    \item
      un segnale \emph{periodico} è rappresentato da componenti
      sinusoidali a frequenze in relazione \textbf{armonica} (multipli
      di \(f_0\), frequenza \emph{fondamentale} e ad ampiezza finita).
    \item
      un segnale \emph{aperiodico} è rappresentato con componenti
      sinusoidali di ampiezza \emph{infinitesima} \(|X(f)|\,df\) e
      frequenza \(f\) variabile con continuità su \(\mathbb{R}\); è un
      segnale periodico di periodo illimitato con \(f_0\) infinitesimo.
      Le armoniche discrete \emph{degenerano} nell'insieme continuo.
    \end{itemize}
  \end{itemize}
\item
  Criteri di esistenza per la trasformata continua di Fourier (TCF)

  \begin{enumerate}
  \def\labelenumii{\arabic{enumii}.}
  \tightlist
  \item
    \(X(f)\) esiste se il segnale \(x(t)\) ha energia finita (condizione
    ``sufficiente'')!
  \item
    Criteri di Dirichlet:

    \begin{enumerate}
    \def\labelenumiii{\arabic{enumiii}.}
    \tightlist
    \item
      la funzione deve essere assolutamente sommabile:
      \(\displaystyle \int_{-\infty}^{\infty} |x(t)| dt < +\infty\)
    \item
      se in qualunque intervallo finito \(t_1 < t < t_2\) è continua o
      presenta un numero finito di discontinuità di prima specie
    \item
      se in qualunque intervallo finito \(t_1 < t < t_2\) la funzione ha
      un numero finito di massimi e minimi.
    \end{enumerate}
  \end{enumerate}

  Allora \(x(t)\) è rappresentabile come TCF e \[
  x(t) = \int_{-\infty}^{\infty}X(f)\ e^{j2\pi ft} \,df =  \left\{ \begin{array}{cl}
  x(t) \ \text{ se continua}  \\
  \frac{x(t_{0}^{+})-x(t_{0}^{-})}{2}\ \text{ se discontinua}
  \end{array} \right.
  \]
\item
  Simmetria Hermitiana della trasformata continua di Fourier

  Possiamo rappresentare \(X(f)\) in forma rettangolare: \[
  X(f) = Re(f) + Im(f) = \int_{-\infty}^\infty x(t) \cos(2\pi ft) \,dt - j\int_{-\infty}^\infty x(t) \sin(2\pi ft) \,dt
  \] \[
  \underbrace{Re(f)=Re(-f)}_{\text{pari}} \text{ e } \underbrace{Im(f)=-Im(-f)}_{\text{dispari}} \Longrightarrow X(f)=X^{*}(-f) \text{ simmetria hermitiana}
  \] infatti \(X(f)=Re(f)+jIm(f)=Re(-f)+jIm(f)=X^{*}(-f)\)

  \begin{itemize}
  \tightlist
  \item
    lo spettro di ampiezza è quindi \emph{pari} a quello di fase
    dispari.
  \end{itemize}
\item
  Parità e disparità:

  \begin{itemize}
  \tightlist
  \item
    se un segnale è \emph{reale e pari} \[
    X(f) = \int_{-\infty}^\infty x(t) \cos(2\pi ft) \,df=
    \left\{ \begin{array}{cl}
    Re(f) = 2\int_{0}^\infty x(t) \cos(2\pi ft) \,dt \\
    Im(f) = 0
    \end{array} \right.
    \] \(\to X(f) = Re(f) \to X(f)=X(-f)\) è reale e pari
  \item
    se un segnale è \emph{dispari e reale} \[
    X(f) = - \int_{-\infty}^\infty x(t) \sin(2\pi ft) \,dt=
    \left\{ \begin{array}{cl}
    Re(f) = 0 \\
    Im(f) = -2\int_{0}^\infty x(t) \sin(2\pi ft) 
    \end{array} \right.
    \] \(\to X(f)=jIm(f) \to X(f)=-X(f)\) è immaginaria pura e dispari
  \end{itemize}
\end{enumerate}

\paragraph{Proprietà della trasformata
continua}\label{proprietuxe0-della-trasformata-continua}

\begin{enumerate}
\def\labelenumi{\arabic{enumi}.}
\setcounter{enumi}{12}
\item
  \textbf{Linearità}

  Dati due segnali \(x_1(t)\) e \(x_2(t)\) con le loro trasformate
  continue di Fourier \(X_1(f)\) e \(X_2(f)\), allora se: \[
  x(t) = ax_1(t) + bx_2(t) \Longleftrightarrow X(f)=aX_1(f)+bX_2(f)
  \] con \(a,b\) costanti, \(X_1(f)=\text{TCF}[x_1(t)]\) e
  \(X_2(f)=\text{TCF}[x_2(t)]\)

  \begin{itemize}
  \tightlist
  \item
    Dimostrazione: \[
      X(f)= \int_{-\infty}^{\infty}x(t)\ e^{-j2\pi ft}\,dt = \int_{-\infty}^{\infty}(ax_1(t) + bx_2(t)) \ e^{-j2\pi ft}\,dt
      \] ma sappiamo che l'integrale è \emph{lineare}, quindi \[
      = a\int_{-\infty}^{\infty}x_1(t)\ e^{-j2\pi ft}\,dt + b\int_{-\infty}^{\infty}x_2(t)\ e^{-j2\pi ft}\,dt = aX_1(f)+bX_2(f)
      \]
  \end{itemize}
\item
  \textbf{Dualità}

  se \(x(t)\Longleftrightarrow X(f)\), allora
  \(X(t)\Longleftrightarrow x(-f)\):

  Se la trasformata continua di Fourier passa ad essere un \emph{segnale
  nel tempo}, allora \(x(-f)\) è la sua trasformata di Fourier. Abbiamo
  quindi una corrispondenza biunivoca tra la funzione e la sua
  trasformata.

  \begin{itemize}
  \item
    Esempio: \[
      \mathop{\mathrm{rect}}(\frac{t}{T})\Longleftrightarrow\mathop{\mathrm{sinc}}(fT)
      \] Ma se nel tempo ho un segnale \(\mathop{\mathrm{sinc}}(bT)\)
    qual è la sua trasformata?

    \(T\mathop{\mathrm{sinc}}(Tt) \Longleftrightarrow \mathop{\mathrm{rect}}({-\frac{f}{T}})\)
    da cui
    \(\mathop{\mathrm{sinc}}(Bt)\Longleftrightarrow \frac{1}{B}\mathop{\mathrm{rect}}(\frac{t}{B})\),
    dove \(B\) indica la banda.
  \item
    Dimostrazione: \[
      x(t) =\int_{-\infty}^{\infty}X(f)\ e^{j2\pi ft} \,df \to x(f) = \int_{-\infty}^{\infty}X(t)\ e^{j2\pi ft} \,dt 
      \] con uno scambio di variabili \(t\) con \(f\). Quindi: \[
      x(-f)=\int_{-\infty}^{\infty}<S-Del>Xx(t)\ e^{-j2\pi ft} \,dt 
      \] Da qui deriviamo che \(x(-f)\) è la trasformata di \(X(t)\)
  \end{itemize}
\item
  \textbf{Ritardo}

  Sia \(X(f)=\text{TCF}[x(t)]\): la trasformata di Fourier di \(x(t)\)
  ritardato nel tempo di una quantità \(t_0\) è pari a: \[
  x(t-t_0) \Longleftrightarrow X(f) \ e^{-j2\pi ft_0}
  \]

  \begin{itemize}
  \tightlist
  \item
    Dimostrazione:
  \end{itemize}

  Applichiamo a \(x(t-t_0)\) la definizione di TCF \[
  x(t-t_0) \Longleftrightarrow \int_{-\infty}^{\infty} x(t-t_0) \ e^{-j2\pi ft} \,dt = \text{ sostituiamo } \Big\{\alpha = t-t_0 \to t=\alpha +t_0, \,dt = \,d\alpha
  \] \[
  x(t-t_0) \Longleftrightarrow \int_{-\infty}^{\infty} x(\alpha) e^{-j2\pi (\alpha +t_0)f} \,d\alpha = e^{-j2\pi ft_0} \int_{-\infty}^{\infty} x(\alpha)\ e^{-j2\pi f\alpha} =  e^{-j2\pi ft_0} \ X(f) 
  \]

  \begin{itemize}
  \tightlist
  \item
    Esempio: \[
    A\mathop{\mathrm{rect}}(\frac{t-\frac{T}{2}}{T}) \Longleftrightarrow AT\mathop{\mathrm{sinc}}(fT)e^{-j\cancel{2}\pi f\frac{T}{\cancel{2}}}
    \]
  \end{itemize}

  Se
  \(y(t)=x(t-t_0) \Rightarrow Y(f) = X(f) \ e^{-j2 pi ft_0} \Rightarrow\)
  Un ritardo modifica lo spettro di \textbf{fase} ma \emph{non cambia}
  il suo spettro di ampiezza, in quanto quest'ultimo di indica quali
  componenti sinusoidali sono necessarie per comporre la forma del
  segnale, mentre lo spettro di fase mi dice con quale \emph{angolo}
  iniziale devono ``partire'' le sinusoidi.

  Quindi se il segnale si sposta nel tempo, allora le sinusoidi hanno
  angoli iniziali diversi, ma sono le stesse. \begin{gather*}
  |Y(f)| = |X(f)|\cdot |e^{-j2\pi ft_0}| = |X(f)| \\
  \phase{Y(f)} = \phase{X(f) \ e^{-j2 pi ft_0}} = \phase{X(f)} + \phase{e^{-j2 pi ft_0}} = \underbrace{\phase{X(f)} - \overbrace{2\pi ft_0}^{=0}}_{\text{\underline{NON} è una traslazione!}}
  \end{gather*}
\item
  \textbf{Cambiamento di scala}

  Si consideri \(y(t)=x(\alpha t)\), effettuando un \emph{cambiamento
  della scala temporale}: \[
  \begin{array}{cl}
  |\alpha | > 1 \ \to \text{ compressione della scala dei tempi} \to \text{ l'evoluzione è "accelerata"}\\
  |\alpha | < 1 \ \to \text{ dilatazione della scala dei tempi} \to \text{ l'evoluzione è "rallentata"}\\
  \alpha  < 0 \ \to \text{ inversione della scala dei tempi}
  \end{array} 
  \] Inoltre vale: \[
  x(\alpha t) \Longleftrightarrow \frac{1}{|\alpha |} X(\frac{f}{\alpha})
  \]

  \begin{itemize}
  \tightlist
  \item
    Dimostrazione: \[
    \cdot \ \ \underline{\alpha > 0} \Rightarrow x(\alpha t) \Longleftrightarrow \int_{-\infty}^{\infty} x(\alpha t) e^{-j2\pi ft} \,dt \text{, ponendo } z=\alpha t \to t = \frac{z}{\alpha}, \,dz = \alpha \,dt \to \,dt = \frac{\,dz}{\alpha} 
    \] \[
    \Rightarrow x(\alpha t) \Longleftrightarrow \int_{-\infty}^{\infty} \frac{x(z) e^{-j2\pi f \frac{z}{\alpha}}}{\alpha} \,dz = \frac{1}{\alpha}\int_{-\infty}^{\infty} x(z) e^{-j2\pi f \frac{z}{\alpha}} \,dz = \frac{1}{\alpha} X(\frac{f}{\alpha})
    \] \[
    \cdot \ \ \underline{\alpha < 0} \Rightarrow x(\alpha t) \Longleftrightarrow \int_{\infty}^{-\infty} \frac{x(z) e^{-j2\pi f \frac{z}{\alpha}}}{\alpha} \,dz = -\frac{1}{\alpha}\int_{-\infty}^{\infty} x(z) e^{-j2\pi f \frac{z}{\alpha}} \,dz = -\frac{1}{\alpha} X(\frac{f}{\alpha})
    \] È da notare come l'inversione dell'integrale nel secondo caso
    l'abbiamo quando \(t \to -\infty, \ z \to +\infty\). Inoltre abbiamo
    sostituito \(z=-\alpha t\).
  \end{itemize}

  Quindi una \emph{dilatazione} nel tempo corrisponde ad una
  \emph{compressione} in frequenza, e \textbf{viceversa}
\item
  \textbf{Modulazione}

  Dato un segnale \(x(t)\) e la sua trasformata \(X(f)\) allora \[
  x(t)\cos(2\pi f_{0}t) \Longleftrightarrow \frac{X(f-f_0)+X(f+f_0)}{2}
  \] dove \(X(f-f_0)\) e \(X(f+f_0)\) sono rispettivamente la replica
  centrata in \(f_0\) e la replica centrata in \(-f_0\).

  \begin{itemize}
  \tightlist
  \item
    Dimostrazione: \begin{gather*}
    \text{TCF}[x(t)\cos(2\pi f_{0}t)] = \int_{-\infty}^{\infty} x(t)\cos(2\pi f_{0}t)  e^{-j2\pi ft} \,dt = 
    \\
    = \frac{1}{2} \int_{-\infty}^{\infty} x(t) [e^{-j2\pi f_0 t}+ e^{-j2\pi f_0 t}]e^{-j2\pi ft} \,dt =\\ \frac{1}{2}\Big[\int_{-\infty}^{\infty}x(t)  e^{-j2\pi (f-f_0)t}\,dt + \int_{-\infty}^{\infty}x(t)  e^{-j2\pi (f+f_0)t}\,dt  \Big] =
    \\
    \frac{X(f-f_0)+X(f+f_0)}{2}
    \end{gather*} Corollario:
    \(x(t)e^{j2\pi f_{0}t} \Longleftrightarrow X(f-f_0) \to\)
    \emph{traslazione in frequenza}
  \end{itemize}
\item
  \textbf{Derivazione}

  Se \(x(t) \to X(f)\), allora: \[
  \odv{}{t} x(t) \Longleftrightarrow j2\pi f \cdot X(f) = Y(f)
  \] Una derivata nel tempo è una \emph{moltiplicazione} in frequenza.

  \begin{itemize}
  \tightlist
  \item
    Dimostrazione:
  \end{itemize}

  Deriviamo entrambi i lati di \(x(t)\): \begin{gather*}
  \odv{}{t}x(t) = \odv{}{t} \int_{-\infty}^{\infty} X(f) e^{j2\pi ft} \,df = \int_{-\infty}^{\infty}\odv{}{t} \Big [X(f) e^{j2\pi ft} \Big ] \,df = \int_{-\infty}^{\infty} X(f) \odv{}{t}e^{j2\pi ft} \,df = 
  \\
  \int_{-\infty}^{\infty} X(f) (2\pi f) e^{j2\pi ft} \,df \Longrightarrow \text{TCF}\Big[\odv{x(t)}{t}\Big] = j2\pi f X(f)
  \end{gather*}

  Il teorema della derivazione \emph{modifica gli spettri}
  \begin{gather*}
  |Y(f)| = 2\pi f |X(f)| \\ \phase{Y(f)} = \phase{X(f)} + \text{sgn}(f)\frac{\pi}{2}
  \end{gather*} Aumenta proporzionalmente l'ampiezza, esaltando le altre
  frequenze, e sfasando di \(\pm \frac{\pi}{2}\)
\item
  \textbf{Integrazione} (deriva dal teorema di derivazione)

  Dato un segnale \(x(t) \Longleftrightarrow X(f)\) e un segnale
  \(y(t) = \int_{-\infty}^{t} x(\alpha) \,d\alpha\), allora vale \[
  \int_{-\infty}^{t}  x(\alpha) \,d\alpha \Longleftrightarrow \frac{X(f)}{j2\pi f}
  \]

  \begin{itemize}
  \tightlist
  \item
    Dimostrazione:
  \end{itemize}

  Segue dal teorema di derivazione e richiede che \(X(0)=0\), al fine di
  evitare che per \(f\to 0\), il rapporto tenda ad infinito.
  \begin{gather*}
  X(0)= 0 \longleftrightarrow X(0)=\underbrace{\int_{-\infty}^{\infty}x(t) \ e^{0}\,dt}_{\text{sottende area} \textbf{ nulla}} \longleftrightarrow y(\infty) = \int_{-\infty}^{\infty} x(t)\,dt = X(0) \to 0 \\ y(t) = \int_{-\infty}^{t} x(\alpha) \,d\alpha \Rightarrow x(t) \odv{}{t}y(t) \Rightarrow X(f)=j2\pi f\cdot Y(f) \Rightarrow Y(f) = \frac{X(f)}{j2\pi f}
  \end{gather*} Anche l'integrale nel tempo si trasforma in
  un'operazione algebrica in frequenza: in questo caso però vengono
  esaltate le componenti a \textbf{bassa} frequenza nello spettro del
  segnale, mentre le alte vengono attenuate; la fase varia sempre di
  \(\pm \frac{\pi}{2}\) \begin{gather*}
  |Y(f)| = \frac{|X(f)|}{2\pi f} \\ \phase{Y(f)} = \phase{X(f)} + \text{sgn}(f)\frac{\pi}{2}
  \end{gather*} Da questo teorema deriva la relazione
  \(A \text{tri}(\frac{t}{T})\Longleftrightarrow AT\mathop{\mathrm{sinc}}^{2}(fT); \ A\mathop{\mathrm{rect}}(\frac{t}{T})\Longleftrightarrow AT\mathop{\mathrm{sinc}}(fT)\)
\item
  \textbf{Prodotto}: è il duale della convoluzione

  Partendo da due segnali \(x(t)\) e \(y(t)\) \[
  z(t)=x(t)\cdot y(t) \Longleftrightarrow X(f) \otimes Y(f)
  \]

  \begin{itemize}
  \tightlist
  \item
    Dimostrazione: \begin{gather*}
    \Rightarrow Z(f) =  \int_{-\infty}^{\infty}x(t)\ y(t)\ e^{-j2\pi ft}\,dt = \int_{t=-\infty}^{\infty} \Big[ \int_{\nu = -\infty}^{\infty} X(\nu) e^{-j2\pi \nu t} \,d\nu \Big ] y(t)\ e^{-j2\pi ft}\,dt= \\ 
    \int_{\nu=-\infty}^{\infty} X(\nu) \Big[ \int_{t = -\infty}^{\infty}  y(t)\ e^{-j2\pi (f-\nu)t}\,dt \Big ] \,d\nu =  \int_{\nu = -\infty}^{\infty} X(\nu) Y(f-\nu) \,d\nu =
    \\ X(f) \otimes Y(f)
    \end{gather*} Quindi: \[
    \underset{PRODOTTO}{x(t) \ y(t)} \Longleftrightarrow \underset{CONVOLUZIONE}{X(f)\otimes Y(f)} \to \text{ la convoluzione è \textit{commutativa}}
    \] Nota Bene: \(\nu\) è \textbf{nu}!
  \end{itemize}
\item
  \textbf{\emph{Convoluzione}}

  Dati due segnali \(x(t)\) e \(y(t)\) sappiamo che: \[
  z(t) = x(t) \otimes y(t) \Longleftrightarrow X(f) \ Y(f)
  \]

  \begin{itemize}
  \item
    Dimostrazione:

    Partiamo sempre dalla definizione di TCF: \begin{gather*}
    z(t) = x(t) \otimes y(t) = \int_{-\infty}^{\infty} x(\alpha) y(t-\alpha) \,d\alpha \Longleftrightarrow Z(f) = \int_{-\infty}^{\infty} z(t) \ e^{-j2\pi ft} \,dt = \\  \int_{-\infty}^{\infty} \int_{-\infty}^{\infty} \Big [ x(\alpha) y(t -\alpha) \,d\alpha  \Big] e^{-j2\pi f(t-\alpha+\alpha)} \,dt =\\
    \int_{\alpha=-\infty}^{\infty} x(\alpha) \underbrace{\Big [\int_{t=-\infty}^{\infty}y(t-\alpha)e^{-j2\pi f(t-\alpha)}\,dt \Big ]}_{Y(f)}  e^{-j2\pi f\alpha} \,d\alpha = \\ \int_{\alpha=-\infty}^{\infty} x(\alpha) \ Y(f) \ e^{-j2\pi f\alpha}\,d\alpha = X(f) \ Y(f)
    \end{gather*}
  \item
    Nota bene:

    \begin{itemize}
    \tightlist
    \item
      la convoluzione ha proprietà commutativa, associativa e
      distributiva.
    \end{itemize}
  \end{itemize}
\end{enumerate}

\subsection{Trasformata di Fourier
generalizzata}\label{trasformata-di-fourier-generalizzata}

\begin{enumerate}
\def\labelenumi{\arabic{enumi}.}
\setcounter{enumi}{21}
\item
  Teorema d'integrazione \textbf{completo}:

  Vogliamo rimuovere il vincolo (o ipotesi) \(X(0)\) che è alla base
  dell'applicabilità del teorema d'integrazione ``incompleto'': ciò
  viene realizzato utilizzando la delta di Dirac.

  Il teorema completo afferma che: \[
  y(t) = \int_{-\infty}^{t} x(\alpha)\,d\alpha \Longleftrightarrow Y(f) = \frac{X(f)}{j2\pi f} + \frac{\delta(f)}{2}\cdot X(0)  
  \] Il nuovo termine rende conto dell'eventuale valor medio diverso da
  zero del segnale!.

  \begin{itemize}
  \item
    Dimostrazione:

    Essendo: \begin{gather*}
      x(t)\otimes u(t) = \int_{-\infty}^{\infty} x(\alpha) \ u(t-\alpha)\,d\alpha = \int_{-\infty}^{t} x(\alpha)\,d\alpha
      \\
      u(t) = \frac{1}{2}\text{sgn}(t)+\frac{1}{2}
      \end{gather*} abbiamo che per la convoluzione
    \(x(t)\otimes u(t)\Longleftrightarrow X(f)U(f)\): \[
      X(f)\ U(f)=X(f)\Big[\frac{1}{j2\pi f}+\frac{\delta(f)}{2}\Big] = \frac{X(f)}{j2\pi f}+ \frac{X(0)}{2}\delta(f)
      \] Questo perché
    \(\text{TCF}(u(t))=U(f)=\frac{1}{j2\pi f}+\frac{1}{2}\delta (f)\);
    l'ultimo termine scompare per segnali ad area nulla: rende conto
    dell'eventuale valor medio diverso da zero del segnale, ed è un
    termine correttivo che rappresenta la funzione impulsiva.
  \end{itemize}
\item
  Teorema della modulazione, alternativa:

  \begin{itemize}
  \item
    Dimostrazione:

    per il teorema del prodotto, \begin{gather*}
      x(t)\cos(2\pi f_{0}t) \Longleftrightarrow X(f) \otimes \Big[ \frac{\delta(f-f_0)+\delta(f+f_0)}{2} \Big] = \\ X(f) \otimes \frac{\delta(f-f_0)}{2} + X(f) \otimes  \frac{\delta(f+f_0)}{2} \\
      \to X(f)\otimes \delta(f-f_0) = \int_{\mathbb{R}}X(\alpha) \delta(f-f_0 -\alpha)\,d\alpha = \int_{\mathbb{R}}X(\alpha) \delta(\alpha) -(f-f_0)\,d\alpha = X(f-f_0) \\
      x(t)\cos(2\pi f_0 t) \Longleftrightarrow \frac{X(f-f_0)+X(f+f_0)}{2}
      \end{gather*}
  \end{itemize}
\end{enumerate}

\subsection{Periodicizzazione}\label{periodicizzazione}

\begin{enumerate}
\def\labelenumi{\arabic{enumi}.}
\setcounter{enumi}{23}
\item
  Prima formula della somma di Poisson:

  Come rendere un segnale \emph{aperiodico} \(x(t)\) \textbf{periodico}
  di periodo \(T_0\). Partiamo da
  \(y(t)=\sum_{n=-\infty}^{\infty} = x(t-nT_0)\) relazione nel tempo tra
  periodico e aperiodico \begin{gather*}
  \to Y_k = \frac{1}{T_0} = \int_{-\frac{T}{2}}^{\frac{T}{2}} y(t) \ e^{-j2 \pi kf_{0}t} \,dt = \frac{1}{T_0} = \int_{-\frac{T}{2}}^{\frac{T}{2}} \sum_{n=-\infty}^{\infty} x(t-nT_0) e^{-j2\pi kf_{0}t} \,dt \\
  \sum_{n=-\infty}^{\infty}\frac{1}{T_0} \int_{-\frac{T}{2}}^{\frac{T}{2}} x(t-nT_0)e^{-j2\pi kf_{0}t} \,dt = \ \ \ \text{ sostituiamo } \left\{ \begin{array}{cl} \alpha = t-t_0 \\ t = \alpha -t_0 \\ \,d\alpha = \,dt \end{array}\right.  \\
  \frac{1}{T_0} \sum_{n=-\infty}^{\infty}\frac{1}{T_0} \int_{-\frac{T}{2}-nT_0}^{\frac{T}{2}-nT_0} x(\alpha) \ e^{-j2\pi kf_{0}(\alpha +nT_0)} \,d\alpha = \\
  \frac{1}{T_0} \sum_{n=-\infty}^{\infty}\frac{1}{T_0} \int_{-\frac{T}{2}-nT_0}^{\frac{T}{2}-nT_0} x(\alpha) e^{-j2\pi kf_{0}\alpha} \cdot \underbrace{e^{-j2\pi k\cancel{f_0}n\cancel{T_0}}}_{\text{multiplo di }2\pi \to e^{0}} \,d\alpha= \\
  \frac{1}{T_0} \sum_{n=-\infty}^{\infty}\frac{1}{T_0} \int_{-\frac{T}{2}-nT_0}^{\frac{T}{2}-nT_0} x(\alpha) \ e^{-j2\pi kf_{0}\alpha} \,d\alpha =
  \frac{1}{T_0} \int_{-\infty}^{\infty}x(\alpha) \ e^{-j2\pi kf_{0}\alpha}\,d\alpha = \underbrace{\frac{1}{T_0}X(kf_0)}_{\text{campionamento in frequenza}}
  \end{gather*} Si ottiene una relazione detta \textbf{campionamento in
  frequenza}. I coefficienti della serie di Fourier del segnale
  periodico \(y(t)\) sono, a meno del fattore \(\frac{1}{T_0}\), i
  campioni della TCF del \emph{segnale base} \(x(t)\) presi in
  corrispondenza delle frequenze armoniche \(kf_0\) \begin{gather*}
  \to \sum_{n=-\infty}^{\infty}x(t-nT_0) = \sum_{k=-\infty}^{\infty} \frac{1}{T_0} X(\frac{k}{T_0})\ e^{+j2\pi kt f_0}
  \end{gather*}
\item
  Seconda formula della somma di Poisson

  Applichiamo alla prima formula di Poisson il teorema della dualità:
  \begin{gather*}
  X(t) \longleftrightarrow x(-f) \\
  x(t) \longleftrightarrow X(f)
  \end{gather*} \begin{gather*}
  \sum_{n=-\infty}^{\infty}x(t-nT_0) = \sum_{k=-\infty}^{\infty} \frac{1}{T_0} X(\frac{k}{T_0})\ e^{+\frac{j2\pi kt}{T_0}} \\
  \sum_{n=-\infty}^{\infty}X(t-nT_0) = \sum_{k=-\infty}^{\infty} \frac{1}{T_0}(x(-\frac{k}{T_0}))\ e^{+\frac{j2\pi kt}{T_0}} \\
  \Rightarrow \sum_{n=-\infty}^{\infty}X(t-nT_0) = \sum_{k=-\infty}^{\infty} \frac{1}{T_0} x(\frac{k}{T_0})\ e^{-\frac{j2\pi kt}{T_0}} \text{ cambio di segno all'indice k} \\
  \to T = \frac{1}{T_0} \Rightarrow \sum_{n=-\infty}^{\infty}X(t-\frac{n}{T}) = T\sum_{k=-\infty}^{\infty} \frac{1}{T_0} x(kT)\ e^{-j2\pi ktT}
  \end{gather*} Adesso, dal punto di vista puramente formale, cambiano
  nome da \(t\) in \(f\), otteniamo un'espressione, otteniamo
  un'espressione \emph{duale} rispetto alla prima formula di Poisson \[
  \sum_{n=-\infty}^{\infty}x(nT)e^{-j2\pi fT} = \frac{1}{T}\sum_{k=-\infty}^{\infty}X(f-\frac{k}{T})
  \]
\end{enumerate}

\subsection{Sistemi}\label{sistemi}

\begin{enumerate}
\def\labelenumi{\arabic{enumi}.}
\setcounter{enumi}{25}
\tightlist
\item
  Teorema di Parseval:
\end{enumerate}

Dato un segnale \(x(t)\) e la sua energia
\(E_{x}=\int_{-\infty}^{\infty} |x(t)|^2 \,dt < +\infty\) (energia
finita), possiamo esprimere l'energia \(E_x\) \emph{anche in frequenza}:
\begin{gather*}
E_{x}=\int_{-\infty}^{\infty} |x(t)|^2 \,dt = \int_{-\infty}^{\infty} x(t) \ x^{\ast} \,dt = \int_{-\infty}^{\infty} x(t) \Big[ \int_{-\infty}^{\infty} X^{*}(f) e^{-j2\pi ft} \,df \Big] \,dt \\
\int_{f=-\infty}^{\infty} X^{\star}(f) \Big [\int_{t=-\infty}^{\infty} x(t) e^{-j2\pi ft} \,dt\Big ] \,df = \int_{-\infty}^{\infty} X^{*}(f) = \int_{-\infty}^{\infty} |X(f)|^{2} \,df
\end{gather*} \(E_x\) è l'energia totale, deriva da \(p_x = |x(t)|^2\)
potenza istantanea integrata o da \(|X(f)|^2\) detta \textbf{densità
spettrale} \(E_x(f)\) integrata.

\begin{enumerate}
\def\labelenumi{\arabic{enumi}.}
\setcounter{enumi}{26}
\tightlist
\item
  Teorema di Wiener-Khinchin
\end{enumerate}

Siamo la densità spettrale di potenza: \[
P_x = \lim_{T\to \infty} \frac{1}{T}\int_{-\infty}^{\infty} x(t) \,dt 
\] e la funzione \emph{densità spettrale di potenza} \[
S_x(f) \triangleq \lim_{T\to \infty} \frac{E_{x_T}(f)}{T} =\lim_{T\to\infty} \frac{|x(t)|^2}{T} 
\] con \(E_{x_T}(f)\) densità di energia del segnale \emph{troncato}
nell'intervallo \([-\frac{T}{2}; \frac{T}{2}]\)

Definiamo \textbf{funzione di autocorrelazione}
\(R_x(\tau)= \int_{-\infty}^{\infty}x(\tau)x(t-\tau)\,dt\) ossia il
segnale moltiplicato per una sua replica \emph{ritardata}. Indica
``quanto il segnale somiglia alla sua replica ritardata'': più \(x(t)\)
è compatta meno somiglierà e meno varrà \(R_x(\tau)\)

Il teorema afferma che la densità spettrale di energia di un segnale
coincide con la trasformata di Fourier della funzione di
autocorrelazione del segnale stesso:

\begin{gather*}
E_x(f)= \int_{-\infty}^{\infty}R_{x}(\tau) e^{-j2\pi ft}\,d\tau \underbrace{=}_{R_x(\tau) \text{ è pari})} 2\int_{0}^{\infty} \cos(2\pi f\tau)R_x(\tau) \,d\tau
\end{gather*}

\begin{itemize}
\tightlist
\item
  Dimostrazione:
\end{itemize}

Partiamo dalla definizione di autocorrelazione:

\begin{gather*}
R_{x} (\tau) = \int_{-\infty}^{\infty} x(\alpha) x(\alpha -t) \,d\alpha = \int_{-\infty}^{\infty} x(\alpha)x(-(t-\alpha)) \,d\alpha = x(\tau) \otimes x(-\tau) = \\
R_{x} (\tau) =  x(\tau) \otimes x(-\tau) \Longleftrightarrow X(f) \ X(-f) = X(f) \ X^{*}(-f) = |X(f)|^{2} = E_x(f)
\end{gather*}

\newpage

\section{Secondo Parziale}\label{secondo-parziale}

\subsection{Processi aleatori
analogici}\label{processi-aleatori-analogici}

\begin{enumerate}
\def\labelenumi{\arabic{enumi}.}
\item
  Un processo aleatorio WSS \(X(t)\) filtrato da un SLS è all'uscita un
  nuovo processo \(Y(t)\) WSS.

  Per far sì che accada il processo \(y(t)\) deve avere:

  \begin{enumerate}
  \def\labelenumii{\arabic{enumii}.}
  \tightlist
  \item
    Media costante;
  \item
    L'autocorrelazione funzione solo di \(\tau\)
  \end{enumerate}

  \begin{itemize}
  \tightlist
  \item
    Dimostrazione:

    \begin{enumerate}
    \def\labelenumii{\arabic{enumii}.}
    \tightlist
    \item
      \begin{align*}
       E[y(t)]&= E[x(t)\otimes h(t)]= E\Big[\int_{-\infty}^{\infty} h(\alpha)\cdot x(t-\alpha) \,d\alpha\Big] =
       =\int_{-\infty}^{\infty} h(\alpha) \ E[x(t-\alpha)] \,d\alpha \\ &= m_{X} \int_{-\infty}^{\infty} h(\alpha) \,d\alpha = m_{X} \ H(0) =  \text{ costante}
       \end{align*}
    \item
      \begin{align*}
       R_{yy}(t_{1},t_{2}) &= \left\{ \begin{array}{lcl}
       t_{1} =t \\
       t_{2} = t + \tau \to \tau &= t_{2}-t_{1}
       \end{array} \right. \to \text{ cambio di variabile} \\
       R_{yy}(t, t + \tau) &= E[y(t)\cdot y(t+\tau)] = E\Big[\int_{-\infty}^{\infty} h(\alpha) \ x(t-\alpha)\,d\,\alpha \ \int_{-\infty}^{\infty}h(\beta) \ x(t+\tau -\beta) d\,\beta\Big]\\
       &=E\Big[\int_{-\infty}^{\infty}\int_{-\infty}^{\infty} h(\alpha) \ h(\beta) \ x(t-\alpha) x(t+\tau -\beta) \,d\alpha \,d\beta \Big]= \\
       &=\int_{-\infty}^{\infty}\int_{-\infty}^{\infty} h(\alpha) \ h(\beta) \ E \Big[x(t-\alpha) x(t+\tau -\beta) \Big] \,d\alpha \,d\beta \\
       &=\int_{-\infty}^{\infty}\int_{-\infty}^{\infty} h(\alpha) \ h(\beta) \ R_{xx}(\tau -\beta +\alpha) \,d\alpha \,d\beta \\&\to R_{yy}(t, t+\tau) = R_{yy(\tau)}
       \end{align*}
    \end{enumerate}

    Quindi il processo \(y(t)\) è WSS.
  \end{itemize}
\end{enumerate}

\subsection{Segnali a tempo discreto
aperiodici}\label{segnali-a-tempo-discreto-aperiodici}

\begin{enumerate}
\def\labelenumi{\arabic{enumi}.}
\setcounter{enumi}{1}
\item
  Trasformata di Fourier per sequenze (definizione, periodo 1,
  denormalizzazione);

  Data la sequenza \emph{aperiodica} \(x[n]\) \textbf{discreta}:
  \begin{align*}
   \overline{X}(f) &= \sum_{n=-\infty}^{\infty}x[n]\ e^{-j2\pi nfT} = \longrightarrow f=F\cdot F_{c} = \frac{F}{T} = \si{\hertz} \longrightarrow \overline{X}(F)
   = \sum_{n=-\infty}^{\infty}x[n] \ e^{j2\pi nF}
   \end{align*} \(\overline{X}(F)\) è \textbf{completamente nota} se
  conosco il suo andamento in un intervallo delle frequenze
  \emph{normalizzate} di ampiezza unitaria:
  \(\underbrace{F\in [-\frac{1}{2}; \frac{1}{2}]}_{\text{Intervallo base}}\)

  \begin{itemize}
  \item
    Periodica di periodo \(1\): \begin{align*}
     \overline{X}(F+1) &= \sum_{n=-\infty}^{\infty} x[n] \ e^{-j2\pi n(F+1)} =
     \sum_{n=-\infty}^{\infty} x[n] e^{-j2\pi nF}\underbrace{\cancel{e^{-j2\pi n}}}_{=1 \text{ (n intero)}} =
     =\sum_{n=-\infty}^{\infty} x[n] e^{-j2\pi nF} = \\ &= \overline{X}(F)
     \end{align*}
  \item
    \emph{Denormalizzazione}:
  \end{itemize}

  Necessaria in quanto se la sequenza \(x[n]\) deriva da un'operazione
  di campionamento, la frequenza normalizzata \textbf{non permette} di
  stabilire un legame con la frequenza (espressa in \si{\Hz}) delle
  componenti nella trasformata del segnale analogico di partenza.

  Se \(T\) è il periodo di campionamento,
  \(\Rightarrow f \triangleq \frac{F}{T} = F \cdot f_{c}\) in
  \si{\hertz}. Sostituendo ottengo \(F=fT\) in \si{\Hz} \begin{gather*}
   \text{TFS}\Big[X[n]\Big] = \overline{X}(f) \triangleq \sum_{n=-\infty}^{\infty} x[n] e^{-j2\pi nfT}
   \end{gather*} \(f\) continua in
  \(\Big[ -\frac{1}{2T}; \frac{1}{2T} \Big] \to \overline{X}(f)\)
  continua. Posso introdurre il \emph{modulo}
  \(\overline{A}(f)=|\overline{X}(f)|\) e lo \emph{spettro di fase}
  \(\overline{\theta}(f)= \phase{\overline{X}(f)}\).

  \(X(f)\) è periodica di periodo pari a
  \(f_c =\frac{1}{T} \Rightarrow \overline{X}(f+\frac{1}{T})= \sum x[n]e^{-j2\pi nFT} \cdot \underbrace{\cancel{e^{2\pi n \frac{1}{T}T}}}_{=1}=\overline{X}(f)\)
\item
  Relazione tra definizione di antitrasformata e trasformata; Criterio
  di convergenza per TFS (solo definizione).

  \[
   x[n] = \text{ITFS}\Big [ \overline{X}(f) \Big] = T \int_{-\frac{1}{2T}}^{\frac{1}{2T}} \overline{X}(f) \ e^{j2\pi n fT} \,df
   \]

  \begin{itemize}
  \tightlist
  \item
    Dimostrazione: \begin{align*}
     \overline{X}(f) &\triangleq \sum_{m=-\infty}^{\infty} x[m] \ e^{-j2\pi m fT} \to \begin{array}{cl}&\text{ moltiplico per osc.ni} \\ &\text{complesse alla frequenza } f \text{ e integro}\end{array}\\
     \int_{-\frac{1}{2T}}^{\frac{1}{2T}} \overline{X}(f) \ e^{j2\pi n fT} \,df &= \int_{-\frac{1}{2T}}^{\frac{1}{2T}} \sum_{m=-\infty}^{\infty} x[m] \ e^{-j2\pi m fT} \ e^{j2\pi n fT} \,df = \\
     & =\sum_{m=-\infty}^{\infty} x[m] \int_{-\frac{1}{2T}}^{\frac{1}{2T}} e^{-j2\pi (m-n)fT}\,df \\
     \text{Studiamo } \int_{-\frac{1}{2T}}^{\frac{1}{2T}} e^{-j2\pi (m-n)fT}\,df &\longrightarrow
     \left\{ \begin{array}{cl}
     \frac{1}{T} & : \ m=n \to \int_{-\frac{1}{2T}}^{\frac{1}{2T}} 1\cdot \,df = \frac{1}{T} \\
     0 & : \ m \neq n \to m-n=k \to \int_{-\frac{1}{2T}}^{\frac{1}{2T}} e^{-j2\pi kfT} = 0
     \end{array} \right. \\    
     \int_{-\frac{1}{2T}}^{\frac{1}{2T}} \overline{X}(f) \ e^{-j2\pi nfT} \,df &= \sum_{m=-\infty}^{\infty}
     x[m] \int_{-\frac{1}{2T}}^{\frac{1}{2T}} e^{-j2\pi(m-n)fT}\,df = \frac{1}{T}x[n] \\
     & \Rightarrow x[n] = T \int_{-\frac{1}{2T}}^{\frac{1}{2T}}\overline{X}(f)\ e^{j2\pi nfT}
     \end{align*} Nota bene: nell'ultima sommatoria ho tutti elementi
    pari a \(0\), tranne il caso \(m=n \to \frac{1}{T}\)
  \item
    Criterio di convergenza: \begin{gather*}
     \sum_{n=-\infty}^{\infty} \Big|x[n]\Big| < +\infty \Rightarrow \exists \text{ TFS}
     \end{gather*} Un criterio di convergenza per l'esistenza della
    trasformata è la \textbf{assoluta sommabilità} della frequenza.
  \end{itemize}
\end{enumerate}

\subsubsection{Teoremi}\label{teoremi}

\begin{enumerate}
\def\labelenumi{\arabic{enumi}.}
\setcounter{enumi}{3}
\item
  Teorema di Linearità; \[
   x[n]=ax_1[n]+bx_2[n] \Rightarrow \overline{X}(f) = a\overline{X_1}(f)+b\overline{X_2}(f)
   \]\\
\item
  Teorema del Ritardo;

  Sia \(x[n]\) una sequenza. \[
   x[n-k] \Longleftrightarrow \overline{X}(f) \cdot e^{-j2\pi kfT}
   \]

  \begin{itemize}
  \tightlist
  \item
    Dimostrazione: \begin{align*}
     \mathop{\mathrm{TFS}}[x[n-k]]& = \sum_{m=-\infty}^{\infty}x[n-k] \ e^{-j2\pi nfT} = \sum_{m=-\infty}^{\infty} x[m] \ e^{-j2\pi(m+k)fT} = e^{-j2\pi kfT}\sum_{m=-\infty}^{\infty}x[m] \ e^{-j2\pi mfT} =\\&=\overline{X}(f)e^{-j2\pi kfT}
     \end{align*}
  \end{itemize}
\item
  Teorema della Modulazione; \[
   x[n]\cdot e^{j2\pi nf_{0}t} \Longleftrightarrow \overline{X}(f-f_{0})
   \]

  \begin{itemize}
  \tightlist
  \item
    Dimostrazione: \begin{gather*}
     \mathop{\mathrm{TFS}}\Big[x[n]e^{j2\pi nf_{0}t}\Big] = \sum_{n=-\infty}^{\infty} x[n] \ e^{-j2\pi nfT} \cdot e^{j2\pi nf_{0}T}= \sum_{n=-\infty}^{\infty} x[n] \ e^{-j2\pi n(f-f_0)T} = \overline{X}(f-f_0)
     \end{gather*}
  \end{itemize}
\item
  Teorema della Somma di Convoluzione;

  Sia \(s[n]\) la sequenza discreta \emph{somma di convoluzione} tra le
  sequenze aperiodiche \(x[n]\) e \(y[n]\). \[
   s[n]= x[n]\otimes y[n] = \sum_{k=-\infty}^{\infty} x[k]\ y[n-k] = \sum_{k=-\infty}^{\infty} y[k] \ x[n-k]
   \] Gode delle stesse proprietà del caso continuo. \[
   \Rightarrow \overline{S}(f) = \overline{X}(f) \cdot \overline{Y}(f)
   \]

  \begin{itemize}
  \tightlist
  \item
    Dimostrazione: \begin{align*}
     \overline{S}(f) &= \sum_{n=-\infty}^{\infty} \sum_{k=-\infty}^{\infty} x[k]\ y[n-k] e^{-j2\pi nfT} = \sum_{k=-\infty}^{\infty} x[k] \underbrace{\sum_{n=-\infty}^{\infty}y[n-k]\ e^{-j2\pi nfT}}_{\text{ritardo}}= \\
     &= \sum_{n=-\infty}^{\infty} x[k] \overline{Y}(f) \ e^{-j2\pi kfT} = \overline{Y}(f)  \sum_{n=-\infty}^{\infty} \ x[k]  e^{-j2\pi kfT} =\overline{Y}(f) \ \overline{X}(f)
     \end{align*}
  \end{itemize}
\item
  Teorema del Prodotto; \[
   p[n]=x[n]\cdot y[n] \Longleftrightarrow \overline{P}(f) = \overline{X}(f) \otimes \overline{Y}(f)
   \]

  \begin{itemize}
  \tightlist
  \item
    Dimostrazione: \begin{align*}
     \overline{P}(f)&=\sum_{n=-\infty}^{\infty} p[n] \ e^{-j2\pi nfT} =  \sum_{n=-\infty}^{\infty} x[n] \ y[n] \ e^{-j2\pi nfT} = \sum_{n=-\infty}^{\infty}
     \underbrace{\Big[T\int_{-\frac{1}{2T}}^{\frac{1}{2T}} \overline{X}(\alpha)\ e^{j2\pi n\alpha T} \,d\alpha\Big]}_{\text{antitrasformata di }\overline{X}(f)} y[n] \ e^{-j2\pi nfT} = \\
     &= T \int_{-\frac{1}{2T}}^{\frac{1}{2T}} \overline{X}(\alpha)  \underbrace{\sum_{n=-\infty}^{\infty} y[n] \ e^{-j2\pi n(f-\alpha)T}}_{\text{dalla modulazione }\to \overline{Y}(f-\alpha)} \,d\alpha=
     T \int_{-\frac{1}{2T}}^{\frac{1}{2T}} \overline{X}(\alpha) \overline{Y}(f-\alpha)\,d\alpha  \\
     &\Rightarrow \overline{P}(f) = T \int_{-\frac{1}{2T}}^{\frac{1}{2T}} \overline{X}(\alpha) \overline{Y}(f-\alpha)\,d\alpha
     \end{align*} Questa è la \textbf{convoluzione ciclica o periodica}.
    L' integrale viene calcolato su un singolo periodo e il risultato è
    diviso per l'ampiezza del periodo \(\frac{1}{T}\)
  \end{itemize}
\item
  Teorema dell'Incremento; \[
   \odv{x(t)}{t}\Big|_{t=nT} \cong \frac{x(nT)-x(nT-T)}{T} = \frac{x[n]-x[n-1]}{T}, \text{ con } x[n] \triangleq x(nT)  
   \] Si introduce l'operatore \textbf{incremento}
  \(\Delta x[n] \triangleq x[n] - x[n-1]\) \newline Usando il
  \emph{teorema del ritardo} per ottenere la trasformata: \[
   \Delta x[n] \Longleftrightarrow \overline{X}(f) - \overline{X}(f)\ e^{-j2\pi fT} = \overline{X}(f) \ (1-e^{-j2\pi fT})
   \] È l'analogo del teorema di derivazione: approssima la derivata del
  rapporto incrementale.
\item
  Teorema della Sequenza Somma.

  Consideriamo la sequenza somma
  \(\displaystyle y[n]=\sum_{k=-\infty}^{n}x[k]\). Dal teorema
  dell'incremento otteniamo la sua trasformata in sequenza: \[
  \overline{Y}(f) = \frac{\overline{X}(f)}{1-e^{-j2\pi fT}}
  \] purché \(\overline{X}(0) = 0\).

  \begin{itemize}
  \tightlist
  \item
    Dimostrazione: \begin{gather*}
    z[n] = \Delta y[n] \Longleftrightarrow \overline{Z}(f)=\overline{Y}(f)[1-e^{-j2\pi fT}] \text{ dal teorema dell'incremento} \\
    \text{però } \Delta y[n] = y[n] - y[n-1] = \sum_{k=-\infty}^{n} x[k] - \sum_{k=-\infty}^{n-1}x[k] = x[n] \\
    \text{quindi } \overline{X}(f) = \overline{Y}(f)[1-e^{-j2\pi fT}] \to \overline{Y}(f) = \frac{\overline{X}(f)}{1-e^{-j2\pi fT}}
    \end{gather*}
  \end{itemize}
\end{enumerate}

\subsubsection{Campionamento:}\label{campionamento}

\begin{enumerate}
\def\labelenumi{\arabic{enumi}.}
\setcounter{enumi}{10}
\item
  Teorema del campionamento (``risultato'' dell'interpolazione
  cardinale);

  Un segnale il cui spettro è \emph{limitato} nella banda \(B\) può
  essere ricostruito a partire dai propri campioni, \textbf{purché}
  \(f_c \geq 2B\)

  \(p(t)\) è un impulso ``diverso'' per generalizzare l'operazione
  d'interpolazione, anche al fine di evitare le discontinuità che lo
  stesso impulso \(p(t)\) introduce nell'interpolazione a mantenimento.
  \begin{align*}
  \hat{x}(t)&= \sum_{n=-\infty}^{\infty} x[n] \ p(t-nT), \ \text{scegliendo } p(t)= \mathop{\mathrm{sinc}}\Big(\frac{f}{T}\Big) \Rightarrow P(f)=T\mathop{\mathrm{rect}}(fT) \\ 
  \hat{X}(f)&= P(f) \ \overline{X}(f) = \cancel{T}\mathop{\mathrm{rect}}\Big(fT\Big)\frac{1}{\cancel{T}}\cdot  \sum_{k=-\infty}^{+\infty} X\Big(f-\frac{k}{T}\Big) \\ &= X(f) \to \text{ molto importante!} \\
     &\boxed{\begin{array}{cl}
    \text{ricampionando il segnale}
    \\
     \text{interpolato al generico istante } t=kT
    \end{array}}&
    \\
  \hat{x}(kT) &=\sum_{n=-\infty}^{+\infty} x(nT) \ \mathop{\mathrm{sinc}}\Big(\frac{kT-nT}{T}\Big), \text{ ma } \mathop{\mathrm{sinc}}(k-n)=
  \left\{ \begin{array}{cl}
  0 & n\neq k \\
  1 & k=n
  \end{array} 
  \right. \\
  &\text{Quindi: } \hat{x}(kT)=\sum_{n=-\infty}^{+\infty}x[n] \ \underbrace{\delta[k-n]}_{1\text{ sse }n=k}= x[k] = x(kT)
  \end{align*} Il segnale di partenza coincide con il segnale
  interpolato.
\item
  Relazione tra TCF e TFS.

  Sappiamo che nel dominio ``temporale'' è valida la relazione
  \(x[n]=x(nT)\): vogliamo trovare una relazione simile anche dal punto
  di vista frequenziale:\\
  \begin{align*}
  \overline{X}(f) &\triangleq \sum_{n=-\infty}^{\infty} x[n] \ e^{-j2\pi nfT} = \sum_{n=-\infty}^{\infty} x(nT) \ e^{-j2\pi nfT}=
  \boxed{\text{Sapendo che: } x(nT)=\int_{-\infty}^{\infty}X(\alpha) e^{j2\pi\alpha nT}\,d\alpha} \\
  &= \sum_{n=-\infty}^{\infty} \int_{-\infty}^{\infty} X(\alpha) \ e^{-j2\pi \alpha nT} \,d\alpha \cdot e^{-j2\pi nfT}
  = \int_{-\infty}^{\infty} X(\alpha) \sum_{n=-\infty}^{\infty} e^{-j2\pi n(f-\alpha)T} \,d\alpha
  \end{align*} ma il segnale \emph{pettine di Dirac} è esprimibile in
  serie di Fourier con coefficienti pari a \(\frac{1}{T}\): \[
  \displaystyle
  \boxed{\displaystyle
  \begin{array}{c} \displaystyle
  \sum_{n=-\infty}^{\infty} \delta(t-nT) = \frac{1}{T} \sum_{k=-\infty}^{\infty} e^{\frac{j2\pi kt}{T}} \\
  \text{Trasformata di Fourier} \updownarrow \\ \displaystyle
  \sum_{n=-\infty}^{\infty} e^{-j2\pi nfT} = \frac{1}{T}\sum_{k=-\infty}^{\infty} \delta(f-\frac{k}{T})
  \end{array}
  }
  \] \begin{align*}
  &\int_{-\infty}^{\infty}X(\alpha) \frac{1}{T} \sum_{k=-\infty}^{\infty} \delta(f-\alpha -\frac{k}{T})\,d\alpha = \frac{1}{T}\sum_{k=-\infty}^{\infty} \int_{-\infty}^{\infty}X(\alpha) \delta(f-\alpha -\frac{k}{T})\,d\alpha = \\
  &=\delta \text{ è pari } = \frac{1}{T}\sum_{k=-\infty}^{\infty} \int_{-\infty}^{\infty}X(\alpha) \delta\Big(\alpha -(f- \frac{k}{T})\Big)\,d\alpha = \text{ prodotto tra } X(\alpha) \text{ e Dirac centrato in } f -\frac{k}{T} \\
  &=\text{per la proprietà campionatrice della delta di Dirac } \longrightarrow \overline{X}(f) = \frac{1}{T}\sum_{k=-\infty}^{\infty}X(f-\frac{k}{T})
  \end{align*} Quindi questa relazione dimostra che la trasformata di
  Fourier di una sequenza ottenuta per campionamento si può ricavare
  come \emph{periodicizzazione} della trasformata del segnale analogico
  di partenza, con un periodo di ripetizione in frequenza pari alla
  frequenza di campionamento \(\frac{1}{T}\)!
\end{enumerate}

\subsection{Segnali a tempo discreto
periodici}\label{segnali-a-tempo-discreto-periodici}

\begin{enumerate}
\def\labelenumi{\arabic{enumi}.}
\setcounter{enumi}{12}
\item
  Trasformata discreta di Fourier (definizione);

  Supponiamo \(x[n]\) periodica di periodo \(N_0\) \[
  \underset{\text{antitrasformata discreta di Fourier}}{x[n]=\frac{1}{N_0}\sum_{k=0}^{N_0 -1}\overline{X}_{k} \ e^{j\frac{2\pi kn}{N_0}}}; \
  \underset{\text{Trasformata discreta di Fourier}} {\overline{X}_k = \sum_{n=0}^{N_0 -1} x[n] \ e^{-j \frac{2\pi kn}{N_0}}}
  \]
\item
  La trasformata di una sequenza periodica è essa stessa periodica
  (stesso periodo \(N_0\)); \begin{gather*}
  \overline{X}_{k+N_0} = \sum_{n=0}^{N_0 -1} x[n] e^{-j2\pi(k+N_0)\frac{n}{N_0}} = \sum_{n=0}^{N_0 -1} x[n] e^{-j2\pi k\frac{n}{N_0}} \ e^{-j2\pi n} = \overline{X}_k
  \end{gather*}
\item
  La relazione di sintesi di una TDF discende da quella di analisi;
  \begin{align*}
  x[n] &=\frac{1}{N_0} \sum_{k=0}^{N_0 -1} \overline{X}_k \ e^{j\frac{2\pi kn}{N_0}} \Rightarrow \overline{X}_k = \sum_{n=0}^{N_0 -1} x[n] e^{-j\frac{2\pi kn}{N_0}} \\
  &\text{Moltiplichiamo per } e^{-j \frac{2\pi nm}{N_0}} \text{ con } (0\leq m \leq N_0 - 1) \text{ ed effettuiamo una somma sul periodo } (n)& \\
  &\sum_{n=0}^{N_0 -1} x[n] e^{-\frac{j2\pi mn}{N_0}} =\frac{1}{N_0} \sum_{n=0}^{N_0 -1} \sum_{k=0}^{N_0 -1}\overline{X}_k e^{j\frac{2\pi n(k-m)}{N_0}} = \frac{1}{N_0}\sum_{k=0}^{N_0 -1}\overline{X}_k \sum_{n=0}^{N_0 -1} e^{j\frac{2\pi n(k-m)}{N_0}}= 
  \end{align*} \begin{align*}
  \boxed{
      \begin{array}{cl}
          \text{Sviluppando il secondo membro:}
          \displaystyle\frac{1}{N_0}\sum_{k=0}^{N_0 -1}\overline{X}_k \sum_{n=0}^{N_0 -1} e^{j\frac{2\pi n(k-m)}{N_0}} \\
          \text{La seconda sommatoria si sviluppa come}: \\
          \displaystyle 
          \sum_{n=0}^{N_0-1}e^{j\frac{2\pi n(k-m)}{N_0}}=\sum_{n=0}^{N_0-1}\Bigg(e^{j\frac{2\pi(k-m)}{N_0}}\Bigg)^{n}=
          \left\{
          \begin{array}{cl}
              \displaystyle \frac{1-e^{j2\pi(k-m)}}{1-e^{j\frac{2\pi(k-m)}{N_0}}}=0 & k\neq m\\
              \displaystyle \sum_{n=0}^{N_0-1}e^{j\frac{2\pi n(k-m)}{N_0}}=\sum_{n=0}^{N_0-1}1=N_0 & k=m
          \end{array}\right. \\
          \text{ovvero: } \displaystyle\sum_{n=0}^{N_0-1}e^{j\frac{2\pi n(k-m)}{N_0}}=\left\{\begin{array}{cl}\text{N}_0&\text{per n=m}\\0&\text{per}\ne\text{m}\end{array}\right.= \delta[k-m] N_0
      \end{array}
  }
  \end{align*} \begin{align*}
  &=\frac{1}{N_0} \sum_{n=0}^{N_0 -1} \sum_{k=0}^{N_0 -1} \overline{X}_k e^{-j\frac{2\pi mn}{N_0}} \ e^{j\frac{2\pi nk}{N_0}} = \frac{1}{N_0}\sum_{k=0}^{N_0 -1} \overline{X}_k \ \delta[k-m] N_0 = \frac{\cancel{N_0}}{\cancel{N_0}} \overline{X}_m  \\
  &\text{ per sostituzione infine: } \sum_{n=0}^{N_0 -1} x[n] e^{-j\frac{2\pi nm}{N_0}} = \overline{X}_m&
  \end{align*}
\end{enumerate}

\subsubsection{Proprietà:}\label{proprietuxe0-1}

Notazione:
\(\operatorname{DFT}_{N_0}\Big\{x[n]\Big\} = \overline{X}_k, \text{ con } 0\leq n, k \leq N_0 -1\)

\begin{enumerate}
\def\labelenumi{\arabic{enumi}.}
\setcounter{enumi}{15}
\item
  Proprietà di Linearità; \[
  \operatorname{DFT}_{N_0}\Big\{ax[n]+by[n]\Big\} = a\overline{X}_k + b\overline{Y}_k
  \]
\item
  Proprietà di Traslazione Circolare; \[
  \operatorname{DFT}_{N_0}\Big\{ x[n-n_0] \Big\} = \overline{X}_k \ e^{-\frac{j2\pi kn_0}{N_0}}
  \]

  \begin{itemize}
  \tightlist
  \item
    Dimostrazione: \begin{align*}
    \operatorname{DFT}_{N_0}\Big\{ x[n-n_0] \Big\} &= \sum_{n=0}^{N_0 -1} x[n-n_0] \ e^{-j2\pi\frac{n}{N_0}k} =
    \boxed{
    \begin{array}{cl}
    \text{cambio di variabile} \\
    p  = n - n_0 \\
    n  = p + n_0
    \end{array}} \\
    &= \sum_{p=-n_0}^{N_0 - 1 - n_0} x[p] \ e^{-j\frac{2\pi(p+n_0)k}{N_0}} = \sum_{p=-n_0}^{N_0 - 1 - n_0} x[p] \ e^{-j\frac{2\pi kp}{N_0}} \ e^{-j\frac{2\pi n_{0}k}{N_0}}=\\
    &= e^{-j\frac{2\pi kn_0}{N_0}} \sum_{p=-n_0}^{N_0 - 1 - n_0} x[p] \ e^{-j\frac{2\pi kp}{N_0}} = e^{-j\frac{2\pi kn_0}{N_0}} \sum_{p=0}^{N_0 - 1} x[p] \ e^{-j\frac{2\pi kp}{N_0}} \\
    &\operatorname{DFT}_{N_0}\Big\{ x[n-n_0] \Big\} = e^{-j\frac{2\pi kn_0}{N_0}} \cdot \overline{X}_k &
    \end{align*} Si dice traslazione \textbf{circolare} in quanto,
    osservando la periodicità della sequenza originale e di quella
    traslata, è possibile notare come i campioni che ``escono'' a destra
    dell'intervallo rientrano alla sinistra dell'intervallo stesso.
  \end{itemize}
\item
  Proprietà di Traslazione In Frequenza; \[
  \operatorname{DFT}_{N_0}\Big\{ x[n] \ e^{-j\frac{2\pi k_0 n}{N_0}} \Big\} = \overline{X}_{k-k_0}
  \]
\item
  Proprietà di Inversione Temporale; \begin{gather*}
  \operatorname{DFT}_{N_0}\Big\{ x[-n] \Big\} = \overline{X}_{-k} = \overline{X}_{N_0 - k} 
  \end{gather*}

  \begin{itemize}
  \tightlist
  \item
    Dimostrazione: \begin{align*}
    \operatorname{DFT}_{N_0}\Big\{ x[-n] \Big\} &= \sum_{n=0}^{N_0 -1} x[-n] \ e^{-j\frac{2\pi kn}{N_0}} = \sum_{n=0}^{N_0 -1} x[N_0 -n] \ e^{-j\frac{2\pi kn}{N_0}} =
    \boxed{
    \begin{array}{cl}
    \text{cambio di variabile} \\
    p  = N_0 - n \\
    n  = N_0 - p
    \end{array}}\\
    &= \sum_{p=1}^{N_0} x[p] \ e^{-j\frac{2\pi k(N_0 -p)}{N_0}} = \underbrace{e^{-j2\pi k}}_{=1} \ \sum_{p=0}^{N_0 -1} x[p] \ e^{-j\frac{2\pi(-k)p}{N_0}} = \overline{X}_{-k} = \overline{X}_{N_0 - k}
    \end{align*} dove nel primo passaggio, abbiamo usato la periodicità
    della sequenza x{[}n{]} e nel penultimo passaggio, per cambiare gli
    indici della sommatoria, abbiamo usato le proprietà di periodicità
    della sequenza e della funzione esponenziale, come già visto.
  \end{itemize}
\item
  Proprietà di coniugazione; \[
  \operatorname{DFT}_{N_0}\Big\{ x^{*}[n] \Big\} = \overline{X}^{*}_{-k} = \overline{X}^{*}_{N_0 - k}
  \]
\item
  Simmetria per sequenze reali (pari e dispari);

  Per una sequenza reale \(x[n]\) abbiamo: \[
  \operatorname{DFT}_{N_0}\Big\{ x[n] \Big\} = \operatorname{DFT}_{N_0}\Big\{x^{*}[n] \Big\} \to \overline{X}_k = \overline{X}^{*}_{-k} = \overline{X}^{*}_{N_0 -k}
  \] da cui derivano le proprietà di simmetria per il modulo e per la
  fase: \begin{gather*}
  \Big|\overline{X}_k \Big| = \Big|\overline{X}_{-k}\Big| = \Big| \overline{X}_{N_0 - k} \Big| \\
  \phase{\overline{X}_k} = -\phase{\overline{X}_{-k}} = -\phase{\overline{X}_{N_0 -k}}
  \end{gather*} Tali relazioni implicano che il modulo della sequenza
  \(X[k]\) è simmetrico rispetto al valore \(k = \frac{N}{2}\), mentre
  la fase è antisimmetrica rispetto a tale valore.

  \begin{itemize}
  \tightlist
  \item
    per sequenze di lunghezza \textbf{pari}, il centro di simmetria
    coincide con un campione della sequenza;
  \item
    per sequenze di lunghezza \textbf{dispari}, invece, il centro di
    simmetria coincide con un punto equidistante tra due campioni.
  \end{itemize}
\item
  Teorema di Parseval per sequenze; \begin{gather*}
  \left\{ \begin{array}{cl} \displaystyle
  \sum_{n=0}^{N_0 -1} x[n] \ y^{*}[n] = \frac{1}{N_0} \sum_{k=0}^{N_0 -1}\overline{X}_k \ \overline{Y}^{*}_{k} \\
  \displaystyle\sum_{n=0}^{N_0 -1} \Big|x[n]\Big|^{2} = \frac{1}{N_0} \sum_{k=0}^{N_0 -1} \Big|\overline{X}_{k}\Big|^{2}
  \end{array} \right.
  \end{gather*}

  \begin{itemize}
  \tightlist
  \item
    Dimostrazione: \begin{gather*}
    \sum_{n=0}^{N_0 -1} x[n] \ y^{*}[n] =\sum_{n=0}^{N_0 -1}x[n] \ \Big(\frac{1}{N_0}\sum_{k=0}^{N_0 - 1}\overline{Y}_k^{*} \ e^{j\frac{2\pi kn}{N_0}}\Big)^{*} 
    = \frac{1}{N_0} \sum_{k=0}^{N_0 -1} \overline{Y}_k \sum_{n=0}^{N_0 -1 }x[n] \ e^{-j\frac{2\pi kn}{N_0}} = \frac{1}{N_0} \sum_{k=0}^{N_0 -1} \overline{X}_k \ \overline{Y}^{*}_k
    \end{gather*}

    \begin{itemize}
    \tightlist
    \item
      Ponendo \(x[n]=y[n]\) si ottiene la seconda relazione.
    \end{itemize}
  \end{itemize}

  \begin{quote}
  L'energia si mantiene nei domini.
  \end{quote}
\item
  Teorema del Prodotto;

  Consideriamo adesso la sequenza (periodica) \(p[n]\) data dal
  \emph{prodotto} fra la sequenza \(x[n]\) e la sequenza \(y[n]\)
  entrambe periodiche di periodo \(N_0\) \[
  p[n] =x[n] \ y[n]
  \] e calcoliamone la trasformata discreta di Fourier: \begin{align*}
  \overline{P}_k &= \sum_{n=0}^{N_0 - 1} p[n] \ e^{-j2\pi\frac{nk}{N_0}} = \sum_{n=0}^{N_0 - 1}x[n] \ y[n] e^{-j2\pi\frac{nk}{N_0}}
  = \sum_{n=0}^{N_0 -1}\frac{1}{N_0}\sum_{m=0}^{N_0 -1}\overline{X}_m \ e^{j2\pi\frac{nm}{N_0}}\cdot y[n] \ e^{-j2\pi\frac{nk}{N_0}}
  \end{align*} Dove \(x[n]\) è stata scomposta in serie discreta di
  Fourier. Inoltre è stata utilizzata una variabile ``muta'' \(m\)
  nell'antitrasformata per non creare ambiguità con la variabile \(k\),
  da cui dipende la trasformata. \begin{align*}
  &= \frac{1}{N_0} \sum_{n=0}^{N_0 -1}\sum_{m=0}^{N_0 -1} \overline{X}_m \ e^{j2\pi \frac{nm}{N_0}} \cdot y[n] e^{-j2\pi\frac{nk}{N_0}}
  \end{align*} Invertendo l'ordine delle sommatorie otteniamo:
  \begin{align*}
  \overline{P}_k &= \sum_{m=0}^{N_0 -1} \overline{X}_m \frac{1}{N_0}\sum_{n=0}^{N_0 -1} y[n] \ e^{-j2\pi{\frac{n(k-m)}{N_0}}} = \frac{1}{N_0}\sum_{m=0}^{N_0 -1}\overline{X}_m \ \overline{Y}_{k-m}= \\ 
  &= \frac{1}{N_0}\cdot \overline{X}_k \otimes \overline{Y}_k
  \end{align*} La convoluzione tra le due trasformate discrete è una
  somma di convoluzione \textbf{ciclica} tra le due sequenze periodiche
  \(\overline{X}_k\) e \(\overline{Y}_k\) in ambito \emph{frequenziale}.
  In conclusione: \begin{align*}
  p[n] = x[n] \ y[n] \Longleftrightarrow \overline{P}_k = \frac{1}{N_0}\cdot \overline{X}_k \otimes \overline{Y}_k
  \end{align*}
\item
  Teorema della Convoluzione (+ relazioni tra convoluzione lineare e
  circolare).

  Consideriamo ora la sequenza \(z[n]\) come somma di convoluzione
  ciclica o circolare tra le due sequenze \(x[n]\) e \(y[n]\),
  periodiche di periodo \(N_0\) \[
  z[n] = x[n] \otimes y[n] = \sum_{m=0}^{N_0 -1}x[m] \ y[n-m] = \sum_{m=0}^{N_0 -1}y[m] \ x[n-m]
  \] La somma di convoluzione gode di tutte le proprietà citate per la
  somma di convoluzione tra sequenze periodiche. \newline Calcoliamo la
  trasformata discreta di \(z[n]\): \begin{align*}
  \overline{Z}_k &= \sum_{n=0}^{N_0 -1} z[n] \ e^{-j2\pi{\frac{nk}{N_0}}} = \sum_{n=0}^{N_0 -1} \sum_{m=0}^{N_0 -1} x[m] \ y[n-m] e^{-j2\pi{\frac{nk}{N_0}}} = 
  \sum_{m=0}^{N_0 -1} x[m] \sum_{n=0}^{N_0 -1} y[n-m] e^{-j2\pi\frac{nk}{N_0}} = \\
  &= \sum_{m=0}^{N_0 -1} x[m] \ \overline{Y}_k \ e^{-j2\pi\frac{mk}{N_0}} = \overline{Y}_k \sum_{m=0}^{N_0 -1} x[m] \ e^{-j2\pi\frac{mk}{N_0}}= \\
  &= \overline{X}_k \cdot \overline{Y}_k
  \end{align*} Quindi: \[
  x[n]\otimes y[n] \Longleftrightarrow \overline{X}_k \cdot \overline{Y}_k
  \]

  \begin{itemize}
  \item
    Relazioni tra convoluzione lineare e circolare (per due sequenze di
    lunghezza finita):

    Siano \(x[n]\) e \(h[n]\) due sequenze di lunghezza \(L\) e \(M\):
    il ``supporto'' sul quale le sequenze hanno campioni \emph{non
    nulli} è \([0, L-1]\) e \([0, M-1]\).

    \begin{itemize}
    \tightlist
    \item
      La convoluzione \emph{lineare} è ottenuta dalla relazione: \[
        y_{l}[n] = x[n] * h[n] = \sum_{k=-\infty}^{\infty} x[k] \ h[n-k]
        \] dove le sequenze sono considerate \textbf{aperiodiche} e
      \(y_l[n]\) ha una lunghezza \emph{finita} e pari a \(L+M-1\)
      campioni, considerando il ``supporto'' dove le sequenze hanno
      campioni non nulli. Vi è quindi una \emph{sovrapposizione} tra
      campioni non nulli delle due sequenze in \([0, L+M-2]\)
    \item
      La convoluzione \emph{circolare} invece, a causa della diversa
      lunghezza delle due sequenze, richiede di fissare un
      \textbf{periodo comune} \(N\), per eseguire l'\textbf{estensione
      periodica} delle sequenze: l'unico vincolo da porre diventa quindi
      \(N \geq \text{max}(L,M)\), appendendo quindi in fondo alle
      sequenze un numero di zeri (pari a \(N-L\) o \(N-M\)) prima di
      estendere periodicamente le due sequenze. La convoluzione
      circolare è data da: \[
        y_{c}[n] = x[n] \otimes h[n] =\sum_{k=0}^{N-1}x[k] \ h[n-k]
        \]
    \end{itemize}
  \end{itemize}
\end{enumerate}

\subsubsection{Generale:}\label{generale}

\begin{enumerate}
\def\labelenumi{\arabic{enumi}.}
\setcounter{enumi}{24}
\item
  Fast Fourier Transform (FFT).

  Supponiamo di avere in memoria un numero \(N_0\) di valori della
  sequenza periodica \(x[n]\) e calcoliamone numericamente la
  trasformata discreta: \[
  \overline{X}_k = \sum_{n=0}^{N_0 -1} \  e^{-j\frac{1\pi kn}{N_0}} =\Big\{ x[0]\cdot e^{-j0}+x[1]\cdot e^{j\frac{2\pi k}{N_0}}+x[2]\cdot e^{-j\frac{2\pi 2k}{N_0}} + \cdots + x[N_0 -1]\cdot e^{-j\frac{2\pi(N_0 -1)k}{N_0}}\Big\}
  \] Vogliamo quindi determinare il numero di \emph{operazioni
  necessarie} sia per la trasformata che per l'antitrasformata
  (ignorando il fattore di scala \(\frac{1}{N_0}\)): \[
  x[n] = \overline{X}_0 \cdot e^{j0}+\overline{X}_1\cdot e^{j\frac{2\pi k}{N_0}}+\overline{X}_2\cdot e^{j\frac{2\pi 2k}{N_0}} + \cdots + \overline{X}_{N_0 -1}\cdot e^{j\frac{2\pi(N_0 -1)k}{N_0}}
  \] Supponiamo quindi di avere valori \emph{complessi} \(z=a+jb\), ma
  \textbf{precalcolati}, ovvero già in memoria. Per il calcolo di un
  \emph{singolo campione} \(X[k]\) è necessario eseguire \(N_0\)
  moltiplicazioni complesse e \(N_0 -1\) addizioni complesse, le quali
  sono tradotte dai calcolatori in operazioni nel campo reale,
  eseguendole tra le parti reali e immaginari dei numeri complessi
  coinvolti:

  \begin{itemize}
  \tightlist
  \item
    per eseguire un'addizione complessa è necessario eseguire 2
    addizioni reali: \((a + jb) + (c + jd) = (a + c) + j(b + d)\)
  \item
    per eseguire una moltiplicazione complessa è necessario eseguire 4
    moltiplicazioni reali e 2 addizioni reali,
    \((a + jb) \cdot (c + jd) = (ac - bd) + j(ad + bc)\).
  \end{itemize}

  Quindi per un singolo campione \(X[k]\) sono necessarie \(N_0\)
  moltiplicazioni complesse e \(N_0 -1\) somme complesse. Inoltre, per
  ogni valore di \(k\) sono necessarie \(8N_0 -2\) operazioni reali, e
  dato che la sequenza è composta da \(N_0\) valori la complessità
  diventa \textbf{quadratica}: \[
  (8N_0 -2)N_0 = 8N_{0}^{2} -2N_0 \approx 8N_{0}^2
  \] Per velocizzare i calcoli è stato quindi introdotto l'algoritmo
  \textbf{Fast Fourier Transform} o FFT, il quale viene applicato se
  \emph{\(N_0\) è una potenza del \(2 \to N_0 = 2^{M}\)} \begin{align*}
  \overline{X}_k &= \sum_{m=0}^{\frac{N_0}{2}-1} x[2m]\ e^{-j\frac{2\pi(2m)k}{N_0}} + \sum_{m=0}^{\frac{N_0}{2}-1} x[2m+1] \ e^{-j\frac{2\pi(2m-1)k}{N_0}}= \\
  & = \overbrace{\sum_{m=0}^{\frac{N_0}{2}-1} x[2m]\ e^{-j\frac{2\pi mk}{\frac{N_0}{2}}}}^{\overline{P}_k}+e^{-j\frac{2\pi k}{\frac{N_0}{2}}} \cdot \overbrace{\sum_{m=0}^{\frac{N_0}{2}-1} x[2m+1] \ e^{-j\frac{2\pi mk}{\frac{N_0}{2}}}}^{\overline{D}_k}
  \end{align*} con \(k= 0, \dots , N_0 -1\). \(\overline{P}_k\) è la
  trasformata ottenuta dai \(\frac{N_0}{2}\) campioni di \emph{indice
  pari} di \(x[n]\), mentre \(\overline{D}_k\) indica la trasformata
  della sequenza ottenuta dai \(\frac{N_0}{2}\) campioni di \emph{indice
  dispari} Questa scomposizione è ricorsiva dell'ordine, in quanto la
  trasformata di ordine \(N_0\) è espressa come combinazione lineare di
  due trasformate di ordine \(\frac{N_0}{2}\): questo concetto può
  essere esteso al numero di operazioni \(N_{FFT}\): \[
  N_{FFT}(N_0)=N_{FFT}(\frac{N_0}{2})+N_{FFT}(\frac{N_0}{2})+6N_0 + 2N_0
  \] tenendo conto che per ogni \(k\) (dei coefficienti dispari) è
  necessario moltiplicare \(D_k\) per un esponenziale complesso
  (precalcolato, 6 operazioni reali) e poi sommare con \(P_k\) (2
  operazioni reali).

  Il procedimento viene poi ripetuto ricorsivamente. Infine, iterando la
  formula: \[
  N_{FFT}(N_0) = 6N_0 + 8N_0\log_2 N_0 \approx 8N_0 \log_2 N_0
  \] la complessità risulta \textbf{logaritmica}! Ad esempio, con
  \(N_0=1024\) \[
  \frac{N_{TDF}(N_0)}{N_{FFT}(N_0)} = \frac{8N_{0}^{\cancel{2}}}{\cancel{8N_0}\log_2 N_0} = \frac{N_0}{\log_2 N_0} \approx 100
  \] Il vantaggio conseguito dall'utilizzo di FFT al posto della
  trasformata classica aumenta al crescere di \(N_0\)
\end{enumerate}

\subsection{Sistemi monodimensionali a tempo
discreto}\label{sistemi-monodimensionali-a-tempo-discreto}

\[
y[n] = \text{T}\Big[x[m], n\Big] = \text{T}\Big[x[n] \Big]
\]

\subsubsection{Proprietà dei sistemi;}\label{proprietuxe0-dei-sistemi}

\begin{enumerate}
\def\labelenumi{\arabic{enumi}.}
\setcounter{enumi}{25}
\item
  SLS a tempo discreto: risposta impulsiva. \[
  h[n] = \text{T}\Big[ \delta[n] \Big]
  \] Nota \(h[n]\): \begin{align*}
  y[n] &= \text{T}\Big[ x[n] \Big] = \text{T}\Big[ \sum_{k=-\infty}^{\infty}x[k] \ \delta[n-k] \Big] = \text{per la linearità} \\
  &= \sum_{k=-\infty}^{\infty}x[k] \text{T}\Big[ \delta[n-k] \Big] = \text{per la stazionarietà: } \to
  \left \{\begin{array}{cl}
  \text{T}\Big[ \delta[n-k] \Big]=h[n-k]
  \end{array}\right. \\
  &= \sum_{k=-\infty}^{\infty}x[k] \ h[n-k] = x[n] \otimes h[n]
  \end{align*} Quindi l'uscita è una \emph{somma di convoluzione} tra la
  sequenza in ingresso e la risposta impulsiva del sistema stesso!

  Un sistema lineare e stazionario (SLS) può essere:

  \begin{itemize}
  \tightlist
  \item
    FIR (Finite Impulse Response) se la sua risposta impulsiva è
    costituita da un \textbf{numero finito di campioni};
  \item
    IIR (Infinite Impulse Response) se la sua risposta impulsiva è
    costituita da un numero \textbf{infinito} di campioni.
  \end{itemize}

  È possibile dimostrare come la \emph{condizione necessaria e
  sufficiente} per la stabilità in senso BIBO di un SLS è
  l'\textbf{'assoluta sommabilità} della sua risposta impulsiva: \[
  \sum_{k=-\infty}^{\infty} \Big| h[k] \Big| < \infty
  \] Da ciò deriviamo che

  \begin{itemize}
  \tightlist
  \item
    i sistemi FIR sono stabili: la sommatoria diventa una somma
    \emph{finita} di quantità \emph{limitate};
  \item
    mentre i sistemi IIR non lo sono sempre. Diventa necessario
    controllare la validità della seguente condizione:

    \begin{itemize}
    \item
      \textbf{sequenza causale}:

      un SLS è causale se e solo se la sua risposta impulsiva è una
      sequenza causale, cioè: \[
        h[n] = 0 \text{ se } n<0, \text{ ovvero } h[n] = h[n] \ u[n] 
        \]
    \end{itemize}
  \end{itemize}
\end{enumerate}

\subsubsection{Proprietà:}\label{proprietuxe0-2}

\begin{enumerate}
\def\labelenumi{\arabic{enumi}.}
\setcounter{enumi}{26}
\item
  Sistemi a cascata e in parallelo;

  \begin{itemize}
  \tightlist
  \item
    In un sistema a \textbf{cascata} l'ordine degli stessi può essere
    variato senza alterare l'uscita: \begin{align*}
      h[n] = h_{1}[n] \otimes h_{2}[n] \to y[n] = x[n] \otimes \Big(h_{1}[n] \otimes h_{2}[n] \Big)
      \end{align*} Questo per la proprietà commutativa del prodotto di
    concoluzione discreta.
  \item
    Due sistemi \textbf{in parallelo}: \begin{align*}
      h[n] = h_{1}[n] + h_{2}[n] \to y[n] = x[n] \otimes \Big[ h_{1}[n] + h_{2}[n] \Big]
      \end{align*}
  \end{itemize}
\item
  Risposta in frequenza;

  \begin{enumerate}
  \def\labelenumii{\arabic{enumii}.}
  \tightlist
  \item
    La risposta in frequenza di un SLS a tempo discreto è la
    \textbf{trasformata di Fourier} della risposta impulsiva \(h[n]\)
    del sistema stesso \[
     \overline{H}(f) = \sum_{n=-\infty}^{\infty} h[n] \ e^{-j2\pi nfT}
     \]
  \item
    Inoltre \(\overline{H}(f)\) è pari al rapporto tra le trasformate
    \(\overline{Y}(f)\) e \(\overline{X}(f)\) rispettivamente della
    sequenza in uscita \(y[n]\) e in ingresso \(x[n]\): \[
     \overline{H}(f)=\frac{\overline{Y}(f)}{\overline{X}(f)}
     \]
  \item
    La risposta in frequenza è data dal rapporto fra la sequenza di
    uscita y{[}n{]} e quella di ingresso x{[}n{]} quando x{[}n{]} è una
    oscillazione complessa alla frequenza f: \[
     \overline{H}(f) = \frac{y[n]}{x[n]}\Bigg|_{x[n]=e^{j2\pi nfT}}
     \] È possibile inoltre definire, data la risposta in frequenza
    \(\overline{H}(f)\), la \emph{risposta in ampiezza}
    \(\overline{A}(f)=|\overline{H}(f)|\), la quale determina la
    \textbf{selettività} di un SLS e la sua \emph{risposta in fase}
    \(\overline{\theta}(f)=\phase{\overline{H}(f)}\)
  \end{enumerate}
\item
  Filtri a tempo discreto.

  La \textbf{condizione di non distorsione} viene riformulata come
  segue: \[
  y[n]=Kx[n-n_0]
  \] \(K\) ed \(n_0\) rappresentano rispettivamente il guadagno ed il
  ritardo del sistema. Nel dominio della frequenza, questa condizione si
  traduce nei due seguenti requisiti per la risposta in ampiezza e la
  risposta in fase: \[
  \overline{A}(f) = K, \ \ \overline{\theta}(f)=-2\pi fn_{0}T
  \] È sufficiente che queste condizioni siano verificate nell'ambito
  della banda del segnale per garantire assenza di distorsioni

  Le caratteristiche di selettività di un filtro a tempo discreto con
  risposta in frequenza (che è una funzione periodica di periodo 1/T)
  sono determinate dall'andamento della sua risposta in ampiezza in un
  solo periodo della funzione (as esempio
  \(\Big[-\frac{1}{2T}, \frac{1}{2T}\Big]\)).

  Le basse frequenze sono sempre prossime alla frequenza nulla, mentre
  le alte sono comunque prossime al limite superiore dell'intervallo,
  ovvero \(\frac{1}{2T}\):

  \begin{itemize}
  \tightlist
  \item
    \textbf{passa basso} \begin{align*}
      h_{\text{LP}}[n] &= 2BT \mathop{\mathrm{sinc}}(2nBT) \Longleftrightarrow \overline{H}_{\text{LP}} = \frac{1}{T}\sum_{k=-\infty}^{\infty} T \cdot \operatorname{rect}\left(\frac{f-\frac{k}{T}}{2 B}\right)
      \end{align*}
  \item
    \textbf{passa alto} \begin{align*}
      h_{\text{HP}}[n] = \delta[n] -2BT\mathop{\mathrm{sinc}}(2nBT) \Longleftrightarrow \overline{H}_{\text{HP}}(f) = 1- \overline{H}_{\text{LP}}(f) 
      \end{align*} I filtri ideali \textbf{non sono causali!}
  \end{itemize}
\end{enumerate}

\subsection{Quantizzazione (per alcune di queste domande guardare
quesiti)}\label{quantizzazione-per-alcune-di-queste-domande-guardare-quesiti}

\begin{enumerate}
\def\labelenumi{\arabic{enumi}.}
\setcounter{enumi}{29}
\item
  Formule e definizioni (passo, dinamica D, bit B, fattore di scala
  A\ldots);

  L'operazione di quantizzazione rende \textbf{discreta} l'ampiezza dei
  campioni associando loro un valore di ampiezza scelto da un insieme
  finito di possibili \emph{valori} (detti livelli di quantizzazione:
  essi sono i ``rappresentanti'' di intervalli di quantizzazione
  contigui: gli estremi di questi intervalli sono detti ``soglie''). \[
  \hat{x}(nT) = \text{Q}\Big[x(nT)\Big]
  \] La quantizzazione è un'operazione \emph{lossy}, in quanto essendo
  un'operazione irreversibile, una volta quantizzato il segnale
  l'informazione originale non potrà essere più recuperata, commettendo
  un errore \[
  e(nT)=\hat{x}(nT)-x(nT)
  \] Per effettuare l'operazione di quantizzazione dividiamo
  l'intervallo di variazione di ampiezza dei suoi campioni in intervalli
  di quantizzazione contigui \((x_i, x_{i+1})\), dove gli estremi
  rappresentano le \emph{soglie} di quantizzazione.

  Al momento della progettazione del quantizzatore sono fissate sia le
  soglie che i livelli!

  L'operazione quindi consiste nel \textbf{selezionare l'intervallo più
  corretto per ogni campione} \(x(nt)\) e associare al suo interno un
  valore \(\hat{x}_i\) detto \emph{livello} dell'intervallo selezionato.

  \begin{itemize}
  \item
    ``Definizioni'':

    \begin{itemize}
    \tightlist
    \item
      \textbf{Passo}: indicato con \(\Delta\), rappresenta la distanza
      tra livelli (e soglie) di quantizzazione.
    \item
      \textbf{bit}: indicato con \(B\), serve a determinare il numero di
      possibili livelli di quantizzazione, rappresentati con notazione
      binaria, pari a \(2^{B}\)
    \item
      \textbf{dinamica}: indicata con \(D\) rappresenta l'ampiezza
      dell'intervallo di valori che i livelli di quantizzazione riescono
      a coprire.
    \end{itemize}
  \item
    Quantizzatori uniformi

    Ottenuti imponendo una distanza costante tra le soglie e i livelli
    (\(\Delta\) costante).

    Per avere una \emph{buona rappresentazione del segnale}, la dinamica
    \(D \approx\) \textbf{intervallo variazione ampiezza dei campioni}!
    \[
      D > X_{\text{max}}-X_{\text{min}}
      \] In un processo \emph{aleatorio gaussiano}, i cui campioni
    seguono la distribuzione di probabilità Gaussiana (quindi con un
    intervallo di variazione \emph{illimitato}), si rende necessario
    ipotizzare un intervallo di variazione dei campioni \textbf{finito}
    e di dimensione tale da rendere \emph{minima} la probabilità che
    esca da tale intervallo (\textbf{overflow}). Considerando il caso di
    valor medio \emph{nullo}: \begin{align*}
      E\Big[f(x)\Big]=0 &\to \text{gli intervalli sono: }
      \left\{\begin{array}{cl}
      [-3\sigma, 3\sigma] & \approx 95,45 \% \\ \relax 
      [-4\sigma, 4\sigma] & \approx 99,73 \%
      \end{array} \right. \to \Delta > 8\sigma
      \end{align*}

    Per far sì che il passo non sia né eccessivamente grande (livelli di
    quantizzazione usati molto minori rispetto a quelli a disposizione),
    né troppo piccolo (commessi errori rilevanti (di overflow), quando
    si quantizzano campioni al di fuori della dinamica del
    quantizzatore) \(\Delta, D, B\) sono legati secondo: \[
      \Delta=\frac{D}{2^B}
      \]
  \end{itemize}
\item
  Tipologie di quantizzatori (midrise, midtread, arrotondamento e
  troncamento);

  Vedi risposte 33, 34 quesiti
\item
  Errore di quantizzazione e modello;

  L'errore di quantizzazione può essere visto come una sequenza che
  \emph{si somma} (modello additivo) al segnale campionato: \[
  e(nT)=\hat{x}(nt)-x(nT) \Rightarrow \hat{x}(nT) = e(nT)+x(nT)
  \] Modelliamo quindi \(e(nT)\) come un processo aleatorio, indicandolo
  come \textbf{rumore di quantizzazione} e con le seguenti ipotesi:

  \begin{enumerate}
  \def\labelenumii{\arabic{enumii}.}
  \item
    \(e(nT)\) sia un processo stazionario in senso lato: quindi media,
    potenza e varianza \emph{costanti} e \textbf{non} dipendono da n
    (distanza tra i campioni);
  \item
    che la densità di probabilità dell'\textbf{ampiezza dell'errore di
    quantizzazione} sia di tipo \textbf{uniforme}, permettendo di
    valutare tali costanti distinguendo i casi di quantizzazione per
    troncamento e per arrotondamento:

    \begin{itemize}
    \tightlist
    \item
      densità probabilità errore troncamento: \begin{align*}
          P_e(e)&= \left\{
          \begin{array}{ll}
          \frac{1}{\Delta}& -\Delta<e\leq 0\\
          0 & \text{altrove}
          \end{array}\right. \to E\Big[e(nT)\Big]= 0 = E\Big[e^{2}(nT)\Big] = \frac{\Delta^{2}}{12} \\
          E[e(nT)]&=\int\limits_{-\Delta}^{0}e\ p_{e}(e) \,de=\frac{1}{\Delta}\cdot\frac{e^{2}}{2}\Bigg|_{-\Delta}^{0}=-\frac{\Delta}{2} \\
          E[e^{2}(nT)]&=\int\limits_{-\Delta}^{0}e^{2} p_{e}(e) de=\frac{\Delta^{2}}{3}=\frac{1}{\Delta} \frac{e^{3}}{3}\Bigg|_{-\Delta}^{0} \\
          \sigma_{e}^{2}&=E[(e(nT)-E[e(nT)])^{2}]=E[e^{2}(nT)]-\Big(E[e(nT)]\Big)^{2} =\\
          &= \int\limits_{-\Delta}^{0}\left(e+\frac{\Delta}{2}\right)^{2} p_{e}(e) de=\frac{\Delta^{2}}{12}
          \end{align*}
    \item
      densità errore arrotondamento: \begin{align*}
          p_e(e)&=\begin{cases}\frac{1}{\Delta}&-\frac{\Delta}{2}<e\leq\frac{\Delta}{2}\\0&\text{altrove,}\end{cases}\\
          E[e(nT)]&=\int\limits_{-\frac{\Delta}{2}}^{\frac{\Delta}{2}}e\ p_e(e) \,de=\frac{1}{\Delta}\frac{e^{2}}{2}\Bigg|_{-\frac{\Delta}{2}}^{\frac{\Delta}{2}}=0\\
          E[e^2(nT)]&=\sigma_{e}^2=\int\limits_{-\frac{\Delta}{2}}^{\frac{\Delta}{2}}e^2\ p_e(e) \,de=\frac{1}{\Delta}\frac{e^3}{3}\Bigg|_{-\frac{\Delta}{2}}^{\frac{\Delta}{2}}=\frac{\Delta^2}{12}
      \end{align*}
    \end{itemize}
  \item
    \(\{e(nT)\}\) incorrelato con processo \(\{x(nT)\}\);

    \(\displaystyle E\Big[\{x(nT)\} \{e(nT)\}\Big] = E\Big[\{x(nT)\}\Big]\cdot E\Big[\{e(nT)\}\Big]\)
  \item
    I campioni del processo \(\{e(nT)\}\) sono \textbf{incorrelati} tra
    loro;

    Per la quantizzazione con troncamento si ha: \begin{align*}
    E\Big[e(nT)\ e\Big((n+m)T\Big)\Big]=\left\{\begin{array}{ll}E[e^2(nT)]=\frac{\Delta^2}{3}&m=0\\
    \\E[e(nT)]E\Big[e\Big((n+m)T\Big)\Big]=\frac{\Delta}{2}\cdot\frac{\Delta}{2} =\frac{\Delta^2}{4}&m\neq0,\end{array}\right.
    \end{align*} mentre per la quantizzazione con arrotondamento abbiamo
    \begin{align*}
    E\Big[e(nT)e\Big((n+m)T\Big)\Big]=\left\{\begin{array}{ll}E[e^2(nT)]=\sigma_e^2=\frac{\Delta^2}{12}&m=0\\\\0&m\ne0.\end{array}\right.
    \end{align*} In questo caso, l'errore di quantizzazione è un
    \textbf{processo bianco}.
  \end{enumerate}
\item
  Definizione Signal To Noise Ratio (SNR) e formule.

  È il rapporto tra la \emph{potenza del segnale} e la \emph{potenza
  dell'errore di quantizzazione}: \[
  \text{SNR}_q = \frac{S}{\sigma^2_e}
  \] Ciò vale con l'ipotesi di un quantizzatore che utilizzi
  l'arrotondamento. \(S\) rappresenta la potenza del segnale (è
  necessario conoscerla oltre ai parametri del quantizzatore)\ldots{} e
  \(\Delta=\frac{D}{2^{B}}\): \[
  \text{SNR}_{q}=\frac{S}{\sigma_e^2}=\frac{S}{\frac{1}{12}\Delta^2}=\frac{S}{\frac{1}{12}(\frac{D}{2^B})^2}=\frac{12\cdot S\cdot2^{2B}}{D^2}
  \] Esprimendo \(\text{SNR}_{q}\) in scala logaritmica \begin{align*}
  \text{SNR}_{q, \text{dB}} &=10\log_{10}SNR_{\mathrm{q}}=10\log_{10}\left(\frac{12\cdot S\cdot2^{2B}}{D^{2}}\right) \\
  &=(20 \log_{10}2)B+10\log_{\mathbf{1}0}\left(\frac{12S}{D^2}\right) \\
  &\approx6.02B+10\log_{10}\left(\frac{12S}{D^2}\right)\ \mathrm{dB}. 
  \end{align*}
\end{enumerate}

\end{document}
