% Options for packages loaded elsewhere
\PassOptionsToPackage{unicode}{hyperref}
\PassOptionsToPackage{hyphens}{url}
%
\documentclass[
]{article}
\usepackage{amsmath,amssymb}
\usepackage{iftex}
\ifPDFTeX
  \usepackage[T1]{fontenc}
  \usepackage[utf8]{inputenc}
  \usepackage{textcomp} % provide euro and other symbols
\else % if luatex or xetex
  \usepackage{unicode-math} % this also loads fontspec
  \defaultfontfeatures{Scale=MatchLowercase}
  \defaultfontfeatures[\rmfamily]{Ligatures=TeX,Scale=1}
\fi
\usepackage{lmodern}
\ifPDFTeX\else
  % xetex/luatex font selection
\fi
% Use upquote if available, for straight quotes in verbatim environments
\IfFileExists{upquote.sty}{\usepackage{upquote}}{}
\IfFileExists{microtype.sty}{% use microtype if available
  \usepackage[]{microtype}
  \UseMicrotypeSet[protrusion]{basicmath} % disable protrusion for tt fonts
}{}
\makeatletter
\@ifundefined{KOMAClassName}{% if non-KOMA class
  \IfFileExists{parskip.sty}{%
    \usepackage{parskip}
  }{% else
    \setlength{\parindent}{0pt}
    \setlength{\parskip}{6pt plus 2pt minus 1pt}}
}{% if KOMA class
  \KOMAoptions{parskip=half}}
\makeatother
\usepackage{xcolor}
\setlength{\emergencystretch}{3em} % prevent overfull lines
\providecommand{\tightlist}{%
  \setlength{\itemsep}{0pt}\setlength{\parskip}{0pt}}
\setcounter{secnumdepth}{-\maxdimen} % remove section numbering
\ifLuaTeX
\usepackage[bidi=basic]{babel}
\else
\usepackage[bidi=default]{babel}
\fi
\babelprovide[main,import]{italian}
% get rid of language-specific shorthands (see #6817):
\let\LanguageShortHands\languageshorthands
\def\languageshorthands#1{}
\usepackage{cancel}
\usepackage{steinmetz}
\usepackage{derivative}
\usepackage{mathtools}
\DeclareMathOperator{\sinc}{sinc}
\DeclareMathOperator{\rect}{rect}
\ifLuaTeX
  \usepackage{selnolig}  % disable illegal ligatures
\fi
\IfFileExists{bookmark.sty}{\usepackage{bookmark}}{\usepackage{hyperref}}

\usepackage{geometry}
\geometry{
	a4paper,
	total={170mm,257mm},
	left=20mm,
	top=20mm,
}
\usepackage{bookmark}
\IfFileExists{xurl.sty}{\usepackage{xurl}}{} % add URL line breaks if available
\urlstyle{same}
\hypersetup{
  pdflang={it},
  hidelinks,
  pdfcreator={LaTeX via pandoc}}

\author{}
\date{}

\begin{document}

\section{Lista teoremi Teoria dei
Segnali}\label{lista-teoremi-teoria-dei-segnali}

\subsection{Introduzione}\label{introduzione}

\begin{enumerate}
\def\labelenumi{\arabic{enumi}.}
\item
  \(x(t)\) segnale periodico di periodo \(T_0\) e ha potenza media su un
  intervallo finita, allora ha \(\bar{P}\) finita e calcolabile sul
  periodo:

  Per definizione
  \(\bar{P}= \frac{1}{T} \int_{-\frac{T}{2}}^{\frac{T}{2}} |x(t)|^{2} \,dt\).
  Scegliamo come periodo \(NT_0\), in quanto se \(N\to \infty\) vale
  come \(T\to \infty\).

  Quindi
  \(\displaystyle \lim_{N \to \infty} \frac{1}{NT_0} \int_{-\frac{NT_0}{2}}^{\frac{NT_0}{2}} |x(t)|^{2} \,dt\)
  equivale a \(N\) integrali cui si aggiunge un \(T_0\) ogni volta:

  \[
   \displaystyle \lim_{N \to \infty} \frac{1}{\cancel{N}T_0} \cdot \cancel{N} \Big \{ \int_{-\frac{T_0}{2}}^{\frac{T_0}{2}} |x(t)|^{2} \,dt \} = \bar{P} = \frac{1}{T_0} \int_{-\frac{T_0}{2}}^{\frac{T_0}{2}} |x(t)|^{2} \,dt
   \]
\end{enumerate}

\subsection{Segnali periodici a tempo
continuo}\label{segnali-periodici-a-tempo-continuo}

\subsubsection{Serie di Fourier}\label{serie-di-fourier}

\begin{enumerate}
\def\labelenumi{\arabic{enumi}.}
\setcounter{enumi}{1}
\item
  Da forma polare a complessa (o rettangolare)

  la forma polare della serie di Fourier è data da: \[
   x(t) = A_0 + 2 \sum_{k=1}^{\infty} A_{k} \cos (2\pi kf_{0}t + \theta_{K})
   \] \[
   = A_0 + \cancel{2} \sum_{k=1}^{\infty} A_{k} \frac{e^{j(2\pi kf_{0}t + \theta_{K})}+e^{-j(2\pi kf_{0}t + \theta_{K})}}{\cancel{2}} \to \text{ uso formula di Eulero per il coseno}
   \] \[
   = A_0 + \sum_{k=1}^{\infty} A_{k} e^{j2\pi kf_{0}t + \theta_{K}}  + \sum_{k=1}^{\infty} A_{k} e^{-j2\pi kf_{0}t + \theta_{K}} \to \text{ separo le due esponenziali}
   \] \[
   = x_0 + \sum_{k=1}^{\infty} A_{k} e^{j\theta_{k}} e^{j2\pi kf_{0}t}  + \sum_{k=1}^{\infty} A_{k} e^{-j\theta_{k}} e^{-j2\pi kf_{0}t}  \to \text{ raggruppo le sommatorie}
   \] \[
   = x_0 + \sum_{k=1}^{\infty} X_k e^{j2\pi kf_{0}t}  + \sum_{k=1}^{\infty} X_k e^{-j2\pi kf_{0}t}
   \] \[
   \Rightarrow x(t) = \sum_{k=-\infty}^{\infty} X_k \ e^{j2\pi kf_{0}t} \text{ forma complessa della serie di Fourier}
   \]
\item
  Come si calcolano i coefficienti \(X_n\)?

  Partendo dalla forma complessa, moltiplico a destra e a sinistra per
  \(e^{-j2\pi kf_{0}t}\), integrando sul periodo \(T_0\). \[
   \int_{-\frac{T_0}{2}}^{\frac{T_0}{2}} x(t)e^{-j2\pi kf_{0}t} \,dt = \int_{-\frac{T_0}{2}}^{\frac{T_0}{2}} \sum_{k=-\infty}^{\infty} X_k \ e^{j2\pi kf_{0}t}  e^{-j2\pi nf_{0}t} \,dt
   \] Porto fuori la sommatoria e raccolgo \(e\): per ipotesi la serie
  converge. \[
   \int_{-\frac{T_0}{2}}^{\frac{T_0}{2}} x(t)e^{-j2\pi kf_{0}t} \,dt = \sum_{k=-\infty}^{\infty} X_k  \int_{-\frac{T_0}{2}}^{\frac{T_0}{2}} e^{j2\pi (k-n)f_{0}t} 
   \] L'integrale al secondo membro viene calcolato per \(k\neq n\) \[
   \to \int_{-\frac{T_0}{2}}^{\frac{T_0}{2}} e^{j2\pi (k-n)f_{0}t} = \frac{e^{j2\pi (k-n)f_{0}t}}{j2\pi (k-n)f_{0}} \Big|_{-\frac{T_0}{2}}^{\frac{T_0}{2}}=
   \] \[
   \frac{e^{j\cancel{2}\pi (k-n)\cancel{f_{0}}\frac{\cancel{T_0}}{\cancel{2}}} - e^{-j\cancel{2}\pi (k-n)\cancel{f_{0}}\frac{\cancel{T_0}}{\cancel{2}}}}{2j \cdot \pi (k-n)f_{0}} \to \text{ uso formula di Eulero per il seno}
   \] \[
   \frac{\sin (\pi (k-n))}{\pi (k-n)f_{0}} = \left\{ \begin{array}{cl}
   k=n\to T_0  \\
   k\neq n \to 0
   \end{array} \right.
   \to \text{ sostituiamo questo risultato}
   \] \[
   \int_{-\frac{T_0}{2}}^{\frac{T_0}{2}} x(t)e^{-j2\pi nf_{0}t} \,dt = X_{n}T_{0} \Rightarrow X_{n} = \frac{1}{T_0}\int_{-\frac{T_0}{2}}^{\frac{T_0}{2}} x(t)e^{-j2\pi nf_{0}t} \,dt 
   \]
\item
  Forma rettangolare dalla forma polare \[
   x(t) = A_{0} + 2 \sum_{k=1}^{\infty} A_{k} \cos(2\pi k f_{0}t + \theta_{k})
   \] usiamo la formula di addizione
  \(\cos(\alpha + \beta) = \cos\alpha\cos\beta - \sin\alpha\sin\beta\)
  \[
   x(t) = A_{0} + 2 \sum_{k=1}^{\infty} (A_{k}(\cos(2\pi k f_{0})\cos({\theta_k}) - \sin(2\pi k f_{0})\sin({\theta_k})))
   \] \[
   x(t) = a_{0} + 2 \sum_{k=1}^{\infty} [a_{k} \cos(2\pi k f_{0}) - b_{k}\sin(2\pi k f_{0})]
   \] sapendo che
  \(a_{0} = A_{0}, \ a_{k} = A_{k}\cos(\theta_k), \ b_{k} = B_{k}\sin(\theta_k)\).

  Abbiamo quindi ottenuto la forma rettangolare della serie di Fourier,
  dove si nota che un segnale \emph{periodico} \(x(t)\) può essere
  espresso tramite una \textbf{somma di seni e coseni}.

  Il coefficiente \(X_n\) può essere espresso anche come: \[
   X_k = \frac{1}{T_0} \int_{[T_0]} x(t) (\cos(2\pi kf_{0}t) - j\sin(2\pi kf_{0}t))
   \] \[
   X_k = a_k +jb_k = A_k \cos(\theta_k) + jA_k\sin(\theta_k) = A_k e^{j\theta_k}
   \]
\item
  Criterio di Dirichlet (per \(x(t)\) periodico):

  È una serie di condizioni che se incontrate sono sufficienti per poter
  sviluppare un dato segnale \(x(t)\) in serie di Fourier:

  \begin{itemize}
  \tightlist
  \item
    \(x(t)\) deve essere \emph{assolutamente integrabile sul periodo}:
    ovvero \((\int_{[T_0]}|x(t)| \,dt < \infty)\)
  \item
    \(x(t)\) deve essere \emph{continua} (o avere un numero
    \emph{finito} di discontinuità di prima specie)
  \item
    \(x(t)\) deve essere \emph{derivabile sul periodo} \(T_0\), escluso
    al più un numero finito di punti, dove comunque esiste
    \textbf{finita} sia la derivata destra che la derivata sinistra

    \begin{itemize}
    \tightlist
    \item
      quest'ultima ipotesi è equivalente a: \(x(t)\) presenta un numero
      finito di massimi e minimi nel periodo La serie \textbf{converge}
      al valore assunto da \(x(t)\) dove \emph{continua} e alla
      semisomma dei limiti sinistro e destro se discontinua.
    \end{itemize}
  \end{itemize}
\end{enumerate}

\subsubsection{Spettro di un segnale periodico e
reale}\label{spettro-di-un-segnale-periodico-e-reale}

\paragraph{Proprietà}\label{proprietuxe0}

\begin{enumerate}
\def\labelenumi{\arabic{enumi}.}
\setcounter{enumi}{5}
\item
  Simmetria Hermitiana dello spettro reale:

  I coefficienti \(X_k\) sono generalmente quantità complesse del tipo
  \begin{gather*}
   X_k = |X_{k}|e^{j\phase{X_{k}}}
   \end{gather*} \(X_{k}\)può essere rappresentata tramite spettro di
  ampiezza e spettro di fase, discreti (esiste solo in corrispondenza
  delle armoniche \(k\)) \[
   X_k = \frac{1}{T_0} \int_{-\frac{T_0}{2}}^{\frac{T_0}{2}} x(t) e^{-j2\pi kf_{0}t} \,dt
   \] Analizziamone il coniugato \(X^{*}_{k}\): \[
   X_k^{*} = \Big(\frac{1}{T_0} \int_{-\frac{T_0}{2}}^{\frac{T_0}{2}} x(t) e^{-j2\pi kf_{0}t} \,dt \Big)^{*} = \frac{1}{T_0} \int_{-\frac{T_0}{2}}^{\frac{T_0}{2}} x(t)^{*} e^{+j2\pi kf_{0}t} \,dt = \frac{1}{T_0} \int_{-\frac{T_0}{2}}^{\frac{T_0}{2}} x(t) e^{-j2\pi (-k)f_{0}t} \,dt
   \] È da notare come \(x(t)^{*}=x(t)\), dal momento che il segnale
  \(x(t)\) è reale.

  Quindi \(X_{k}^{*}=X_{-k}\): i coefficienti \(X_k\) di un segnale
  \emph{reale} sono \textbf{simmetrici hermitiani}, ossia hanno \emph{lo
  stesso modulo e fase opposta} \[
   X_{-k}=X_{k}^{*} \Longleftrightarrow \left\{ \begin{array}{cl}
   |X_k|=|X_{-k}| \ \text{ stesso modulo}  \\
   \phase{X_{k}} = -\phase{X_{-k}} \ \text{ fase opposta}
   \end{array} \right.
   \] In definitiva per un segnale reale: - lo spettro d'ampiezza è
  \textbf{simmetrico} rispetto a \(k \to\) pari - lo spettro di fase è
  \textbf{antisimmetrico} rispetto a \(k \to\) dispari
\item
  Linearità dello spettro reale:

  Se \(x(t)\) e \(y(t)\) sono due segnali con periodo \(T_0\)
  \emph{reali} allora vale: \[
   z(t)=ax(t)+by(t) \Longleftrightarrow Z_{k} = aX_{k} + bY_{k}
   \] Somma di \emph{oscillazioni} alle (o con?) le stesse frequenze dei
  segnali \(x(t)\) e \(y(t)\). \[
   Z_{k} = \frac{1}{T_0} \int_{-\frac{T_0}{2}}^{\frac{T_0}{2}} z(t) e^{-j2\pi kf_{0}t} \,dt = \frac{1}{T_0} \int_{-\frac{T_0}{2}}^{\frac{T_0}{2}} (ax(t)+by(t)) e^{-j2\pi kf_{0}t} \,dt 
   \] \[
   = \frac{a}{T_0} \int_{-\frac{T_0}{2}}^{\frac{T_0}{2}} x(t) e^{-j2\pi kf_{0}t} \,dt + \frac{b}{T_0} \int_{-\frac{T_0}{2}}^{\frac{T_0}{2}} y(t) e^{-j2\pi kf_{0}t} \,dt = aX_{k}+bY_{k}
   \]
\item
  Parità e disparità del segnale

  \begin{itemize}
  \item
    Se \(x(t)\) è \textbf{pari}, allora il coefficiente
    \(X_{k} = X_{-k}\); se il segnale è anche \textbf{reale} vale
    \(X_{k}=X_{-k}=X^{*}_{k} \Longleftrightarrow X_k \in \mathbb{R}\).

    \(X_k = X_{-k}\) (con un cambio di variabile
    \(\alpha=-t \to \,dt=-\,d\alpha\)). \[
      X_{k} = \frac{1}{T_0} \int_{-\frac{T_0}{2}}^{\frac{T_0}{2}} x(t) e^{-j2\pi kf_{0}t} \,dt \Longleftrightarrow X_{-k} = \frac{1}{T_0} \int_{-\frac{T_0}{2}}^{\frac{T_0}{2}} x(t) e^{-j2\pi (-k)f_{0}t} \,dt 
      \] Utilizziamo il cambio di variabile \[
      X_{-k} = \frac{1}{T_0} \int_{\frac{T_0}{2}}^{-\frac{T_0}{2}} x(-\alpha) e^{-j2\pi (\cancel{-}(\cancel{-}\alpha)) kf_{0}} -\,d\alpha = \frac{1}{T_0} \int_{-\frac{T_0}{2}}^{\frac{T_0}{2}} x(\alpha) e^{-j2\pi\alpha kf_{0}} \,d\alpha =
      \] \[
      -\frac{1}{T_0} \int_{\frac{T_0}{2}}^{-\frac{T_0}{2}} x(\alpha) e^{-j2\pi\alpha kf_{0}} \,d\alpha = X_k
      \] dato che il segnale \(\in \mathbb{R}\) lo possiamo
    rappresentare come (perché essendo reale ha fase nulla?): \[
      x(t) = X_{0} + 2\sum_{k=1}^{\infty}X_{k}\cos(2\pi kf_{0}t)
      \] Dimostrazione: \[
      x(t) = \sum_{k=-\infty}^{\infty} X_{k} e^{j2\pi kf_{0}t} = X_{0} + \sum_{k=1}^{\infty} X_{k} e^{j2\pi kf_{0}t} + \sum_{k=-\infty}^{-1} X_{k} e^{j2\pi kf_{0}t} =
      \] \[
      = X_{0} + \sum_{k=1}^{\infty} X_{k} e^{j2\pi kf_{0}t} + \sum_{k=1}^{\infty} X_{-k} e^{-j2\pi kf_{0}t} =
      \] \[
      = X_{0} + \sum_{k=1}^{\infty} X_{k} e^{j2\pi kf_{0}t} + \sum_{k=1}^{\infty} X_{k} e^{-j2\pi kf_{0}t} =
      \] \[
      = X_{0} + 2 \sum_{k=1}^{\infty} X_{k} \frac{e^{j2\pi kf_{0}t}+ e^{-j2\pi kf_{0}t}}{2} = X_{0} + 2 \sum_{k=1}^{\infty} X_{k} \cos{(2\pi k f_{0}t)}
      \] Da ciò deduco che un segnale reale e pari è esprimibile in
    serie di soli \emph{coseni} (i quali sono a loro volta pari).

    Possiamo inoltre scrivere i coefficienti \(X_k\) in modo
    semplificato, data la \emph{parità} del segnale: \begin{gather*}
      X_k = \frac{1}{T_0}\int_{-\frac{T_0}{2}}^{\frac{T_0}{2}} x(t) e^{-j2\pi kf_{0}t} \,dt = \\
      \frac{1}{T_0}\int_{-\frac{T_0}{2}}^{\frac{T_0}{2}} \underbrace{x(t)}_{pari}\cdot\underbrace{\cos{(2\pi kf_{0}t)}}_{pari} \,dt - \frac{j}{T_0}\int_{-\frac{T_0}{2}}^{\frac{T_0}{2}} \underbrace{x(t)}_{pari}\cdot\underbrace{\sin{(2\pi kf_{0}t)}}_{dispari}\,dt=
      \\
      \frac{2}{T_0} \int_{0}^{\frac{T_0}{2}} x(t)\cdot\cos{(2\pi kf_{0}t)} \,dt - 0
      \end{gather*}
  \item
    se \(x(t)\) è \textbf{dispari}, allora anche i coefficienti \(X_k\)
    saranno dispari. Inoltre, dato che \(x(t)\in\mathbb{R}\), \(X_k\)
    sarà un \textbf{immaginario puro}, ed \[
      x(t)=2j \sum_{k=1}^{\infty} X_{k} \sin{(2\pi k f_{0}t)} \text{ e } X_k=-\frac{2j}{T_0}\int_{0}^{\frac{T_0}{2}}x(t)\sin(2\pi kf_0 t) \,dt
      \] Dimostrazione:

    \begin{itemize}
    \tightlist
    \item
      Dato che \(x(t)\in\mathbb{R}\), \(X_k\), allora vale
      \(X_{-k} = -X_{k} = X_{k}^{*} \Rightarrow X_{k}^{*}=-X_k\), quindi
      è un immaginario puro!
    \item
      Per \(X_k\): \[
        X_k = \frac{1}{T_0}\int_{-\frac{T_0}{2}}^{\frac{T_0}{2}} x(t) e^{-j2\pi kf_{0}t} \,dt =
        \] \begin{gather*}
        \frac{1}{T_0}\int_{-\frac{T_0}{2}}^{\frac{T_0}{2}} \underbrace{x(t)}_{\text{dispari}}\cdot\underbrace{\cos(2\pi kf_{0}t)}_{\text{pari}} \,dt - \frac{j}{T_0}\int_{-\frac{T_0}{2}}^{\frac{T_0}{2}} \underbrace{x(t)}_{\text{dispari}}\cdot\underbrace{\sin(2\pi kf_{0}t)}_{\text{dispari}}\,dt=
        \\
        -\frac{j}{T_0} \int_{0}^{\frac{T_0}{2}} x(t)\cdot\sin{(2\pi kf_{0}t)} \,dt 
        \end{gather*}
    \end{itemize}
  \item
    \textbf{Note varie}

    \begin{itemize}
    \tightlist
    \item
      Se \(x(t)\) è pari i suoi coefficienti \(X_k\) sono reali e lo
      spettro di fase vale \(0\) o \(\pm \pi\); mentre se \(x(t)\) è
      dispari i suoi coefficienti \(X_k\) sono immaginari puri e lo
      spettro di ampiezza non viene toccato: un segnale dispari è solo
      ``spostato'' nel tempo.
    \item
      È da notare come la diversa velocità di un segnale dipenda dal suo
      andamento temporale: le variazioni brusche comportano la presenza
      di \textbf{armoniche}{[}\^{}1{]} con \(k\) più elevato per
      rappresentare la velocimento alta(?):

      \begin{itemize}
      \tightlist
      \item
        più il segnale è regolare meno armoniche sono necessarie per
        ``ricreare'' il segnale

        \begin{itemize}
        \tightlist
        \item
          \(\frac{1}{k} \to\) funzioni discontinue: dente di sega
          ideale, onda quadra, onda quadra ``antisimmetrica'', rect
        \item
          \(\frac{1}{k^2} \to\) funzioni continue a derivata
          discontinua: onda triangolare. {[}\^{}1{]}: TODO: definire
          meglio armoniche
        \end{itemize}
      \end{itemize}
    \end{itemize}
  \end{itemize}
\end{enumerate}

\subsection{Segnali aperiodici a tempo
continuo}\label{segnali-aperiodici-a-tempo-continuo}

\subsubsection{Trasformata continua di
Fourier}\label{trasformata-continua-di-fourier}

\begin{quote}
Una funzione non periodica, definita tra \(-\infty\) e \(\infty\), può
essere rappresentata come \textbf{somma} di \textbf{infinite armoniche
semplici} di ampiezza \emph{infinitesima} e di frequenza variabile con
continuità tra \(-\infty\) e \(\infty\)
\end{quote}

\begin{enumerate}
\def\labelenumi{\arabic{enumi}.}
\setcounter{enumi}{8}
\item
  Dal segnale periodico al segnale aperiodico\ldots{}

  Partiamo dall'impulso rettangolare \emph{aperiodico}
  \(\mathop{\mathrm{rect}}{\frac{t}{T}}\): \begin{gather*}
   x(t) = \mathop{\mathrm{rect}}{\frac{t}{T}} \to x_{p}(t) = \sum \mathop{\mathrm{rect}}(\frac{t-nT_0}{T}) \text{ treno di impulsi rettangolari}
   \end{gather*} possiamo vedere \(x(t)\) come caso limite di \(x_p(t)\)
  con periodo \(T_0 \to \infty\) \[
   x(t) = \lim_{T_0 \to \infty} x_{p}(t)
   \]

  \begin{enumerate}
  \def\labelenumii{\arabic{enumii}.}
  \tightlist
  \item
    la frequenza diventa infinitesima \((f_0 = \frac{1}{T_0})\)
  \item
    si riduce la \emph{distanza tra le armoniche}, ossia si
    \textbf{infittisce} lo spettro;
  \item
    \(X_k=\frac{1}{T_0}\int_{-\frac{T_0}{2}}^{\frac{T_0}{2}} x_p(t) \ e^{-j2\pi kf_0 t}\,dt\),
    l'ampiezza assume valori \emph{sempre più piccoli}
  \end{enumerate}

  Usiamo il coefficiente \emph{modificato} \(X(f_0 k) = T_0 X_k\) per
  ovviare il problema. Riscriviamo \(x_p(t)\) e \(X_k\)\\
  \begin{gather*}
   x_{p}(t) = \sum_{k=-\infty}^{\infty}X(kf_0)\ e^{j2\pi kf_0 t} \cdot f_0 \to x(t) = \underbrace{\int_{-\infty}^{\infty}X(f)\ e^{j2\pi ft}\,df}_{\text{integrale di Fourier}}
   \end{gather*} Le armoniche si \emph{infittiscono talmente tanto} da
  non essere più distinte ma \textbf{continue}.

  \begin{gather*}
   X(kf_0) = T_0 \ X_k = \int_{-\frac{T_0}{2}}^{\frac{T_0}{2}}  x_{p}(t) \ e^{-j2\pi kf_0 t}\,dt \to \underbrace{X(f) = \int_{-\infty}^{\infty}X(f)\ e^{j2\pi ft}\,dt}_{\text{trasformata continua di Fourier}}
   \end{gather*}

  \(X(f)\) è una \textbf{funzione complessa della variabile continua
  \(f\)}, quindi è di spettro continuo.

  \begin{itemize}
  \tightlist
  \item
    Nota: \textbf{differenze tra segnali continui periodici e
    aperiodici}:

    \begin{itemize}
    \tightlist
    \item
      un segnale \emph{periodico} è rappresentato da componenti
      sinusoidali a frequenze in relazione \textbf{armonica} (multipli
      di \(f_0\), frequenza \emph{fondamentale} e ad ampiezza finita).
    \item
      un segnale \emph{aperiodico} è rappresentato con componenti
      sinusoidali di ampiezza \emph{infinitesima} \(|X(f)|\,df\) e
      frequenza \(f\) variabile con continuità su \(\mathbb{R}\); è un
      segnale periodico di periodo illimitato con \(f_0\) infinitesimo.
      Le armoniche discrete \emph{degenerano} nell'insieme continuo.
    \end{itemize}
  \end{itemize}
\item
  Criteri di esistenza per la trasformata continua di Fourier (TCF)

  \begin{enumerate}
  \def\labelenumii{\arabic{enumii}.}
  \tightlist
  \item
    \(X(f)\) esiste se il segnale \(x(t)\)
  \item
    Criteri di Dirichlet:

    \begin{enumerate}
    \def\labelenumiii{\arabic{enumiii}.}
    \tightlist
    \item
      la funzione deve essere assolutamente sommabile:
      \(\displaystyle \int_{-\infty}^{\infty} |x(t)| dt < +\infty\)
    \item
      se in qualunque intervallo finito \((t_1, t_2)\) è continua o
      presenta un numero finito di discontinuità di prima specie
    \item
      se in qualunque intervallo finito \((t_1, t_2)\) la funzione ha un
      numero finito di massimi e minimi.
    \end{enumerate}
  \end{enumerate}

  Allora \(x(t)\) è rappresentabile come TCF e \[
  x(t) = \int_{-\infty}^{\infty}X(f)\ e^{j2\pi fT} \,df =  \left\{ \begin{array}{cl}
  x(t) \ \text{ se continua}  \\
  \frac{x(t_{0}^{+})-x(t_{0}^{-})}{2}\ \text{ se discontinua}
  \end{array} \right.
  \] 
\item
  Simmetria Hermitiana della trasformata continua di Fourier

  Possiamo rappresentare \(X(f)\) in forma rettangolare: \[
  X(f) = Re(f) + Im(f) = \int_{-\infty}^\infty x(t) \cos(2\pi ft) \,dt - j\int_{-\infty}^\infty x(t) \sin(2\pi ft) \,dt
  \] \[
  \underbrace{Re(f)=Re(-f)}_{\text{pari}} \text{ e } \underbrace{Im(f)=-Im(-f)}_{\text{dispari}} \Longrightarrow X(f)=X^{*}(-f) \text{ simmetria hermitiana}
  \] infatti \(X(f)=Re(f)+jIm(f)=Re(-f)+jIm(f)=X^{*}(-f)\)

  \begin{itemize}
  \tightlist
  \item
    lo spettro di ampiezza è quindi \emph{pari} a quello di fase
    dispari.
  \end{itemize}
\item
  Parità e disparità:

  \begin{itemize}
  \tightlist
  \item
    se un segnale è \emph{reale e pari} \[
    X(f) = \int_{-\infty}^\infty x(t) \cos(2\pi ft) \,df=
    \left\{ \begin{array}{cl}
    Re(f) = 2\int_{0}^\infty x(t) \cos(2\pi ft) \,dt \\
    Im(f) = 0
    \end{array} \right.
    \] \(\to X(f) = Re(f) \to X(f)=X(-f)\) è reale e pari
  \item
    se un segnale è \emph{dispari e reale} \[
    X(f) = - \int_{-\infty}^\infty x(t) \sin(2\pi ft) \,dt=
    \left\{ \begin{array}{cl}
    Re(f) = 0 \\
    Im(f) = -2\int_{0}^\infty x(t) \sin(2\pi ft) 
    \end{array} \right.
    \] \(\to X(f)=jIm(f) \to X(f)=-X(f)\) è immaginaria pura e dispari
  \end{itemize}
\end{enumerate}

\paragraph{Proprietà della trasformata
continua}\label{proprietuxe0-della-trasformata-continua}

\begin{enumerate}
\def\labelenumi{\arabic{enumi}.}
\setcounter{enumi}{12}
\item
  \textbf{Linearità}

  Dati due segnali \(x_1(t)\) e \(x_2(t)\) con le loro trasformate
  continue di Fourier \(X_1(f)\) e \(X_2(f)\), allora se: \[
  x(t) = ax_1(t) + bx_2(t) \Longleftrightarrow X(f)=aX_1(f)+bX_2(f)
  \] con \(a,b\) costanti, \(X_1(f)=\text{TCF}[x_1(t)]\) e
  \(X_2(f)=\text{TCF}[x_2(t)]\)

  \begin{itemize}
  \tightlist
  \item
    Dimostrazione: \[
      X(f)= \int_{-\infty}^{\infty}x(t)\ e^{-j2\pi ft}\,dt = \int_{-\infty}^{\infty}(ax_1(t) + bx_2(t)) \ e^{-j2\pi ft}\,dt
      \] ma sappiamo che l'integrale è \emph{lineare}, quindi \[
      = a\int_{-\infty}^{\infty}x_1(t)\ e^{-j2\pi ft}\,dt + b\int_{-\infty}^{\infty}x_2(t)\ e^{-j2\pi ft}\,dt = aX_1(f)+bX_2(f)
      \]
  \end{itemize}
\item
  \textbf{Dualità}

  se \(x(t)\Longleftrightarrow X(f)\), allora
  \(X(t)\Longleftrightarrow x(-f)\):

  Se la trasformata continua di Fourier passa ad essere un \emph{segnale
  nel tempo}, allora \(x(-f)\) è la sua trasformata di Fourier. Abbiamo
  quindi una corrispondenza biunivoca tra la funzione e la sua
  trasformata.

  \begin{itemize}
  \item
    Esempio: \[
      \mathop{\mathrm{rect}}(\frac{t}{T})\Longleftrightarrow\mathop{\mathrm{sinc}}(fT)
      \] Ma se nel tempo ho un segnale \(\mathop{\mathrm{sinc}}(bT)\)
    qual è la sua trasformata?

    \(T\mathop{\mathrm{sinc}}(Tt) \Longleftrightarrow \mathop{\mathrm{rect}}({-\frac{f}{T}})\)
    da cui
    \(\mathop{\mathrm{sinc}}(Bt)\Longleftrightarrow \frac{1}{B}\mathop{\mathrm{rect}}(\frac{t}{B})\),
    dove \(B\) indica la banda.
  \item
    Dimostrazione: \[
      x(t) =\int_{-\infty}^{\infty}X(f)\ e^{j2\pi ft} \,df = \int_{-\infty}^{\infty}X(t)\ e^{j2\pi ft} \,dt 
      \] con uno scambio di variabili \(t\) con \(f\). Quindi: \[
      X(-f)=\int_{-\infty}^{\infty}x(t)\ e^{-j2\pi fT} \,dt 
      \] Da qui deriviamo che \(x(-f)\) è la trasformata di \(X(t)\)
  \end{itemize}
\item
  \textbf{Ritardo}

  Sia \(X(f)=\text{TCF}[x(t)]\): la trasformata di Fourier di \(x(t)\)
  ritardato nel tempo di una quantità \(t_0\) è pari a: \[
  x(t-t_0) \Longleftrightarrow X(f) \ e^{-j2\pi ft_0}
  \]

  \begin{itemize}
  \tightlist
  \item
    Dimostrazione:
  \end{itemize}

  Applichiamo a \(x(t-t_0)\) la definizione di TCF \[
  x(t-t_0) \Longleftrightarrow \int_{-\infty}^{\infty} x(t-t_0) \ e^{-j2\pi ft} \,dt = \Big|\alpha = t-t_0 \to t=\alpha +t_0
  \] \[
  x(t-t_0) \Longleftrightarrow \int_{-\infty}^{\infty} x(\alpha + t_0) e^{-j2\pi (\alpha +t_0)f} \,d\alpha = e^{-j2\pi ft_0} \int_{-\infty}^{\infty} x(\alpha)\ e^{-j2\pi f\alpha} =  e^{-j2\pi ft_0} \ X(f) 
  \]

  \begin{itemize}
  \tightlist
  \item
    Esempio: \[
    A\mathop{\mathrm{rect}}(\frac{t-\frac{T}{2}}{T}) \Longleftrightarrow AT\mathop{\mathrm{sinc}}(fT)e^{-j\cancel{2}\pi f\frac{T}{\cancel{2}}}
    \]
  \end{itemize}

  Se
  \(y(t)=x(t-t_0) \Rightarrow Y(f) = X(f) \ e^{-j2 pi ft_0} \Rightarrow\)
  Un ritardo modifica lo spettro di \textbf{fase} ma \emph{non cambia}
  il suo spettro di ampiezza, in quanto quest'ultimo di indica quali
  componenti sinusoidali sono necessarie per comporre la forma del
  segnale, mentre lo spettro di fase mi dice con quale \emph{angolo}
  iniziale devono ``partire'' le sinusoidi.

  Quindi se il segnale si sposta nel tempo, allora le sinusoidi hanno
  angoli iniziali diversi, ma sono le stesse. \begin{gather*}
  |Y(f)| = |X(f)|\cdot |e^{-j2\pi ft_0}| = |X(f)| \\
  \phase{Y(f)} = \phase{X(f) \ e^{-j2 pi ft_0}} = \phase{X(f)} + \phase{e^{-j2 pi ft_0}} = \underbrace{\phase{X(f)} - \overbrace{2\pi ft_0}^{=0}}_{\text{\underline{NON} è una traslazione!}}
  \end{gather*}
\item
  \textbf{Cambiamento di scala}

  Si consideri \(y(t)=x(\alpha t)\), effettuando un \emph{cambiamento
  della scala temporale}: \[
  \begin{array}{cl}
  |\alpha | > 1 \ \to \text{ compressione della scala dei tempi} \to \text{ l'evoluzione è "accelerata"}\\
  |\alpha | < 1 \ \to \text{ dilatazione della scala dei tempi} \to \text{ l'evoluzione è "rallentata"}\\
  \alpha  < 0 \ \to \text{ inversione della scala dei tempi}
  \end{array} 
  \] Inoltre vale: \[
  x(\alpha t) \Longleftrightarrow \frac{1}{|\alpha |} x(\frac{f}{\alpha})
  \]

  \begin{itemize}
  \tightlist
  \item
    Dimostrazione: \[
    \cdot \ \ \underline{\alpha > 0} \Rightarrow x(\alpha t) \Longleftrightarrow \int_{-\infty}^{\infty} x(\alpha t) e^{-j2\pi ft} \,dt \text{, ponendo } z=\alpha t \to t = \frac{z}{\alpha}, \,dz = \alpha \,dt 
    \] \[
    \Rightarrow x(\alpha t) \Longleftrightarrow \int_{-\infty}^{\infty} \frac{x(z) e^{-j2\pi f \frac{z}{\alpha}}}{\alpha} \,dz = \frac{1}{\alpha}\int_{-\infty}^{\infty} x(z) e^{-j2\pi f \frac{z}{\alpha}} \,dz = \frac{1}{\alpha} X(\frac{f}{\alpha})
    \] \[
    \cdot \ \ \underline{\alpha < 0} \Rightarrow x(\alpha t) \Longleftrightarrow \int_{\infty}^{-\infty} \frac{x(z) e^{-j2\pi f \frac{z}{\alpha}}}{\alpha} \,dz = -\frac{1}{\alpha}\int_{-\infty}^{\infty} x(z) e^{-j2\pi f \frac{z}{\alpha}} \,dz = -\frac{1}{\alpha} X(\frac{f}{\alpha})
    \] È da notare come l'inversione dell'integrale nel secondo caso
    l'abbiamo quando \(t \to -\infty, \ z \to +\infty\). Inoltre abbiamo
    sostituito \(z=-\alpha t\).
  \end{itemize}

  Quindi una \emph{dilatazione} nel tempo corrisponde ad una
  \emph{compressione} in frequenza, e \textbf{viceversa}
\item
  \textbf{Modulazione}

  Dato un segnale \(x(t)\) e la sua trasformata \(X(f)\) allora \[
  x(t)\cos(2\pi f_{0}t) \Longleftrightarrow \frac{X(f-f_0)+X(f+f_0)}{2}
  \] dove \(X(f-f_0)\) e \(X(f+f_0)\) sono rispettivamente la replica
  centrata in \(f_0\) e la replica centrata in \(-f_0\).

  \begin{itemize}
  \tightlist
  \item
    Dimostrazione: \begin{gather*}
    \text{TCF}[x(t)\cos(2\pi f_{0}t)] = \int_{-\infty}^{\infty} x(t)\cos(2\pi f_{0}t)  e^{-j2\pi ft} \,dt = 
    \\
    = \frac{1}{2} \int_{-\infty}^{\infty} x(t) [e^{-j2\pi f_0 t}+ e^{-j2\pi f_0 t}]e^{-j2\pi ft} \,dt =\\ \frac{1}{2}\Big[\int_{-\infty}^{\infty}x(t)  e^{-j2\pi (f-f_0)t}\,dt + \int_{-\infty}^{\infty}x(t)  e^{-j2\pi (f+f_0)t}\,dt  \Big] =
    \\
    \frac{X(f-f_0)+X(f+f_0)}{2}
    \end{gather*} Corollario:
    \(x(t)e^{j2\pi f_{0}t} \Longleftrightarrow X(f-f_0)\)
  \end{itemize}
\item
  \textbf{Derivazione}

  Se \(x(t) \to X(f)\), allora: \[
  \dv{}{t} x(t) \Longleftrightarrow j2\pi f \cdot X(f) = Y(f)
  \] Una derivata nel tempo è una \emph{moltiplicazione} in frequenza.

  \begin{itemize}
  \tightlist
  \item
    Dimostrazione:
  \end{itemize}

  Deriviamo entrambi i lati di \(x(t)\): \begin{gather*}
  \dv{}{t}x(t) = \dv{}{t} \int_{-\infty}^{\infty} X(f) e^{j2\pi ft} \,df = \int_{-\infty}^{\infty}\dv{}{t} \Big [X(f) e^{j2\pi ft} \Big ] \,df = \int_{-\infty}^{\infty} X(f) \dv{}{t}e^{j2\pi ft} \,df = 
  \\
  \int_{-\infty}^{\infty} X(f) (2\pi f) e^{j2\pi ft} \,df \Longrightarrow \text{[TCF]}\dv{x(t)}{t} = j2\pi f X(f)
  \end{gather*}

  Il teorema della derivazione \emph{modifica gli spettri} \[
  |Y(f)| = 2\pi f |X(f)| \\ \phase{Y(f)} = \phase{X(f)} + \text{sgn}(f)\frac{\pi}{2}
  \] Aumenta proporzionalmente l'ampiezza, esaltando le altre frequenze,
  e sfasando di \(\pm \frac{\pi}{2}\) \newpage
\item
  \textbf{Integrazione} (deriva dal teorema di derivazione)

  Dato un segnale \(x(t) \Longleftrightarrow X(f)\) e un segnale
  \(y(t) = \int_{-\infty}^{t} x(\alpha) \,d\alpha\), allora vale \[
  \int_{-\infty}^{t}  x(\alpha) \,d\alpha \Longleftrightarrow \frac{X(f)}{j2\pi f}
  \]

  \begin{itemize}
  \tightlist
  \item
    Dimostrazione:
  \end{itemize}

  Segue dal teorema di derivazione e richiede che \(X(0)=0\), al fine di
  evitare che per \(f\to 0\), il rapporto tenda ad infinito.
  \begin{gather*}
  X(0)= 0 \longleftrightarrow X(0)=\underbrace{\int_{-\infty}^{\infty}x(t) \ e^{0}\,dt}_{\text{sottende area} \textbf{ nulla}} \longleftrightarrow y(\infty) = \int_{-\infty}^{\infty} x(t)\,dt = X(0) \to 0 \\ y(t) = \int_{-\infty}^{t} x(\alpha) \,d\alpha \Rightarrow x(t) \dv{}{t}y(t) \Rightarrow X(f)=j2\pi f\cdot Y(f) \Rightarrow Y(f) = \frac{X(f)}{j2\pi f}
  \end{gather*} Anche l'integrale nel tempo si trasforma in
  un'operazione algebrica in frequenza: in questo caso però vengono
  esaltate le componenti a \textbf{bassa} frequenza nello spettro del
  segnale, mentre le alte vengono attenuate; la fase varia sempre di
  \(\pm \frac{\pi}{2}\) \[
  |Y(f)| = \frac{|X(f)|}{2\pi f} \\ \phase{Y(f)} = \phase{X(f)} + \text{sgn}(f)\frac{\pi}{2}
  \] Da questo teorema deriva la relazione
  \(A \text{tri}(\frac{t}{T})\Longleftrightarrow AT\mathop{\mathrm{sinc}}^{2}(fT); \ A\mathop{\mathrm{rect}}(\frac{t}{T})\Longleftrightarrow AT\mathop{\mathrm{sinc}}(fT)\)
\item
  \textbf{Prodotto}: è il duale della convoluzione

  Partendo da due segnali \(x(t)\) e \(y(t)\) \[
  z(t)=x(t)\cdot y(t) \Longleftrightarrow X(f) \otimes Y(f)
  \]

  \begin{itemize}
  \tightlist
  \item
    Dimostrazione: \begin{gather*}
    \Rightarrow Z(f) =  \int_{-\infty}^{\infty}x(t)\ y(t)\ e^{-j2\pi ft}\,dt = \int_{t=-\infty}^{\infty} \Big[ \int_{\nu = -\infty}^{\infty} X(\nu) e^{-j2\pi \nu t} \,d\nu \Big ] y(t)\ e^{-j2\pi ft}\,dt= \\ \int_{\nu=-\infty}^{\infty} X(\nu) \Big[ \int_{t = -\infty}^{\infty}  y(t)\ e^{-j2\pi (f-\nu)t}\,dt \Big ] \,d\nu =  \int_{\nu = -\infty}^{\infty} X(\nu) Y(t-\nu) \,d\nu = \\ X(f) \otimes Y(f)
    \end{gather*} Quindi: \[
    \underset{PRODOTTO}{x(t) \ y(t)} \Longleftrightarrow \underset{CONVOLUZIONE}{X(f)\otimes Y(f)} \to \text{ la convoluzione è \textit{commutativa}}
    \]
  \end{itemize}
\item
  \textbf{\emph{Convoluzione}}

  Dati due segnali \(x(t)\) e \(y(t)\) sappiamo che: \[
  z(t) = x(t) \otimes y(t) \Longleftrightarrow X(f) \ Y(f)
  \]

  \begin{itemize}
  \item
    Dimostrazione:

    Partiamo sempre dalla definizione di TCF: \begin{gather*}
    z(t) = x(t) \otimes y(t) = \int_{-\infty}^{\infty} x(\alpha) y(t-\alpha) \,d\alpha \Longleftrightarrow Z(f) = \int_{-\infty}^{\infty} z(t) \ e^{-j2\pi ft} \,dt = \\  \int_{-\infty}^{\infty} \int_{-\infty}^{\infty} \Big [ x(\alpha) y(t -\alpha) \,d\alpha  \Big] e^{-j2\pi f(t-\alpha+\alpha)} \,dt =\\
    \int_{\alpha=-\infty}^{\infty} x(\alpha) \underbrace{\Big [\int_{t=-\infty}^{\infty}y(t-\alpha)e^{-j2\pi f(t-\alpha)}\,dt \Big ]}_{Y(f)}  e^{-j2\pi f\alpha} \,d\alpha = \\ \int_{\alpha=-\infty}^{\infty} x(\alpha) \ Y(f) \ e^{-j2\pi f\alpha}\,d\alpha = X(f) \ Y(f)
    \end{gather*}
  \item
    Nota bene:

    \begin{itemize}
    \tightlist
    \item
      la convoluzione ha proprietà commutativa, associativa e
      distributiva.
    \end{itemize}
  \end{itemize}
\end{enumerate}

\subsection{Trasformata di Fourier
generalizzata}\label{trasformata-di-fourier-generalizzata}

\begin{enumerate}
\def\labelenumi{\arabic{enumi}.}
\setcounter{enumi}{21}
\item
  Teorema d'integrazione \textbf{completo}:

  Vogliamo rimuovere il vincolo (o ipotesi) \(X(0)\) che è alla base
  dell'applicabilità del teorema d'integrazione ``incompleto'': ciò
  viene realizzato utilizzando la delta di Dirac.

  Il teorema completo afferma che: \[
  y(t) = \int_{-\infty}^{t} x(\alpha)\,d\alpha \Longleftrightarrow \frac{X(f)}{j2\pi f} + \frac{\delta(f)}{2}\cdot X(0)  
  \] Il nuovo termine rende conto dell'eventuale valor medio diverso da
  zero del segnale!.

  \begin{itemize}
  \item
    Dimostrazione:

    Essendo: \[
      x(t)\otimes u(t) = \int_{-\infty}^{\infty} x(\alpha) \ u(t-\alpha)\,d\alpha = \int_{-\infty}^{t} x(\alpha)\,d\alpha
      \] abbiamo che per la convoluzione
    \(x(t)\otimes u(t)\Longleftrightarrow X(f)U(f)\): \[
      X(f)\ U(f)=X(f)\Big[\frac{1}{j2\pi f}+\frac{\delta(f)}{2}\Big] = \frac{X(f)}{j2\pi f}+ \frac{X(0)}{2}\delta(f)
      \] Questo perché \(\text{TCF}(u(t))=U(f)=\frac{1}{j2\pi f}\);
    l'ultimo termine scompare per segnali ad area nulla: rende conto
    dell'eventuale valor medio diverso da zero del segnale, ed è un
    termine correttivo che rappresenta la funzione impulsiva.
  \end{itemize}
\item
  Teorema della modulazione, alternativa:

  \begin{itemize}
  \item
    Dimostrazione:

    per il teorema del prodotto, \begin{gather*}
      x(t)\cos(2\pi f_{0}t) \Longleftrightarrow X(f) \otimes \Big[ \frac{\delta(f-f_0)+\delta(f+f_0)}{2} \Big] = \\ X(f) \otimes \frac{\delta(f-f_0)}{2} + X(f) \otimes  \frac{\delta(f+f_0)}{2} \\
      \to X(f)\otimes \delta(f-f_0) = \int_{\mathbb{R}}X(\alpha) \delta(f-f_0 -\alpha)\,d\alpha = \int_{\mathbb{R}}X(\alpha) \delta(\alpha) -(f-f_0)\,d\alpha = X(f-f_0) \\
      x(t)\cos(2\pi f_0 t) \Longleftrightarrow \frac{X(f-f_0)+X(f+f_0)}{2}
      \end{gather*}
  \end{itemize}
\end{enumerate}

\subsection{Periodicizzazione}\label{periodicizzazione}

\begin{enumerate}
\def\labelenumi{\arabic{enumi}.}
\setcounter{enumi}{23}
\item
  Prima formula della somma di Poisson:

  Come rendere un segnale \emph{aperiodico} \(x(t)\) \textbf{periodico}
  di periodo \(T_0\). Partiamo da
  \(y(t)=\sum_{n=-\infty}^{\infty} = x(t-nT_0)\) relazione nel tempo tra
  periodico e aperiodico \begin{gather*}
  \to Y_k = \frac{1}{T_0} = \int_{-\frac{T}{2}}^{\frac{T}{2}} y(t) \ e^{-j2 \pi kf_{0}t} \,dt = \frac{1}{T_0} = \int_{-\frac{T}{2}}^{\frac{T}{2}} \sum_{n=-\infty}^{\infty} x(t-nT_0) e^{-j2\pi kf_{0}t} \,dt \\
  \sum_{n=-\infty}^{\infty}\frac{1}{T_0} \int_{-\frac{T}{2}}^{\frac{T}{2}} x(t-nT_0)e^{-j2\pi kf_{0}t} \,dt = \ \ \ \text{ sostituiamo } \left\{ \begin{array}{cl} \alpha = t-t_0 \\ t = \alpha -t_0 \\ \,d\alpha = \,dt \end{array}\right.  \\
  \frac{1}{T_0} \sum_{n=-\infty}^{\infty}\frac{1}{T_0} \int_{-\frac{T}{2}-nT_0}^{\frac{T}{2}-nT_0} x(\alpha) \ e^{-j2\pi kf_{0}(\alpha +nT_0)} \,d\alpha = \\
  \frac{1}{T_0} \sum_{n=-\infty}^{\infty}\frac{1}{T_0} \int_{-\frac{T}{2}-nT_0}^{\frac{T}{2}-nT_0} x(\alpha) e^{-j2\pi kf_{0}\alpha} \cdot \underbrace{e^{-j2\pi k\cancel{f_0}n\cancel{T_0}}}_{\text{multiplo di }2\pi \to e^{0}} \,d\alpha= \\
  \frac{1}{T_0} \sum_{n=-\infty}^{\infty}\frac{1}{T_0} \int_{-\frac{T}{2}-nT_0}^{\frac{T}{2}-nT_0} x(\alpha) \ e^{-j2\pi kf_{0}\alpha} \,d\alpha =
  \frac{1}{T_0} \int_{-\infty}^{\infty}x(\alpha) \ e^{-j2\pi kf_{0}\alpha}\,d\alpha = \underbrace{\frac{1}{T_0}X(kf_0)}_{\text{campionamento in frequenza}}
  \end{gather*} Si ottiene una relazione detta \textbf{campionamento in
  frequenza}. I coefficienti della serie di Fourier del segnale
  periodico \(y(t)\) sono, a meno del fattore \(\frac{1}{T_0}\), i
  campioni della TCF del \emph{segnale base} \(x(t)\) presi in
  corrispondenza delle frequenze armoniche \(kf_0\) \begin{gather*}
  \to \sum_{n=-\infty}^{\infty}x(t-nT_0) = \sum_{k=-\infty}^{\infty} \frac{1}{T_0} X(\frac{k}{T_0})\ e^{+j2\pi kt f_0}
  \end{gather*}
\item
  Seconda formula della somma di Poisson

  Applichiamo alla prima formula di Poisson il teorema della dualità: \[
  X(t) \longleftrightarrow x(-f) \\
  x(t) \longleftrightarrow X(f)
  \] \begin{gather*}
  \sum_{n=-\infty}^{\infty}x(t-nT_0) = \sum_{k=-\infty}^{\infty} \frac{1}{T_0} X(\frac{k}{T_0})\ e^{+\frac{j2\pi kt}{T_0}} \\
  \sum_{n=-\infty}^{\infty}x(t-nT_0) = \sum_{k=-\infty}^{\infty} \frac{1}{T_0}(X(-\frac{k}{T_0}))\ e^{+\frac{j2\pi kt}{T_0}} \\
  \Rightarrow \sum_{n=-\infty}^{\infty}x(t-nT_0) = \sum_{k=-\infty}^{\infty} \frac{1}{T_0} X(\frac{k}{T_0})\ e^{-\frac{j2\pi kt}{T_0}} \text{ cambio di segno all'indice k} \\
  \to T = \frac{1}{T_0} \Rightarrow \sum_{n=-\infty}^{\infty}X(t-\frac{n}{T}) = T\sum_{k=-\infty}^{\infty} \frac{1}{T_0} x(kT)\ e^{-j2\pi ktfT}
  \end{gather*} Adesso, dal punto di vista puramente formale, cambiano
  nome da \(t\) in \(f\), otteniamo un'espressione, otteniamo
  un'espressione \emph{duale} rispetto alla prima formula di Poisson \[
  \sum_{n=-\infty}^{\infty}x(nT)e^{-j2\pi fT} = \frac{1}{T}\sum_{k=-\infty}^{\infty}X(f-\frac{k}{T})
  \]
\end{enumerate}

\subsection{Sistemi}\label{sistemi}

\begin{enumerate}
\def\labelenumi{\arabic{enumi}.}
\setcounter{enumi}{25}
\tightlist
\item
  Teorema di Parseval:
\end{enumerate}

Dato un segnale \(x(t)\) e la sua energia
\(E_{x}=\int_{-\infty}^{\infty} |x(t)|^2 \,dt < +\infty\) (energia
finita), possiamo esprimere l'energia \(E_x\) \emph{anche in frequenza}:

\begin{gather*}
E_{x}=\int_{-\infty}^{\infty} |x(t)|^2 \,dt = \int_{-\infty}^{\infty} x(t) \ x^{\ast} \,dt = \int_{-\infty}^{\infty} x(t) \Big[ \int_{-\infty}^{\infty} X^{*}(f) e^{-j2\pi ft} \,df \Big] \,dt \\
\int_{f=-\infty}^{\infty} X^{\star}(f) \Big [\int_{t=-\infty}^{\infty} x(t) e^{-j2\pi ft} \,dt\Big ] \,df = \int_{-\infty}^{\infty} X^{*}(f) = \int_{-\infty}^{\infty} |X(f)|^{2} \,df
\end{gather*}

\(E_x\) è l'energia totale, deriva da \(p_x = |x(t)|^2\) potenza
istantanea integrata o da \(|X(f)|^2\) detta \textbf{densità spettrale}
\(E_x(f)\) integrata.

\begin{enumerate}
\def\labelenumi{\arabic{enumi}.}
\setcounter{enumi}{26}
\tightlist
\item
  Teorema di Wiener-Khinchin
\end{enumerate}

Siamo la densità spettrale di potenza: \[
P_x = \lim_{T\to \infty} \frac{1}{T}\int_{-\infty}^{\infty} x(t) \,dt 
\] e la funzione \emph{densità spettrale di potenza} \[
S_x(f) \triangleq \lim_{T\to \infty} \frac{E_{x_T}(f)}{T} =\lim_{T\to\infty} \frac{|x(t)|^2}{T} 
\] con \(E_{x_T}(f)\) densità di energia del segnale \emph{troncato}
nell'intervallo \([-\frac{T}{2}; \frac{T}{2}]\)

Definiamo \textbf{funzione di autocorrelazione}
\(R_x(\tau)= \int_{-\infty}^{\infty}x(\tau)x(t-\tau)\,dt\) ossia il
segnale moltiplicato per una sua replica \emph{ritardata}. Indica
``quanto il segnale somiglia alla sua replica ritardata'': più \(x(t)\)
è compatta meno somiglierà e meno varrà \(R_x(\tau)\)

Il teorema afferma che la densità spettrale di energia di un segnale
coincide con la trasformata di Fourier della funzione di
autocorrelazione del segnale stesso:

\begin{gather*}
E_x(f)= \int_{-\infty}^{\infty}R_{x}(\tau) e^{-j2\pi ft}\,d\tau \underbrace{=}_{R_x(\tau) \text{ è pari})} 2\int_{0}^{\infty} \cos(2\pi f\tau)R_x(\tau) \,d\tau
\end{gather*}

\begin{itemize}
\tightlist
\item
  Dimostrazione:
\end{itemize}

Partiamo dalla definizione di autocorrelazione:

\begin{gather*}
R_{x} (\tau) = \int_{-\infty}^{\infty} x(\alpha) x(\alpha -t) \,d\alpha = \int_{-\infty}^{\infty} x(\alpha)x(-(t-\alpha)) \,d\alpha = x(\tau) \otimes x(-\tau) = \\
R_{x} (\tau) =  x(\tau) \otimes x(-\tau) \Longleftrightarrow X(f) \ X(-f) = X(f) \ X^{*}(-f) = |X(f)|^{2} = E_x(f)
\end{gather*}

\end{document}
