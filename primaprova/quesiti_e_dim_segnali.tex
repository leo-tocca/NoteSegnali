% Options for packages loaded elsewhere
\PassOptionsToPackage{unicode}{hyperref}
\PassOptionsToPackage{hyphens}{url}
%
\documentclass[
]{article}
\usepackage{lmodern}
\usepackage[a4paper, total={6in, 10in}]{geometry}
\usepackage{amssymb,amsmath}
\usepackage{ifxetex,ifluatex}
\ifnum 0\ifxetex 1\fi\ifluatex 1\fi=0 % if pdftex
  \usepackage[T1]{fontenc}
  \usepackage[utf8]{inputenc}
  \usepackage{textcomp} % provide euro and other symbols
\else % if luatex or xetex
  \usepackage{unicode-math}
  \defaultfontfeatures{Scale=MatchLowercase}
  \defaultfontfeatures[\rmfamily]{Ligatures=TeX,Scale=1}
\fi
% Use upquote if available, for straight quotes in verbatim environments
\IfFileExists{upquote.sty}{\usepackage{upquote}}{}
\IfFileExists{microtype.sty}{% use microtype if available
  \usepackage[]{microtype}
  \UseMicrotypeSet[protrusion]{basicmath} % disable protrusion for tt fonts
}{}
\makeatletter
\@ifundefined{KOMAClassName}{% if non-KOMA class
  \IfFileExists{parskip.sty}{%
    \usepackage{parskip}
  }{% else
    \setlength{\parindent}{0pt}
    \setlength{\parskip}{6pt plus 2pt minus 1pt}}
}{% if KOMA class
  \KOMAoptions{parskip=half}}
\makeatother
\usepackage{xcolor}
\IfFileExists{xurl.sty}{\usepackage{xurl}}{} % add URL line breaks if available
\IfFileExists{bookmark.sty}{\usepackage{bookmark}}{\usepackage{hyperref}}
\hypersetup{
  hidelinks,
  pdfcreator={LaTeX via pandoc}}
\urlstyle{same} % disable monospaced font for URLs
\setlength{\emergencystretch}{3em} % prevent overfull lines
\providecommand{\tightlist}{%
  \setlength{\itemsep}{0pt}\setlength{\parskip}{0pt}}
\setcounter{secnumdepth}{-\maxdimen} % remove section numbering

\hypersetup{
	pdftitle={Dimostrazioni e quesiti per primo parziale di Teoria dei Segnali},
	pdfauthor={Leonardo Toccafondi},
	hidelinks,
	pdfcreator={LaTeX via pandoc}}

\title{Dimostrazioni e quesiti per primo parziale di Teoria dei Segnali}
\author{}
\date{}

\begin{document}
	\maketitle

\hypertarget{quesiti}{%
\subsection{QUESITI}\label{quesiti}}

\begin{enumerate}
\def\labelenumi{\arabic{enumi}.}

\item
	Un segnale deterministico a tempo continuo s(t) periodico ha Energia e Potenza finite e/o infinite ?
	
\item
Si definisca la Potenza istantanea di un segnale deterministico a
tempo continuo $s(t)$

\item 
Si definisca la potenza media di un segnale deterministico a tempo continuo $s(t)$

\item
Calcolare Energia e Potenza media del segnale a tempo continuo
\(x(t) = Ae ^{-t} u(t)\).

\item
Calcolare l'energia del segnale x(t) = sinc(t).

\item
Calcolare Energia e Potenza media di un segnale a tempo continuo
sinusoidale di ampiezza 2 e di periodo 2 secondi.
\item
Calcolare Energia e Potenza media di un segnale a tempo continuo
sinusoidale di ampiezza A e di periodo T.

\item
Calcolare Energia e Potenza media di un segnale a tempo continuo
costante

\item 
Sotto quali condizioni un segnale deterministico a tempo continuo s(t) ha Potenza media finita?

\item
Giustificare la seguente affermazione: ``Lo spettro di ampiezza di un
segnale tempo-continuo aperiodico reale è una funzione pari''.

\item
Giustificare la seguente affermazione: ``Lo spettro di fase di un
	segnale tempo-continuo aperiodico reale è una funzione dispari''.


\item
Calcolare la quantità
\(\displaystyle \int_{- \infty}^{\infty} sinc(t) dt\)
	
\item 
Tra onda quadra e onda triangolare, quale dei due segnali ha coefficienti della serie di Fourier con modulo che
va a zero più velocemente al crescere di n?

\item
Tra la rampa e l'onda triangolare, quale dei due segnali ha
coefficienti della serie di Fourier con modulo che va a zero più
velocemente al crescere di n, e perché?

\item
Quali sono le proprietà della serie di Fourier di un segnale periodico
pari

\item
Quali sono le proprietà della serie di Fourier di un segnale periodico
dispari

\item 
Se il segnale periodico $x(t)$ è reale e dispari, allora la serie di Fourier si semplifica in modo tale che ...


\item 
Se il segnale periodico $x(t)$ è reale e pari, allora la serie di Fourier si semplifica in modo tale che ...


\item
Elencare le Condizioni di Dirichlet per la convergenza della serie di
Fourier

\item 
Elencare le Condizioni di Dirichlet per la convergenza della trasformata di Fourier



\item 
Se il segnale aperiodico $x(t)$ è pari, allora la sua trasformata di Fourier è	

\item 
Se il segnale aperiodico $x(t)$ è dispari, allora la sua trasformata di Fourier è

\item 
Se il segnale aperiodico $x(t)$ è reale e pari, allora la sua trasformata di Fourier è

\item
Se il segnale aperiodico $x(t)$ è reale e dispari, allora la sua
trasformata di Fourier è\ldots{}

\item 
Se la trasformata di Fourier di $x(t)$ è $X(f)$, allora la trasformata di $x(at)$ con $|a|>1$ risulta modificata in modo
che ...

\item 
Se la trasformata di Fourier di $x(t)$ è $X(f)$, allora la trasformata di $x(at)$ con $|a|<1$ risulta modificata in modo che ...



\item
  In cosa differiscono le trasformate di Fourier di
  \(rect (\frac{t}{T})\) e di \(rect (\frac{t-5}{T})\)? Spiegare le
  differenze sia per lo spettro di ampiezza che per lo spettro di fase.


\item
  Se \(x(t)\) è un segnale complesso e \(X(f)\) la sua trasformata di
  Fourier, quale è la trasformata di Fourier della parte reale di
  \(x(t\))? Giustificare la risposta
  
  \item
  Se \(x(t)\) è un segnale complesso e \(X(f)\) la sua trasformata di
  Fourier, quale è la trasformata di Fourier della parte immaginaria di
  \(x(t\))? Giustificare la risposta
\item
  Quale è la trasformata di Fourier di $ e ^{-j2 \pi t}$?
  Giustificare la risposta
\item
  Se \(x(t) \leftrightarrow X(f)\) sono una funzione e la sua
  trasformata di Fourier, qual è la trasformata di Fourier di \(x(-t)\)?
  Giustificare la risposta



\item
  In cosa differiscono le trasformate di Fourier di $rect(\frac{t}{T})$ e
  di $rect(\frac{t}{2T})$ ? Spiegare le differenze sia per lo spettro di
  ampiezza che per lo spettro di fase.


\item
  Se il segnale periodico x(t) è reale, quali proprietà hanno
  rispettivamente lo spettro di ampiezza e lo spettro di fase?


\item
Un sistema a tempo continuo si definisce causale se\ldots{} Fare poi
un esempio di sistema causale e uno di sistema non causale.

\item
Un sistema a tempo continuo si definisce stabile se

\item
Dato un sistema LTI con risposta impulsiva $h(t) = \delta (t - 1) +\delta (t - 2)$,
disegnare l'uscita quando il suo ingresso è il segnale $x(t) = u(t)$  
\item 
La risposta impulsiva di un sistema LTI causale è
\item
La risposta impulsiva di un sistema LTI causale soddisfa la seguente
condizione:
\item
Un sistema che produce in uscita il valore assoluto del segnale al suo
ingresso è un sistema lineare? Si argomenti la risposta con almeno un
esempio.
\item
Se un sistema LTI tempo-continuo ha una risposta impulsiva con energia
finita, possiamo affermare che il sistema è stabile? Giustificare la
risposta.

\item
Quale è la risposta in frequenza di un sistema LTI retto
dall'equazione differenziale $y(t) - \frac{d^{2}y(t)}{dt^2} = x(t)$?
\item
Definire la densità spettrale di potenza di un segnale \(x(t)\) a
potenza finita.
\item
Un processo aleatorio si definisce stazionario in senso lato se
\item
Elencare le proprietà che definiscono un processo aleatorio
stazionario in senso lato (WSS).
  
\end{enumerate}

\hypertarget{dimostrazioni}{%
\subsection{DIMOSTRAZIONI}\label{dimostrazioni}}

\begin{enumerate}
\def\labelenumi{\arabic{enumi}.}

\item
Dimostrare che se un segnale periodico è reale, allora i coefficienti
della sua espansione in serie di Fourier (nella sua forma complessa)
sono caratterizzati da una simmetria Hermitiana.
\item
  Enunciare e dimostrare il Teorema di dualità della Trasformata
  continua di Fourier
\item
  Enunciare e dimostrare il Teorema del cambiamento di scala della
  Trasformata continua di Fourier
\item
  Enunciare e dimostrare il Teorema di integrazione completo della
  Trasformata continua di Fourier


\item
  Dato un segnale x(t), enunciare e dimostrare la formula somma di
  Poisson che lega i coeffcienti della serie di Fourier di \(\displaystyle y(t) = \sum_{n = -\infty}^{\infty} x(t - nT)\)
  alla trasformata (aperiodica) di Fourier di x(t).
\item
  Dimostrare che la trasformata di Fourier del prodotto di funzioni
  \(x(t) \cdot y(t)\) è la convoluzione di \(X(f)\) con \(Y (f)\).

\item
Enunciare e dimostrare il Teorema di Parseval
\item
  Enunciare il Teorema di Wiener-Khintchine
\end{enumerate}

\newpage

\hypertarget{risposte-quesiti}{%
	\subsection{RISPOSTE QUESITI}\label{risposte-quesiti}}

\begin{enumerate}
	\def\labelenumi{\arabic{enumi}.} 
	\item
	Un segnale deterministico a tempo continuo \(s(t)\) periodico ha
	potenza finita (somma di infinite aree) e energia infinita(pari alla potenza media calcolata in un singolo periodo).
	\item
	La potenza istantanea di un segnale deterministico a tempo continuo
	\(s(t)\) è definita come \(P(t) = |s(t)|^2\).
	\item
	La potenza media di un segnale deterministico a tempo continuo s(t) è
	definita come $ \displaystyle \lim_{s\to\infty} \frac{1}{T} \int_{-\frac{T}{2}}^{\frac{T}{2}} s^2(t)dt$, dove T è il 	periodo del segnale.
	\item
	Per il segnale $x(t) = Ae^{-t}u(t),$ l'energia è $ E = \int_0^\infty |Ae^{t}|^2 dt = A^2 \int_0^\infty e^{-
		2t} dt = \frac{A^2}{2}$ e la potenza media è infinita poiché il segnale non è periodico.
	\item
	L'energia del segnale \(x(t) = sinc(t)\) è pari ad uno, in quanto utilizzando il teorema di Parseval,
	$\int_{-\infty}^{+\infty}|x(t)|^2dt = \int_{-\infty}^{+\infty}|X(f)|^2df$, e la trasformata 
	di $sinc(t)$ è pari a $rect(f)$, pari ad un rettangolo di altezza e base 1.
	\(E = \int_{-\infty}^\infty (\frac{sin(\pi t)}{( \pi t)})^2 dt = 1.\)
	\item
	Per un segnale sinusoidale di ampiezza 2 e periodo 2 secondi,
	l'energia è infinita perché il segnale è periodico e la potenza media
	è \(P_m = \frac{1}{2} * (2^2) = 2.\)
	\item
	Per un segnale sinusoidale di ampiezza A e periodo T, l'energia è
	infinita perché il segnale è periodico e la potenza media è
	\(P_m = \frac{1}{2} * A^2\).
	\item
	Per un segnale costante, l'energia è infinita perché il segnale è
	periodico e la potenza media è \(P_m = C^2,\) dove C è il valore
	costante.
	\item
	Un segnale deterministico a tempo continuo \(s(t)\) ha potenza media
	finita se: \newline $ \displaystyle \lim_{s\to\infty} \frac{1}{T} \int_{-\frac{T}{2}}^{\frac{T}{2}} s^2(t)dt < \infty$.
	\item
	Lo spettro di ampiezza di un segnale tempo-continuo aperiodico reale è
	una funzione pari perché la trasformata di Fourier di un segnale reale
	ha componenti simmetriche rispetto all'asse delle ordinate.
	\item
	Lo spettro di fase di un segnale tempo-continuo aperiodico reale è una
	funzione dispari perché la fase della trasformata di Fourier di un
	segnale reale è uguale all'opposto della fase della sua componente
	coniugata.
	\item
	$ \displaystyle \int_{-\infty}^\infty sinc(t)dt = 1$, in quanto integrale di una
	funzione sinc normalizzata.
	\item
	L'onda triangolare, in quanto non presentando discontinuità, a differenza del segnale onda quadra), nella sua ricostruzione del segnale le componenti ad alta frequenza hanno importanza minore.
	\item
L'onda triangolare, in quanto non presentando discontinuità, a differenza del segnale dente di sega), nella sua ricostruzione del segnale le componenti ad alta frequenza hanno importanza minore.
	\item
	Le proprietà della serie di Fourier di un segnale periodico pari sono:
	solo coefficienti \(a_0\) e \(a_n\) diversi da zero, e tutti i
	coefficienti \(b_n\) sono nulli.
	\item
	Le proprietà della serie di Fourier di un segnale periodico dispari
	sono: tutti i coefficienti \(a_0\) e \(a_n\) sono nulli, e solo i
	coefficienti \(b_n\) sono diversi da zero.
	\item
	Se il segnale periodico \(x(t)\) è reale e dispari, la serie di
	Fourier si semplifica in modo tale che tutti i coefficienti \(a_n\)
	sono nulli e la serie contiene solo termini con seno.
	\item
	Se il segnale periodico \(x(t)\) è reale e pari, la serie di Fourier
	si semplifica in modo tale che tutti i coefficienti \(b_n\) sono nulli
	e la serie contiene solo termini con coseno.
	\item
	Le Condizioni di Dirichlet per la convergenza della serie di Fourier
	sono:
	
	\begin{enumerate}
		\def\labelenumii{(\alph{enumii})}
		\tightlist
		\item
		la funzione deve essere assolutamente integrabile sul periodo $T_0$: $\displaystyle \int_{-\frac{T_0}{2}}^{\frac{T_0}{2}} |x(t)| dt < +\infty$
		\item
		la funzione deve essere continua o presentare un numero finito di discontinuità di prima specie
		\item
		la funzione deve avere un numero finito di massimi e minimi
		all'interno di un periodo. Oppure  $x(t)$ derivabile rispetto al tempo nel periodo $T_0$, esclusi al più un numero finito di disconuità di prima specie.
	\end{enumerate}
	\item
	Le Condizioni di Dirichlet per la convergenza della trasformata di
	Fourier sono simili a quelle per la serie di Fourier:
	
	\begin{enumerate}
		\def\labelenumii{(\alph{enumii})}
		\tightlist
		\item
		la funzione deve essere assolutamente sommabile: $\displaystyle \int_{-\infty}^{\infty} |x(t)| dt < +\infty$
		\item
		se in qualunque intervallo finito $(t_1, t_2)$ è continua o presenta un numero finito di discontinuità di prima specie
		\item
		 se in qualunque intervallo finito $(t_1, t_2)$ la funzione ha	 un numero finito di massimi e minimi.
	\end{enumerate}
	\item
	La trasformata è pari a sua volta.
	\item
	La trasformata è dispari a sua volta.
	\item
	Se il segnale aperiodico \(x(t)\) è reale e pari, la sua trasformata
	di Fourier è reale e pari.
	\item
	Se il segnale aperiodico \(x(t)\) è reale e dispari, la sua
	trasformata di Fourier è immaginaria pura e dispari.
	\item
Se la trasformata di Fourier di $x(t)$ è $X(f)$, allora la trasformata di $x(at)$ con$ |a| > 1$ è$ \frac{1}{a} X(-\frac{f}{a})$. Quindi per il teorema del cambiamento di scala abbiamo una dilatazione in
	\item
Se la trasformata di Fourier di $x(t$) è $X(f)$, allora la trasformata di $x(at)$ con $|a| < 1 $è $\frac{1}{|a|} X(\frac{f}{|a|})$.
	\item
	Le trasformate di Fourier di \(rect (\frac{t}{T})\) e di
	\(rect (\frac{t-5}{T})\) differiscono solo per la fase. Lo spettro di
	ampiezza è lo stesso, mentre lo spettro di fase di
	\(rect (\frac{t-5}{T})\) ha una componente lineare aggiuntiva rispetto
	a \(rect (\frac{t}{T})\).
	\item
	Se \(x(t)\) è un segnale complesso e\$ X(f)\$ la sua trasformata di
	Fourier, la trasformata di Fourier della parte reale di \(x(t)\) è
	\(\frac{X(f) + X^*(-f)}{2}\).
	\item
	Se \(x(t)\) è un segnale complesso e \$X(f) \$la sua trasformata di
	Fourier, la trasformata di Fourier della parte immaginaria di \(x(t)\)
	è \(\frac{-jX(f) + jX^*(-f)}{2}\).
	\item
	La trasformata di Fourier di \(e^{(-j2\pi t)}\) è un impulso di Dirac
	centrato in \(f = 1\).
	\item
	\(x(t) \leftrightarrow X(f)\) sono una funzione e la sua trasformata
	di Fourier, la trasformata di Fourier di \(x(-t)\) è \(X^*(-f)\).
	\item
	Le trasformate di Fourier di \(rect(\frac{t}{T})\) e di
	\(rect(\frac{t}{2T})\) differiscono sia per lo spettro di ampiezza che
	per lo spettro di fase. Lo spettro di ampiezza di
	\(rect(\frac{t}{2T})\)) ha larghezza doppia rispetto a
	\(rect(\frac{t}{T})\), mentre la fase rimane la stessa.
	\item
	Se il segnale periodico \(x(t)\) è reale, lo spettro di ampiezza è
	pari e lo spettro di fase è dispari.
	\item
	Un sistema a tempo continuo si definisce causale se la sua risposta
	impulsiva \(h(t)\) è nulla per \(t < 0\). Un esempio di sistema
	causale è un filtro passa-basso RC, mentre un esempio di sistema non
	causale è un filtro ideale passa-basso.
	\item
	Un sistema a tempo continuo si definisce stabile se la sua risposta
	impulsiva è assolutamente integrabile, ovvero se
	\(\int_{-\infty}^\infty |h(t) dt < \infty\).
	\item
	Dato un sistema LTI con risposta impulsiva
	\(h(t) = \delta(t - 1) + \delta(t - 2)\), l'uscita quando il suo
	ingresso è il segnale \(x(t) = u(t)\) sarà
	\(y(t) = u(t - 1) + u(t - 2).\)
	\item
	La risposta impulsiva di un sistema LTI causale è una funzione
	\(h(t)\) tale che \(h(t) = 0\) per \(t < 0\).
	\item
	La risposta impulsiva di un sistema LTI causale soddisfa la condizione
	\(h(t) = 0\) per \(t < 0\).
	\item
	Un sistema che produce in uscita il valore assoluto del segnale al suo
	ingresso non è un sistema lineare. Ad esempio, considerando i segnali
	\(x_1(t) = t\) e \(x_2(t) = -t\), il sistema produce \(y_1(t) = |t|\)
	e \(y_2(t) = |-t| = t\). Tuttavia, \(y_1(t) + y_2(t) = 2t\), mentre il
	sistema applicato alla somma dei segnali in ingresso produce
	\(|t - t| = 0\), che non è uguale a \(2t\).
	\item
	Se un sistema LTI tempo-continuo ha una risposta impulsiva con energia
	finita, possiamo affermare che il sistema è stabile, perché l'energia
	finita implica che la risposta impulsiva è assolutamente integrabile.
	\item
	La risposta in frequenza di un sistema LTI retto dall'equazione
	differenziale \(y(t) - \frac{d^{2}y(t)}{dt^2} = x(t)\) è
	\(H(f) = 1 + (2 \pi f)^2\).
	\item
	La densità spettrale di potenza di un segnale \(x(t)\) a potenza
	finita è definita come la trasformata di Fourier dell'autocorrelazione
	del segnale: \(S_{xx}(f) = F[R_{xx}(\tau)]\), dove \(R_{xx}(\tau)\) è
	l'autocorrelazione del segnale \(x(t)\).
\end{enumerate}
\end{document}
