%%
% Copyright (c) 2017 - 2023, Pascal Wagler;
% Copyright (c) 2014 - 2023, John MacFarlane
%
% All rights reserved.
%
% Redistribution and use in source and binary forms, with or without
% modification, are permitted provided that the following conditions
% are met:
%
% - Redistributions of source code must retain the above copyright
% notice, this list of conditions and the following disclaimer.
%
% - Redistributions in binary form must reproduce the above copyright
% notice, this list of conditions and the following disclaimer in the
% documentation and/or other materials provided with the distribution.
%
% - Neither the name of John MacFarlane nor the names of other
% contributors may be used to endorse or promote products derived
% from this software without specific prior written permission.
%
% THIS SOFTWARE IS PROVIDED BY THE COPYRIGHT HOLDERS AND CONTRIBUTORS
% "AS IS" AND ANY EXPRESS OR IMPLIED WARRANTIES, INCLUDING, BUT NOT
% LIMITED TO, THE IMPLIED WARRANTIES OF MERCHANTABILITY AND FITNESS
% FOR A PARTICULAR PURPOSE ARE DISCLAIMED. IN NO EVENT SHALL THE
% COPYRIGHT OWNER OR CONTRIBUTORS BE LIABLE FOR ANY DIRECT, INDIRECT,
% INCIDENTAL, SPECIAL, EXEMPLARY, OR CONSEQUENTIAL DAMAGES (INCLUDING,
% BUT NOT LIMITED TO, PROCUREMENT OF SUBSTITUTE GOODS OR SERVICES;
% LOSS OF USE, DATA, OR PROFITS; OR BUSINESS INTERRUPTION) HOWEVER
% CAUSED AND ON ANY THEORY OF LIABILITY, WHETHER IN CONTRACT, STRICT
% LIABILITY, OR TORT (INCLUDING NEGLIGENCE OR OTHERWISE) ARISING IN
% ANY WAY OUT OF THE USE OF THIS SOFTWARE, EVEN IF ADVISED OF THE
% POSSIBILITY OF SUCH DAMAGE.
%%

%%
% This is the Eisvogel pandoc LaTeX template.
%
% For usage information and examples visit the official GitHub page:
% https://github.com/Wandmalfarbe/pandoc-latex-template
%%

% Options for packages loaded elsewhere
\PassOptionsToPackage{unicode}{hyperref}
\PassOptionsToPackage{hyphens}{url}
\PassOptionsToPackage{dvipsnames,svgnames,x11names,table}{xcolor}
%
\documentclass[
  paper=a4,
  ,captions=tableheading
]{scrartcl}
\usepackage{amsmath,amssymb}
% Use setspace anyway because we change the default line spacing.
% The spacing is changed early to affect the titlepage and the TOC.
\usepackage{setspace}
\setstretch{1.2}
\usepackage{iftex}
\ifPDFTeX
  \usepackage[T1]{fontenc}
  \usepackage[utf8]{inputenc}
  \usepackage{textcomp} % provide euro and other symbols
\else % if luatex or xetex
  \usepackage{unicode-math} % this also loads fontspec
  \defaultfontfeatures{Scale=MatchLowercase}
  \defaultfontfeatures[\rmfamily]{Ligatures=TeX,Scale=1}
\fi
\usepackage{lmodern}
\ifPDFTeX\else
  % xetex/luatex font selection
\fi
% Use upquote if available, for straight quotes in verbatim environments
\IfFileExists{upquote.sty}{\usepackage{upquote}}{}
\IfFileExists{microtype.sty}{% use microtype if available
  \usepackage[]{microtype}
  \UseMicrotypeSet[protrusion]{basicmath} % disable protrusion for tt fonts
}{}
\makeatletter
\@ifundefined{KOMAClassName}{% if non-KOMA class
  \IfFileExists{parskip.sty}{%
    \usepackage{parskip}
  }{% else
    \setlength{\parindent}{0pt}
    \setlength{\parskip}{6pt plus 2pt minus 1pt}}
}{% if KOMA class
  \KOMAoptions{parskip=half}}
\makeatother
\usepackage{xcolor}
\definecolor{default-linkcolor}{HTML}{A50000}
\definecolor{default-filecolor}{HTML}{A50000}
\definecolor{default-citecolor}{HTML}{4077C0}
\definecolor{default-urlcolor}{HTML}{4077C0}
\usepackage[margin=2.5cm,includehead=true,includefoot=true,centering,]{geometry}
% add backlinks to footnote references, cf. https://tex.stackexchange.com/questions/302266/make-footnote-clickable-both-ways
\usepackage{footnotebackref}
\setlength{\emergencystretch}{3em} % prevent overfull lines
\providecommand{\tightlist}{%
  \setlength{\itemsep}{0pt}\setlength{\parskip}{0pt}}
\setcounter{secnumdepth}{-\maxdimen} % remove section numbering
\ifLuaTeX
\usepackage[bidi=basic]{babel}
\else
\usepackage[bidi=default]{babel}
\fi
\babelprovide[main,import]{italian}
% get rid of language-specific shorthands (see #6817):
\let\LanguageShortHands\languageshorthands
\def\languageshorthands#1{}
\ifLuaTeX
  \usepackage{selnolig}  % disable illegal ligatures
\fi
\IfFileExists{bookmark.sty}{\usepackage{bookmark}}{\usepackage{hyperref}}
\IfFileExists{xurl.sty}{\usepackage{xurl}}{} % add URL line breaks if available
\urlstyle{same}
\hypersetup{
  pdftitle={Quesiti e dimostrazioni (con soluzioni) per primo parziale di Teoria dei Segnali},
  pdflang={it},
  pdfsubject={Segnali},
  hidelinks,
  breaklinks=true,
  pdfcreator={LaTeX via pandoc with the Eisvogel template}}
\title{Quesiti e dimostrazioni (con soluzioni) per primo parziale di
Teoria dei Segnali}
\author{}
\date{2024-02-22}



%%
%% added
%%


%
% for the background color of the title page
%

%
% break urls
%
\PassOptionsToPackage{hyphens}{url}

%
% When using babel or polyglossia with biblatex, loading csquotes is recommended
% to ensure that quoted texts are typeset according to the rules of your main language.
%
\usepackage{csquotes}

%
% captions
%
\definecolor{caption-color}{HTML}{777777}
\usepackage[font={stretch=1.2}, textfont={color=caption-color}, position=top, skip=4mm, labelfont=bf, singlelinecheck=false, justification=raggedright]{caption}
\setcapindent{0em}

%
% blockquote
%
\definecolor{blockquote-border}{RGB}{221,221,221}
\definecolor{blockquote-text}{RGB}{119,119,119}
\usepackage{mdframed}
\newmdenv[rightline=false,bottomline=false,topline=false,linewidth=3pt,linecolor=blockquote-border,skipabove=\parskip]{customblockquote}
\renewenvironment{quote}{\begin{customblockquote}\list{}{\rightmargin=0em\leftmargin=0em}%
\item\relax\color{blockquote-text}\ignorespaces}{\unskip\unskip\endlist\end{customblockquote}}

%
% Source Sans Pro as the default font family
% Source Code Pro for monospace text
%
% 'default' option sets the default
% font family to Source Sans Pro, not \sfdefault.
%
\ifnum 0\ifxetex 1\fi\ifluatex 1\fi=0 % if pdftex
    \usepackage[default]{sourcesanspro}
  \usepackage{sourcecodepro}
  \else % if not pdftex
    \usepackage[default]{sourcesanspro}
  \usepackage{sourcecodepro}

  % XeLaTeX specific adjustments for straight quotes: https://tex.stackexchange.com/a/354887
  % This issue is already fixed (see https://github.com/silkeh/latex-sourcecodepro/pull/5) but the
  % fix is still unreleased.
  % TODO: Remove this workaround when the new version of sourcecodepro is released on CTAN.
  \ifxetex
    \makeatletter
    \defaultfontfeatures[\ttfamily]
      { Numbers   = \sourcecodepro@figurestyle,
        Scale     = \SourceCodePro@scale,
        Extension = .otf }
    \setmonofont
      [ UprightFont    = *-\sourcecodepro@regstyle,
        ItalicFont     = *-\sourcecodepro@regstyle It,
        BoldFont       = *-\sourcecodepro@boldstyle,
        BoldItalicFont = *-\sourcecodepro@boldstyle It ]
      {SourceCodePro}
    \makeatother
  \fi
  \fi

%
% heading color
%
\definecolor{heading-color}{RGB}{40,40,40}
\addtokomafont{section}{\color{heading-color}}
% When using the classes report, scrreprt, book,
% scrbook or memoir, uncomment the following line.
%\addtokomafont{chapter}{\color{heading-color}}

%
% variables for title, author and date
%
\usepackage{titling}
\title{Quesiti e dimostrazioni (con soluzioni) per primo parziale di
Teoria dei Segnali}
\author{}
\date{2024-02-22}

%
% tables
%

%
% remove paragraph indentation
%
\setlength{\parindent}{0pt}
\setlength{\parskip}{6pt plus 2pt minus 1pt}
\setlength{\emergencystretch}{3em}  % prevent overfull lines

%
%
% Listings
%
%


%
% header and footer
%
\usepackage[headsepline,footsepline]{scrlayer-scrpage}

\newpairofpagestyles{eisvogel-header-footer}{
  \clearpairofpagestyles
  \ihead*{Quesiti e dimostrazioni (con soluzioni) per primo parziale di
Teoria dei Segnali}
  \chead*{}
  \ohead*{2024-02-22}
  \ifoot*{}
  \cfoot*{}
  \ofoot*{\thepage}
  \addtokomafont{pageheadfoot}{\upshape}
}
\pagestyle{eisvogel-header-footer}



%% custom packages
\usepackage{mathtools}
\usepackage[thinc]{esdiff}
\usepackage{physics}
\usepackage{wrapfig}
%\usepackage{titlesec}

%%
%% end added
%%

\begin{document}

%%
%% begin titlepage
%%

%%
%% end titlepage
%%

% \maketitle


\subsection{QUESITI}\label{quesiti}

\begin{enumerate}
\def\labelenumi{\arabic{enumi}.}
\tightlist
\item
  Un segnale deterministico a tempo continuo \(s(t)\) periodico ha
  Energia e Potenza finite e/o infinite?
\item
  Si definisca la Potenza istantanea di un segnale deterministico a
  tempo continuo \(s(t)\)
\item
  Si definisca la potenza media di un segnale deterministico a tempo
  continuo \(s(t)\)
\item
  Calcolare Energia e Potenza media del segnale a tempo continuo
  \(x(t) = Ae ^{-t} u(t)\).
\item
  Calcolare l'energia del segnale \(x(t) = sinc(t)\).
\item
  Calcolare Energia e Potenza media di un segnale a tempo continuo
  sinusoidale di ampiezza 2 e di periodo 2 secondi.
\item
  Calcolare Energia e Potenza media di un segnale a tempo continuo
  sinusoidale di ampiezza \(A\) e di periodo \(T\).
\item
  Calcolare Energia e Potenza media di un segnale a tempo continuo
  costante
\item
  Sotto quali condizioni un segnale deterministico a tempo continuo
  \(s(t)\) ha Potenza media finita?
\item
  Giustificare la seguente affermazione: ``Lo spettro di ampiezza di un
  segnale tempo-continuo aperiodico reale è una funzione pari''.
\item
  Giustificare la seguente affermazione: ``Lo spettro di fase di un
  segnale tempo-continuo aperiodico reale è una funzione dispari''.
\item
  Calcolare la quantità \(\int_{- \infty}^{\infty} sinc(t) dt\)
\item
  Tra onda quadra e onda triangolare, quale dei due segnali ha
  coefficienti della serie di Fourier con modulo che va a zero più
  velocemente al crescere di \(n\) e perché?
\item
  Tra la rampa e l'onda triangolare, quale dei due segnali ha
  coefficienti della serie di Fourier con modulo che va a zero più
  velocemente al crescere di \(n\), e perché?
\item
  Quali sono le proprietà della serie di Fourier di un segnale periodico
  pari
\item
  Quali sono le proprietà della serie di Fourier di un segnale periodico
  dispari
\item
  Se il segnale periodico \(x(t)\) è reale e pari, allora la serie di
  Fourier si semplifica in modo tale che \ldots{}
\item
  Se il segnale periodico \(x(t)\) è reale e dispari, allora la serie di
  Fourier si semplifica in modo tale che \ldots{}
\item
  Elencare le Condizioni di Dirichlet per la convergenza della serie di
  Fourier
\item
  Elencare le Condizioni di Dirichlet per la convergenza della
  trasformata di Fourier
\item
  Se il segnale aperiodico \(x(t)\) è pari, allora la sua trasformata di
  Fourier è
\item
  Se il segnale aperiodico \(x(t)\) è dispari, allora la sua trasformata
  di Fourier è
\item
  Se il segnale aperiodico \(x(t)\) è reale e pari, allora la sua
  trasformata di Fourier è
\item
  Se il segnale aperiodico \(x(t)\) è reale e dispari, allora la sua
  trasformata di Fourier è\ldots{}
\item
  Se la trasformata di Fourier di \(x(t)\) è \(X(f)\), allora la
  trasformata di \(x(\alpha t)\) con \(|\alpha| >1\) risulta modificata
  in modo che \ldots{}
\item
  Se la trasformata di Fourier di \(x(t)\) è \(X(f)\), allora la
  trasformata di \(x(\alpha t)\) con \(|\alpha| >1\) risulta modificata
  in modo che \ldots{}
\item
  In cosa differiscono le trasformate di Fourier di
  \(rect (\frac{t}{T})\) e di \(rect (\frac{t-5}{T})\)? Spiegare le
  differenze sia per lo spettro di ampiezza che per lo spettro di fase.
\item
  In cosa differiscono le trasformate di Fourier di
  \(rect(\frac{t}{T})\) e di \(rect(\frac{t}{4T})\)? Spiegare le
  differenze sia per lo spettro di ampiezza che per lo spettro di fase
\item
  Se \(x(t)\) è un segnale complesso e \(X(f)\) la sua trasformata di
  Fourier, quale è la trasformata di Fourier della parte reale di
  \(x(t\))? Giustificare la risposta
\item
  Se \(x(t)\) è un segnale complesso e \(X(f)\) la sua trasformata di
  Fourier, quale è la trasformata di Fourier della parte immaginaria di
  \(x(t\))? Giustificare la risposta
\item
  Quale è la trasformata di Fourier di \(e ^{-j2 \pi t}\)? Giustificare
  la risposta
\item
  Se \(x(t) \leftrightarrow X(f)\) sono una funzione e la sua
  trasformata di Fourier, qual è la trasformata di Fourier di \(x(-t)\)?
  Giustificare la risposta
\item
  In cosa differiscono le trasformate di Fourier di
  \(rect(\frac{t}{T})\) e di \(rect(\frac{t}{2T})\)? Spiegare le
  differenze sia per lo spettro di ampiezza che per lo spettro di fase.
\item
  Se il segnale periodico x(t) è reale, quali proprietà hanno
  rispettivamente lo spettro di ampiezza e lo spettro di fase?
\item
  Un sistema a tempo continuo si definisce causale se\ldots{} Fare poi
  un esempio di sistema causale e uno di sistema non causale.
\item
  Un sistema a tempo continuo si definisce stabile se
\item
  Dato un sistema LTI con risposta impulsiva
  \(h(t) = \delta (t - 1) +\delta (t - 2)\), disegnare l'uscita quando
  il suo ingresso è il segnale \(x(t) = u(t)\)
\item
  La risposta impulsiva di un sistema LTI causale è\ldots{}
\item
  La risposta impulsiva di un sistema LTI causale soddisfa la seguente
  condizione:
\item
  Un sistema che produce in uscita il valore assoluto del segnale al suo
  ingresso è un sistema lineare? Si argomenti la risposta con almeno un
  esempio.
\item
  Se un sistema LTI tempo-continuo ha una risposta impulsiva con energia
  finita, possiamo affermare che il sistema è stabile? Giustificare la
  risposta.
\item
  Quale è la risposta in frequenza di un sistema LTI retto
  dall'equazione differenziale \(y(t) - \frac{d^{2}y(t)}{dt^2} = x(t)\)?
\item
  Definire la densità spettrale di potenza di un segnale \(x(t)\) a
  potenza finita.
\item
  Un processo aleatorio si definisce stazionario in senso lato se
\item
  Elencare le proprietà che definiscono un processo aleatorio
  stazionario in senso lato (WSS).
\item
  Enunciare le proprietà che devono essere soddisfatte affinché un
  processo aleatorio sia stazionario in senso lato
\end{enumerate}

\subsection{DIMOSTRAZIONI}\label{dimostrazioni}

\begin{enumerate}
\def\labelenumi{\arabic{enumi}.}
\tightlist
\item
  Dimostrare che se un segnale periodico è reale, allora i coefficienti
  della sua espansione in serie di Fourier (nella sua forma complessa)
  sono caratterizzati da una simmetria Hermitiana.
\item
  Enunciare e dimostrare il Teorema di dualità della Trasformata
  continua di Fourier
\item
  Enunciare e dimostrare il Teorema del cambiamento di scala della
  Trasformata continua di Fourier
\item
  Enunciare e dimostrare il Teorema di integrazione completo della
  Trasformata continua di Fourier
\item
  Enunciare e dimostrare il Teorema della moltiplicazione (o prodotto?)
  della Trasformata continua di Fourier.
\item
  Dato un segnale x(t), enunciare e dimostrare la formula somma di
  Poisson che lega i coeffcienti della serie di Fourier di
  \(\displaystyle y(t) = \sum_{n = -\infty}^{\infty} x(t - nT)\) alla
  trasformata (aperiodica) di Fourier di x(t).
\item
  Dimostrare che la trasformata di Fourier del prodotto di funzioni
  \(x(t) \cdot y(t)\) è la convoluzione di \(X(f)\) con \(Y (f)\).
\item
  Enunciare e dimostrare il Teorema di Parseval
\item
  Enunciare il Teorema di Wiener-Khintchine
\end{enumerate}

\newpage

\subsection{RISPOSTE QUESITI}\label{risposte-quesiti}

\begin{enumerate}
\def\labelenumi{\arabic{enumi}.}
\tightlist
\item
  Un segnale deterministico a tempo continuo \(s(t)\) periodico ha
  potenza finita (pari alla potenza media calcolata in un singolo
  periodo) e energia infinita (somma di infinite aree).
\item
  La potenza istantanea di un segnale deterministico a tempo continuo
  \(s(t)\) è definita come \(P(t) = |s(t)|^2\).
\item
  La potenza media di un segnale deterministico a tempo continuo
  \(s(t)\) è definita come \[
  p(t) = \lim_{T\to\infty} \frac{1}{T} \int_{-\frac{T}{2}}^{\frac{T}{2}} s^2(t) \,dt
  \] dove T è il periodo del segnale.
\item
  Per \(x(t) = Ae^{-t}u(t),\) l'energia è \[
  E = \int_{0}^{\infty} |Ae^{-t}|^2 dt = A^2 \int_0^\infty e^{-2t} dt = \frac{A^2}{2}
  \] La potenza media è pari a: \[
  \displaystyle P_t = \lim_{T \to \infty} \frac{1}{T} \int_{-\frac{T}{2}}^{\frac{T}{2}} |x(t)|^2 dt = \lim_{T \to \infty} \frac{1}{T} \int_{-\frac{T}{2}}^{\frac{T}{2}} |A e^{-t} u(t)|^2 dt =  \lim_{T \to \infty} \frac{A^2}{T} \int_{0}^{\frac{T}{2}} e^{-2t} dt = \lim_{T \to \infty} \frac{A^2}{T}*0 = 0
  \]
\item
  L'energia del segnale \(x(t) = sinc(t)\) è pari ad uno, in quanto
  utilizzando il teorema di Parseval,
  \(\displaystyle\int_{-\infty}^{+\infty}|x(t)|^2dt = \int_{-\infty}^{+\infty}|X(f)|^2df\),
  e la trasformata di \(sinc(t)\) è pari a \(rect(f)\), quindi
  equivalente ad un rettangolo di altezza e base 1.
  \(\displaystyle E = \int_{-\infty}^\infty (\frac{sin(\pi t)}{( \pi t)})^2 dt = 1.\)
\item
  Per un segnale sinusoidale di ampiezza 2 e periodo 2 secondi,
  l'energia è infinita perché il segnale è periodico e la potenza media
  è \(P_m = \frac{1}{2} * (2^2) = 2.\)
\item
  Per un segnale sinusoidale di ampiezza A e periodo T, l'energia è
  infinita perché il segnale è periodico e la potenza media è
  \(P_m = \frac{1}{2} * A^2\).
\item
  Per un segnale costante, l'energia è infinita perché il segnale è
  periodico e la potenza media è \(P_m = C^2,\) dove C è il valore
  costante.
\item
  Un segnale deterministico a tempo continuo \(s(t)\) ha potenza media
  finita se:
  \(\displaystyle \lim_{s\to\infty} \frac{1}{T} \int_{-\frac{T}{2}}^{\frac{T}{2}} s^2(t)dt < \infty\).
\item
  Lo spettro di ampiezza di un segnale tempo-continuo aperiodico reale è
  una funzione pari. La trasformata di Fourier di un segnale reale gode
  della proprietà di simmetria Hermitiana \((X(f) = X^*(-f))\): quindi
  ha componenti simmetriche rispetto all'asse delle ordinate
  \((R(f)=R(-f); A(f)=|X(f)| \to A(f)=A(-f))\).
\item
  Lo spettro di fase di un segnale tempo-continuo aperiodico reale è una
  funzione dispari. La trasformata di Fourier di un segnale reale gode
  della proprietà di simmetria Hermitiana \((X(f) = X^*(-f))\): quindi
  la fase della trasformata di Fourier di un segnale reale è uguale
  all'opposto della fase della sua componente coniugata.
  \((I(f)=-I(-f); \Theta(f)=\angle X(f) \to \Theta(f)=-\Theta(-f))\).
\item
  \(\displaystyle \int_{-\infty}^\infty sinc(t)dt = 1\), in quanto
  integrale di una funzione \(sinc\) normalizzata.
\item
  L'onda triangolare, non presentando discontinuità (a differenza del
  segnale onda quadra), nella sua ricostruzione del segnale tramite
  serie di Fourier le componenti ad alta frequenza hanno importanza
  minore.
\item
  L'onda triangolare, non presentando discontinuità, (a differenza del
  segnale dente di sega), nella sua ricostruzione del segnale tramite
  serie di Fourier le componenti ad alta frequenza hanno importanza
  minore.
\item
  \begin{enumerate}
  \def\labelenumii{\arabic{enumii}.}
  \tightlist
  \item
    Coefficiente della serie di Fourier è una funzione
    pari:\(X_k = X_{-k}\);
  \item
    se il segnale è anche reale: \(X_k\) reale e pari
    \((X_k = X^*_k) \to\) si sviluppa in soli coseni.
    \(\displaystyle x(t)= X_0 + 2 \sum_{k=1}^{\infty} X_k cos (2\pi k f_0 t)\)
  \end{enumerate}
\item
  \begin{enumerate}
  \def\labelenumii{\arabic{enumii}.}
  \tightlist
  \item
    Coefficiente della serie di Fourier è una funzione
    dispari:\(X_{-k} = -X_{k}\);
  \item
    se il segnale è anche reale: \(X_k\) immaginaria pura e dispari
    \((-X_k = X^*_k) \to\) si sviluppa in soli seni.
    \(\displaystyle x(t)= 2j \sum_{k=1}^{\infty} X_k sin (2\pi k f_0 t)\)
  \end{enumerate}
\item
  \(\displaystyle x(t)=X_0 + 2\sum_{k=1}^{\infty} X_k cos (2\pi k f_0 t) \to X_k = \frac{2}{T_0} \int_{0}^{\frac{T_0}{2}} x(t) cos (2\pi k f_0 t) dt\)
\item
  \(\displaystyle x(t)= 2j \sum_{k=1}^{\infty} X_k sin (2\pi k f_0 t) \to X_k = -\frac{2j}{T_0} \int_{0}^{\frac{T_0}{2}} x(t) sin (2\pi k f_0 t)dt\)
\item
  Le Condizioni di Dirichlet per la convergenza della serie di Fourier
  sono:

  \begin{enumerate}
  \def\labelenumii{\arabic{enumii}.}
  \tightlist
  \item
    la funzione deve essere assolutamente integrabile sul periodo
    \(T_0\):
    \(\displaystyle \int_{-\frac{T_0}{2}}^{\frac{T_0}{2}} |x(t)| dt < +\infty\)
  \item
    la funzione deve essere continua o presentare un numero finito di
    discontinuità di prima specie
  \item
    la funzione deve avere un numero finito di massimi e minimi
    all'interno di un periodo. Oppure \(x(t)\) derivabile rispetto al
    tempo nel periodo \(T_0\), esclusi al più un numero finito di
    discontinuità di prima specie.
  \end{enumerate}
\item
  Le Condizioni di Dirichlet per la convergenza della trasformata di
  Fourier sono simili a quelle per la serie di Fourier:

  \begin{enumerate}
  \def\labelenumii{\arabic{enumii}.}
  \tightlist
  \item
    la funzione deve essere assolutamente sommabile:
    \(\displaystyle \int_{-\infty}^{\infty} |x(t)| dt < +\infty\)
  \item
    se in qualunque intervallo finito \((t_1, t_2)\) è continua o
    presenta un numero finito di discontinuità di prima specie
  \item
    se in qualunque intervallo finito \((t_1, t_2)\) la funzione ha un
    numero finito di massimi e minimi.
  \end{enumerate}
\item
  La trasformata è pari a sua volta.
\item
  La trasformata è dispari a sua volta.
\item
  Se il segnale aperiodico \(x(t)\) è reale e pari, la sua trasformata
  di Fourier è reale e pari.
\item
  Se il segnale aperiodico \(x(t)\) è reale e dispari, la sua
  trasformata di Fourier è immaginaria pura e dispari.
\item
  Se la trasformata di Fourier di \(x(t)\) è \(X(f)\), allora la
  trasformata di \(x(\alpha t)\) con\(|\alpha| > 1\) è
  \(\displaystyle \frac{1}{\alpha} X(\frac{f}{\alpha})\). Quindi per il
  teorema del cambiamento di scala, con una compressione nel tempo
  abbiamo una dilatazione in frequenza.
\item
  Se la trasformata di Fourier di \(x(t)\) è \(X(f)\), allora la
  trasformata di \(x(\alpha t)\) con\(|\alpha| < 1\) è
  \(\displaystyle \frac{1}{|\alpha|} X(\frac{f}{\alpha})\). Quindi per
  il teorema del cambiamento di scala, con una dilatazione nel tempo
  abbiamo una compressione in frequenza.
\item
  Le trasformate di Fourier di \(rect (\frac{t}{T})\) e di
  \(rect (\frac{t-5}{T})\) differiscono solo per la fase. Lo spettro di
  ampiezza è lo stesso, mentre lo spettro di fase di
  \(rect (\frac{t-5}{T})\) ha una componente lineare aggiuntiva rispetto
  a \(rect (\frac{t}{T})\).
\item
  Per il teorema del cambiamento di scala, lo spettro di ampiezza di
  \(rect(\frac{t}{4T})\) viene alterato rispetto a
  \(rect(\frac{t}{T})\), mentre lo spettro di fase non viene alterato.
  (Dilatazione nel tempo \(\leftrightarrow\) Compressione in frequenza)
\item
  \(\displaystyle R(f) = \int_{- \infty}^{+\infty} x(t) cos(2\pi ft)dt\)
  Dato che \(\displaystyle X(f) = R(f)+ jI(f)\)
\item
  \(\displaystyle I(f) =- \int_{- \infty}^{+\infty} x(t) sin(2\pi ft)dt\)
  Dato che \(\displaystyle X(f) = R(f)+ jI(f)\)
\item
  La trasformata di Fourier di \(e^{-j2\pi t}\) è un impulso di Dirac
  centrato in \(f = 1\). Utilizzando il teorema della traslazione in
  frequenza: \(\to 1*e^{-j2 \pi t} =\Leftrightarrow \delta (f+1)\)
\item
  Per il teorema del cambiamento di scala:
  \(\displaystyle x(\alpha t) = \frac{1}{|\alpha|} X(\frac{f}{\alpha})\)
  con \(\alpha = -1\), allora: \(\to x(-t) \Leftrightarrow X(-f)\)
\item
  Per il teorema del cambiamento di scala, lo spettro di ampiezza di
  \(rect(\frac{t}{2T})\) viene alterato rispetto a
  \(rect(\frac{t}{T})\), mentre lo spettro di fase non viene alterato.
  (Dilatazione nel tempo \(\leftrightarrow\) Compressione in frequenza)
\item
  Se il segnale periodico \(x(t) \in \mathbb{R}\), dato che il
  coefficiente della serie di Fourier \(X_k\) gode della simmetria
  Hermitiana,
  \(X_{-k} =X_k^* = \begin{cases} |X_k| =|X_{-k}| \\ \angle X_k = - \angle X_{-k} \end{cases} \to\)
  lo spettro di ampiezza è pari e lo spettro di fase è dispari.
\item
  Quando il valore dell'uscita al tempo \(t\) dipende soltanto dai
  valori assunti dall'ingresso agli istanti precedenti.
  \(\to y(t) = T[x(\alpha), \alpha < t, t]\) Un esempio di sistema
  causale è un filtro passa-basso RC (o un moltiplicatore?), mentre un
  esempio di sistema non causale è un derivatore.
\item
  Quando il sistema, se sollecitato da un segnale con ampiezza limitata,
  produce in uscita un segnale a sua volta con ampiezza limitata.
\item
  Dato un sistema LTI con risposta impulsiva
  \(h(t) = \delta(t - 1) + \delta(t - 2)\), l'uscita quando il suo
  ingresso è il segnale \(x(t) = u(t)\) sarà
  \(y(t) = u(t - 1) + u(t - 2).\)
\item
  La risposta impulsiva di un sistema LTI causale è un segnale causale
  \(h(t)\) tale che \(h(t) = 0\) per \(t < 0\).
\item
  La risposta impulsiva di un sistema LTI causale soddisfa la condizione
  \(h(t) = 0\) per \(t < 0\).
\item
  Un sistema che produce in uscita il valore assoluto del segnale
  (raddrizzatore a doppia onda) al suo ingresso non è un sistema
  lineare. Ad esempio: se
  \(\displaystyle x(t)=  x_1(t)+x_2(t) \to T[x(t)]  = |x(t)| = |x_1(t)+x_2(t)| \neq y_1(t)+y_2(t) = |x_1(t)|+|x_2(t)|\).
  Altro esempio: Siano \(x_1(t) = t\) e \(x_2(t) = -t\): il sistema
  produce \(y_1(t) = |t|\) e \(y_2(t) = |-t| = t\). Tuttavia,
  \(y_1(t) + y_2(t) = 2t\), mentre il sistema applicato alla somma dei
  segnali in ingresso produce \(|t - t| = 0\), che non è uguale a\(2t\).
\item
  Se un sistema LTI tempo-continuo ha una risposta impulsiva con energia
  finita, possiamo affermare che il sistema è stabile, perché l'energia
  finita implica che la risposta impulsiva è assolutamente integrabile
  \(\to  \displaystyle \int_{-\infty}^\infty |h(t)| dt < \infty\).
\item
  Passando nel dominio della frequenza:
  \(\displaystyle Y(f)(1-(j2\pi f)^2)=X(f)\). Sapendo che
  \(Y(f)=X(f)H(f)\),
  \(\displaystyle H(f)X(f)(1+4\pi^2 f^2)=H(f) \to H(f)=\frac{1}{1+4\pi^2 f^2}\)
\item
  La densità spettrale di potenza di un segnale \(x(t)\) è definita
  come:
  \(\displaystyle S_x(f) \triangleq \lim_{T \to \infty} \frac{E_{X_T}(f)}{T} = \lim_{T \to \infty} \frac{|X_T(f)|^2}{T}\).
  Per il teorema di Wiener-Khintchine afferma però che la densità
  spettrale di potenza\(^*\) è uguale alla trasformata di Fourier della
  funzione di autocorrelazione così modificata: \[
  R_x(\tau) \triangleq  \lim_{T \to \infty} \frac{1}{T} \int_{-\frac{T}{2}}^{\frac{T}{2}}x(t)x(t-\tau)
  \] La densità spettrale di potenza è:
  \begin{gather*}\displaystyle S_x(f)= \int_{-\infty}^{\infty} R_x(\tau)e^{-j2 \pi f \tau} d\tau = 2\int_{0}^{\infty}  R_x(\tau) cos(2 \pi f \tau) d\tau
  \end{gather*}
\end{enumerate}

\end{document}
