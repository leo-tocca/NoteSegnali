% Options for packages loaded elsewhere
\PassOptionsToPackage{unicode}{hyperref}
\PassOptionsToPackage{hyphens}{url}
%
\documentclass[
]{article}
\usepackage{lmodern}
\usepackage[a4paper, total={6in, 10in}]{geometry}
\usepackage{amssymb,amsmath}
\usepackage{ifxetex,ifluatex}
\ifnum 0\ifxetex 1\fi\ifluatex 1\fi=0 % if pdftex
  \usepackage[T1]{fontenc}
  \usepackage[utf8]{inputenc}
  \usepackage{textcomp} % provide euro and other symbols
\else % if luatex or xetex
  \usepackage{unicode-math}
  \defaultfontfeatures{Scale=MatchLowercase}
  \defaultfontfeatures[\rmfamily]{Ligatures=TeX,Scale=1}
\fi
% Use upquote if available, for straight quotes in verbatim environments
\IfFileExists{upquote.sty}{\usepackage{upquote}}{}
\IfFileExists{microtype.sty}{% use microtype if available
  \usepackage[]{microtype}
  \UseMicrotypeSet[protrusion]{basicmath} % disable protrusion for tt fonts
}{}
\makeatletter
\@ifundefined{KOMAClassName}{% if non-KOMA class
  \IfFileExists{parskip.sty}{%
    \usepackage{parskip}
  }{% else
    \setlength{\parindent}{0pt}
    \setlength{\parskip}{6pt plus 2pt minus 1pt}}
}{% if KOMA class
  \KOMAoptions{parskip=half}}
\makeatother
\usepackage{xcolor}
\IfFileExists{xurl.sty}{\usepackage{xurl}}{} % add URL line breaks if available
\IfFileExists{bookmark.sty}{\usepackage{bookmark}}{\usepackage{hyperref}}
\hypersetup{
  hidelinks,
  pdfcreator={LaTeX via pandoc}}
\urlstyle{same} % disable monospaced font for URLs
\setlength{\emergencystretch}{3em} % prevent overfull lines
\providecommand{\tightlist}{%
  \setlength{\itemsep}{0pt}\setlength{\parskip}{0pt}}
\setcounter{secnumdepth}{-\maxdimen} % remove section numbering

\hypersetup{
	pdftitle={Dimostrazioni e quesiti per primo parziale di Teoria dei Segnali},
	pdfauthor={Leonardo Toccafondi},
	hidelinks,
	pdfcreator={LaTeX via pandoc}}

\title{Dimostrazioni e quesiti per primo parziale di Teoria dei Segnali}
\author{}
\date{}

\begin{document}
	\maketitle

\hypertarget{quesiti}{%
\subsection{QUESITI}\label{quesiti}}

\begin{enumerate}
\def\labelenumi{\arabic{enumi}.}

\item
	Un segnale deterministico a tempo continuo s(t) periodico ha Energia e Potenza finite e/o infinite ?
	
\item
Si definisca la Potenza istantanea di un segnale deterministico a
tempo continuo $s(t)$

\item 
Si definisca la potenza media di un segnale deterministico a tempo continuo $s(t)$

\item
Calcolare Energia e Potenza media del segnale a tempo continuo
\(x(t) = Ae ^{-t} u(t)\).

\item
Calcolare l'energia del segnale x(t) = sinc(t).

\item
Calcolare Energia e Potenza media di un segnale a tempo continuo
sinusoidale di ampiezza 2 e di periodo 2 secondi.
\item
Calcolare Energia e Potenza media di un segnale a tempo continuo
sinusoidale di ampiezza A e di periodo T.

\item
Calcolare Energia e Potenza media di un segnale a tempo continuo
costante

\item
Giustificare la seguente affermazione: ``Lo spettro di ampiezza di un
segnale tempo-continuo aperiodico reale è una funzione pari''.


\item
Calcolare la quantità
\(\displaystyle \int_{- \infty}^{\infty} sinc(t) dt\)
	
\item 
Tra onda quadra e onda triangolare, quale dei due segnali ha coefficienti della serie di Fourier con modulo che
va a zero più velocemente al crescere di n?

\item
Tra la rampa e l'onda triangolare, quale dei due segnali ha
coefficienti della serie di Fourier con modulo che va a zero più
velocemente al crescere di n, e perché?

\item 
Se il segnale periodico x(t) è reale e dispari, allora la serie di Fourier si semplifica in modo tale che ...


\item 
Se il segnale periodico $x(t)$ è reale e pari, allora la serie di Fourier si semplifica in modo tale che ...


\item
Elencare le Condizioni di Dirichlet per la convergenza della serie di
Fourier

\item
Quali sono le proprietà della serie di Fourier di un segnale periodico
pari

\item 
Se il segnale aperiodico $x(t)$ è pari, allora la sua trasformata di Fourier è	



\item 
Se il segnale aperiodico $x(t)$ è reale e pari, allora la sua trasformata di Fourier è

\item 
Se la trasformata di Fourier di $x(t)$ è $X(f)$, allora la trasformata di $x(at)$ con $|a|>1$ risulta modificata in modo
che ...



\item
  Se il segnale aperiodico x(t) è reale e dispari, allora la sua
  trasformata di Fourier è\ldots{}
\item
  In cosa differiscono le trasformate di Fourier di
  \(rect (\frac{t}{T})\) e di \(rect (\frac{t-5}{T})\)? Spiegare le
  differenze sia per lo spettro di ampiezza che per lo spettro di fase.


\item
  Se \(x(t)\) è un segnale complesso e \(X(f)\) la sua trasformata di
  Fourier, quale è la trasformata di Fourier della parte reale di
  \(x(t\))? Giustificare la risposta
\item
  Quale è la trasformata di Fourier di $ e ^{-j2 \pi t}$?
  Giustificare la risposta
\item
  Se \(x(t) \leftrightarrow X(f)\) sono una funzione e la sua
  trasformata di Fourier, qual è la trasformata di Fourier di \(x(-t)\)?
  Giustificare la risposta






\item
  In cosa differiscono le trasformate di Fourier di $rect(\frac{t}{T})$ e
  di $rect(\frac{t}{2T})$ ? Spiegare le differenze sia per lo spettro di
  ampiezza che per lo spettro di fase.


\item
  Se il segnale periodico x(t) è reale, quali proprietà hanno
  rispettivamente lo spettro di ampiezza e lo spettro di fase?


\item
  Se il segnale aperiodico x(t) è dispari, allora la sua trasformata di
  Fourier è


\item
Un sistema a tempo continuo si definisce causale se\ldots{} Fare poi
un esempio di sistema causale e uno di sistema non causale.

\item
Un sistema a tempo continuo si definisce stabile se

\item
Dato un sistema LTI con risposta impulsiva $h(t) = \delta (t - 1) +\delta (t - 2)$,
disegnare l'uscita quando il suo ingresso è il segnale $x(t) = u(t)$  
\item 
La risposta impulsiva di un sistema LTI causale è
\item
La risposta impulsiva di un sistema LTI causale soddisfa la seguente
condizione:
\item
Un sistema che produce in uscita il valore assoluto del segnale al suo
ingresso è un sistema lineare? Si argomenti la risposta con almeno un
esempio.
\item
Se un sistema LTI tempo-continuo ha una risposta impulsiva con energia
finita, possiamo affermare che il sistema è stabile? Giustificare la
risposta.
\item
  Un processo aleatorio si definisce stazionario in senso lato se
\item
  Elencare le proprietà che definiscono un processo aleatorio
  stazionario in senso lato (WSS).
\item
Quale è la risposta in frequenza di un sistema LTI retto
dall'equazione differenziale $y(t) - \frac{d^{2}y(t)}{dt^2} = x(t)$?
\item
Definire la densità spettrale di potenza di un segnale \(x(t)\) a
potenza finita.
  
\end{enumerate}

\hypertarget{dimostrazioni}{%
\subsection{DIMOSTRAZIONI}\label{dimostrazioni}}

\begin{enumerate}
\def\labelenumi{\arabic{enumi}.}
\item
  Enunciare e dimostrare il Teorema di dualità della Trasformata
  continua di Fourier
\item
  Enunciare e dimostrare il Teorema del cambiamento di scala della
  Trasformata continua di Fourier
\item
  Enunciare e dimostrare il Teorema di integrazione completo della
  Trasformata continua di Fourier
\item
  Enunciare e dimostrare il Teorema di Parseval
\item
  Dimostrare che se un segnale periodico è reale, allora i coefficienti
  della sua espansione in serie di Fourier (nella sua forma complessa)
  sono caratterizzati da una simmetria Hermitiana.
\item
  Dato un segnale x(t), enunciare e dimostrare la formula somma di
  Poisson che lega i coeffcienti della serie di Fourier di \(\displaystyle y(t) = \sum_{n = -\infty}^{\infty} x(t - nT)\)
  alla trasformata (aperiodica) di Fourier di x(t).
\item
  Dimostrare che la trasformata di Fourier del prodotto di funzioni
  \(x(t) \cdot y(t)\) è la convoluzione di \(X(f)\) con \(Y (f)\).
\item
  Enunciare e dimostrare il teorema di integrazione completo della
  trasformata continua di Fourier.
\item
  Enunciare il criterio di Dirichlet della serie di Fourier
\item
  Enunciare il Teorema di Wiener-Khintchine
\end{enumerate}

\end{document}
