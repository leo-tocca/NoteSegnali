% Options for packages loaded elsewhere
\PassOptionsToPackage{unicode}{hyperref}
\PassOptionsToPackage{hyphens}{url}
%
\documentclass[
]{article}
\usepackage{amsmath,amssymb}
\usepackage{iftex}
\ifPDFTeX
  \usepackage[T1]{fontenc}
  \usepackage[utf8]{inputenc}
  \usepackage{textcomp} % provide euro and other symbols
\else % if luatex or xetex
  \usepackage{unicode-math} % this also loads fontspec
  \defaultfontfeatures{Scale=MatchLowercase}
  \defaultfontfeatures[\rmfamily]{Ligatures=TeX,Scale=1}
\fi
\usepackage{lmodern}
\ifPDFTeX\else
  % xetex/luatex font selection
\fi
% Use upquote if available, for straight quotes in verbatim environments
\IfFileExists{upquote.sty}{\usepackage{upquote}}{}
\IfFileExists{microtype.sty}{% use microtype if available
  \usepackage[]{microtype}
  \UseMicrotypeSet[protrusion]{basicmath} % disable protrusion for tt fonts
}{}
\makeatletter
\@ifundefined{KOMAClassName}{% if non-KOMA class
  \IfFileExists{parskip.sty}{%
    \usepackage{parskip}
  }{% else
    \setlength{\parindent}{0pt}
    \setlength{\parskip}{6pt plus 2pt minus 1pt}}
}{% if KOMA class
  \KOMAoptions{parskip=half}}
\makeatother
\usepackage{xcolor}
\ifLuaTeX
  \usepackage{luacolor}
  \usepackage[soul]{lua-ul}
\else
  \usepackage{soul}
\fi
\setlength{\emergencystretch}{3em} % prevent overfull lines
\providecommand{\tightlist}{%
  \setlength{\itemsep}{0pt}\setlength{\parskip}{0pt}}
\setcounter{secnumdepth}{-\maxdimen} % remove section numbering
\usepackage{cancel}
\usepackage{steinmetz}
\usepackage{derivative}
\usepackage{mathtools}
\usepackage{siunitx}
\usepackage[italian]{babel}
\DeclareMathOperator{\sinc}{sinc}
\DeclareMathOperator{\rect}{rect}
\DeclareMathOperator{\tfs}{TFS}
\newcommand{\dft}{\operatorname{DFT}}
\newcommand{\tf}[1]{\text{T}\Big[ #1 \Big]}
\newcommand{\Abs}[1]{\Big| #1 \Big|}
\newcommand{\ov}[1]{\overline{#1}}
\usepackage{geometry}
    \geometry{
        a4paper,
        total={170mm,257mm},
        left=20mm,
        top=20mm,
    }
\ifLuaTeX
  \usepackage{selnolig}  % disable illegal ligatures
\fi
\IfFileExists{bookmark.sty}{\usepackage{bookmark}}{\usepackage{hyperref}}
\IfFileExists{xurl.sty}{\usepackage{xurl}}{} % add URL line breaks if available
\urlstyle{same}
\hypersetup{
  hidelinks,
  pdfcreator={LaTeX via pandoc}}

\author{}
\date{}

\begin{document}

\subsection{QUESITI}\label{quesiti}

\subparagraph{Processo WSS}\label{processo-wss}

\begin{enumerate}
\def\labelenumi{\arabic{enumi}.}
\tightlist
\item
  Enunciare le proprietà che devono essere soddisfatte affinché un
  processo aleatorio sia stazionario in senso stretto {[}\(^{**}\){]}
\item
  Enunciare le proprietà che devono essere soddisfatte affinché un
  processo aleatorio sia stazionario in senso lato
\item
  Elencare le proprietà che caratterizzano un processo aleatorio
  stazionario in senso lato (WSS).
\item
  Cos'è il rumore bianco?{[}\(^{**}\){]}
\item
  Se \(x(t)\) è un processo WSS in ingresso ad un sistema SLS con
  risposta impulsiva \(h(t)\), il segnale \(y(t)\) ottenuto in uscita è
  a sua volta un processo WSS? {[}\(^{**}\){]}
\end{enumerate}

\subparagraph{Trasformata di Fourier per una
Sequenza}\label{trasformata-di-fourier-per-una-sequenza}

\begin{enumerate}
\def\labelenumi{\arabic{enumi}.}
\setcounter{enumi}{5}
\tightlist
\item
  Che significato ha l'espansione della sequenza \(x[n]\)?
  {[}\(^{**}\){]}
\item
  Qual è la condizione sufficiente per l'esistenza della trasformata per
  sequenze? {[}\(^{**}\){]}
\item
  Cosa s'intende per denormalizzazione e a cosa serve?
\item
  Enunciare e spiegare il teorema dell'Incremento della TDF per sequenze
  reali
\item
  Enunciare e spiegare la condizione di convergenza della Trasformata di
  Fourier per sequenze (vedi n°6)
\item
  Sia \(x[n] = cos(2\pi 0.1 n), \ n \in \mathbb{Z}\). Quale è la sua
  trasformata di Fourier per sequenze? Giustificare la risposta.
\item
  Se \(x[n]\) ha trasformata di Fourier per sequenze
  \(\overline{X}(F)\), quale sequenza \(y[n]\) ha trasformata di Fourier
  \(\overline{Y}(F) = \overline{X}(F - F_0)\)? Giustificare la risposta.
\item
  Calcolare la trasformata di Fourier della sequenza \(x[n]\) formata
  dall'impulso rettangolare discreto, cioè \(x[n]= u[n]-u[n-N]\)
\end{enumerate}

Campionamento e interpolazione

\begin{enumerate}
\def\labelenumi{\arabic{enumi}.}
\setcounter{enumi}{13}
\tightlist
\item
  Definizione teorema del campionamento e condizione di Nyquist
  {[}\(^{**}\){]}
\item
  Cosa si intende per \emph{aliasing}? Come si evita e perché va
  evitato. {[}\(^{**}\){]}
\item
  Il segnale \(x(t) = e^{-t} u(t)\) può essere campionato con assoluta
  assenza di aliasing? Giustificare la risposta.
\item
  Data una serie di campioni \(x(nT)\) ottenuti campionando il segnale
  analogico di partenza \(x(t)\), scrivere la relazione che lega il
  segnale ricostruito \(\hat{x}(t)\) ai campioni \(x(nT)\), nel caso di
  interpolazione lineare.
\item
  Dato il segnale campionato \(x_c(t)\) ottenuto campionando il segnale
  analogico di partenza \(x(t)\), scrivere la relazione che lega lo
  spettro di \(x_c(t)\) allo spettro di \(x(t)\).
\item
  Differenza tra interpolazione a mantenimento, cardinale (e/o lineare)
  \(^{**}\)
\item
  Quali sono gli svantaggi e i vantaggi dell'interpolazione a
  mantenimento? {[}\(^{**}\){]}
\item
  Scrivere l'espressione del segnale interpolato in funzione dei valori
  della sequenza di campioni nel caso di Interpolazione a mantenimento
\end{enumerate}

\subparagraph{Trasformata di Fourier
discreta}\label{trasformata-di-fourier-discreta}

\begin{enumerate}
\def\labelenumi{\arabic{enumi}.}
\setcounter{enumi}{21}
\tightlist
\item
  Che differenza troviamo tra una trasformata discreta ed una per
  sequenze {[}\(^{**}\){]}
\item
  Da quanti campioni non nulli è composta la DFT a 20 campioni del
  segnale di durata finita \(x[n] = cos^2(\frac{2\pi}{10}n)\),
  \(n = 0, 1, . . . , 19\)? Giustificare la risposta.
\item
  Scrivere i Teoremi del prodotto e della convoluzione della DFT
\item
  Enunciare e spiegare la Proprietà di Simmetria della TDF per sequenze
  reali
\item
  Scrivere la proprietà di traslazione in frequenza della DFT
\item
  Enunciare la proprietà di traslazione circolare della Trasformata
  Discreta di Fourier
\item
  Enunciare e spiegare la Proprietà di Simmetria della Trasformata
  Discreta di Fourier per sequenze reali
\item
  Sia data una sequenza finita \(x[n]\), di lunghezza \(N = 10\)
  campioni. Indicando con \(X_{10}[k]\) e \(X_{20}[k]\) le DFT di
  \(x[n]\) calcolate, rispettivamente, con periodicità \(L = 10\) e
  \(L = 20\), quali campioni di \(X_{20}[k]\) coincidono con campioni di
  \(X_{10}[k]\)? Giustificare la risposta.
\end{enumerate}

\subparagraph{Sistemi LTI discreti}\label{sistemi-lti-discreti}

\begin{enumerate}
\def\labelenumi{\arabic{enumi}.}
\setcounter{enumi}{29}
\tightlist
\item
  Scrivere la risposta impulsiva di un sistema discreto che implementa
  una finestra mobile
\item
  Scrivere la risposta impulsiva di un sistema discreto che implementa
  un accumulatore o integratore numerico
\item
  Descrivere un filtro discreto a media mobile
\item
  Descrivere il filtro detto accumulatore o integratore numerico
\item
  Descrivere il filtro derivatore numerico o operatore
\item
  Giustificare la seguente affermazione: ``Un sistema LTI è stabile se
  la sua funzione di trasferimento ha una regione di convergenza che
  include la circonferenza unitaria del piano z''. \(^*\).
\item
  Si supponga di voler usare un algoritmo di convoluzione veloce per
  eseguire il filtraggio di un segnale con un sistema LTI di tipo FIR,
  avente una risposta impulsiva lunga \(N = 200\) campioni. Misurando la
  complessità in termini di moltiplicazioni reali per campione di
  uscita, è più conveniente usare (per il calcolo della convoluzione
  circolare) una FFT con periodicità \(L = 2048\) oppure una con
  periodicità \(L = 512\)? Giustificare la risposta.
\item
  In uno schema di convoluzione veloce, quante moltiplicazioni reali per
  campione di uscita devono essere effettuate? Giustificare la risposta
\item
  Spiegare la differenza tra sistemi lineari e stazionari a tempo
  discreto di tipo FIR e di tipo IIR.
\end{enumerate}

\subparagraph{Quantizzazione}\label{quantizzazione}

\begin{enumerate}
\def\labelenumi{\arabic{enumi}.}
\setcounter{enumi}{38}
\tightlist
\item
  Spiegare la differenza tra un quantizzatore uniforme (a passo
  \(\Delta\) e a \(B\) bit) di tipo midtread e uno di tipo midrise
\item
  Dato un quantizzatore uniforme, scrivere le relazioni che permettono
  di trovare il valore quantizzato \(\hat{x}(nT)\) a partire dal
  campione \(x(nT)\) nel caso dell'operazione di arrotondamento e di
  troncamento
\item
  Enunciare le ipotesi che usualmente vengono assunte per il rumore di
  quantizzazione.
\item
  Giustificare la seguente affermazione: ``Quantizzando un segnale
  sinusoidale di ampiezza unitaria con un convertitore
  analogico-digitale avente \(B\) bit di quantizzazione e dinamica
  \([-1, 1]\) si ottiene un rapporto segnale-rumore 0 (espresso in
  \(dB\)) dato da \(SNR \approx 6.02B + 1.76\)''.
\end{enumerate}

\subparagraph{Sistemi di comunicazione
digitale}\label{sistemi-di-comunicazione-digitale}

\begin{enumerate}
\def\labelenumi{\arabic{enumi}.}
\setcounter{enumi}{42}
\tightlist
\item
  Spiegare il ruolo svolto dal codificatore di canale in un sistema di
  comunicazione digitale
\item
  Spiegare il ruolo svolto da un codificatore di sorgente nella catena
  di trasmissione digitale.
\item
  Spiegare il ruolo svolto da un modulatore digitale nella catena di
  trasmissione digitale.
\item
  Come ridurre gli effetti del rumore?
\item
  Descrivere gli schemi di base di modulazione digitale utilizzati in un
  sistema di comunicazione
\item
  Spiegare quali sono le tecniche multiplex utilizzate in un sistema di
  trasmissione digitale
\end{enumerate}

\subsection{DIMOSTRAZIONI}\label{dimostrazioni}

\begin{enumerate}
\def\labelenumi{\arabic{enumi}.}
\item
  Enunciare e dimostrare il Teorema del prodotto della TDF di una
  sequenza
\item
  Mostrare come la TDF di una sequenza ottenuta per campionamento si
  ricava come periodicizzazione della TDF del segnale di partenza
  {[}\(^{**}\){]}
\item
  Enunciare e dimostrare la relazione che esiste tra la trasformata di
  Fourier di una sequenza \(x[n]\) ottenuta per campionamento di un
  segnale continuo \(x(t)\), e la Trasformata di Fourier di \(x(t)\)
  stesso
\item
  Dimostrare che dalla relazione di antitrasformata discreta di Fourier
  discende la relazione di trasformata discreta di Fourier
\item
  Enunciare e dimostrare il Teorema della convoluzione della Trasformata
  discreta di Fourier
\item
  Enunciare e dimostrare il teorema di Parseval nella sua forma valida
  per sequenze aperiodiche e relative trasformate di Fourier per
  sequenze.
\item
  Dimostrare che, data una sequenza \(x[n]\) di \(N\) campioni,
  definendo la sequenza DFT mediante la formula

  \[
  \displaystyle
  X[K] = \sum_{n=0}^{N-1} x[n] e^{-j \frac{2\pi}{N} kn}, \ \ \ k = 0, 1 , \dots, N-1
  \]

  allora la sequenza \(x[n]\) è ricavabile da \[
  \displaystyle
  x[n] = \frac{1}{N} \sum_{n=0}^{N-1} X[K] = e^{j \frac{2\pi}{N} kn}
   \]
\item
  Enunciare e dimostrare la relazione tra le trasformate di Fourier
  della sequenza somma \(y[n]\) di una sequenza data \(x[n]\), e la
  Trasformata di \(x[n]\) stessa {[}\(^{***}\){]}
\item
  Enunciare e dimostrare la relazione che esiste tra la trasformata di
  Fourier di una sequenza \(x[n]\) ottenuta per campionamento di un
  segnale continuo \(x(t)\), e la Trasformata di Fourier di \(x(t)\)
  stesso
\item
  Calcolare i parametri della densità di probabilità dell'ampiezza
  dell'errore di quantizzazione nei casi di arrotondamento e di
  troncamento
\end{enumerate}

NOTA BENE: con {[}\(^{**}\){]} si intendono domande aggiunte da me! con
{[}\(^{***}\){]} si intende una domanda spostata dai quesiti alle
dimostrazioni

\newpage

\subsection{RISPOSTE QUESITI}\label{risposte-quesiti}

\begin{enumerate}
\def\labelenumi{\arabic{enumi}.}
\item
  Un processo aleatorio \(x\) è detto stazionario se la sua
  \emph{funzione di autocorrelazione}
  \(R_{XX}(t_1, t_2) = E[X(t_1), X(t_2)]\) dipende solo dalla distanza
  (o \emph{lag}) tra i campioni e \textbf{non} dai singoli istanti
  \(t_1\) e \(t_2\). - Un processo aleatorio è stazionario in senso
  stretto (SSS) se tutte le sue proprietà statistiche rimangono
  invariate nel tempo. Ovvero se la densità di probabilità rispetto ai
  singoli campioni, sia quella congiunta rispetto a più campioni non
  sono alterate da una transazione solidale nel tempo applicata agli
  indici dei campioni. Va verificato per tutti gli ordini i con
  \(i= 0, \dots, \infty\) \[
  f_x (x, t) = f_x(x, t+ \Delta t)\] \[
  f_x(x_1, x_2, t_1, t_2) = f_x(x_1, x_2, t_1 + \Delta t_1, t_2 + \Delta t_2)...
  \]
\item
  Un processo aleatorio è stazionario in senso lato (WSS):

  \begin{itemize}
  \tightlist
  \item
    la sua media e la sua varianza rimangono costanti nel tempo
    \(\to E[x(t)] = m_x(t) = m_x\)\\
  \item
    la sua autocorrelazione dipende solo dallo scarto temporale e non
    dal tempo assoluto (funzione solo dello scarto \(\tau\)).
    \(\to R_{xx}(t, t+\tau) = R_{xx}(\tau)\)
  \end{itemize}
\item
  Le proprietà che caratterizzano un processo aleatorio WSS includono:

  \begin{itemize}
  \item
    \(R_x(\tau) = R_x(-\tau) \to\) la funzione di autocorrelazione è una
    funzione pari
  \item
    \(R_x(0)= E[X^2(t)] = P_x \geq 0 \to\) potenza costante
  \item
    \(|R_x(\tau)| \leq R_x(0) \to\) la funzione di autocorrelazione ha
    il proprio massimo in \(\tau = 0\)
  \end{itemize}
\item
  Il rumore bianco è un tipo di segnale casuale che ha una potenza
  spettrale costante in tutto lo spettro di frequenze. Questo significa
  che tutte le frequenze sono presenti con la stessa intensità. Inoltre
  il rumore/processo bianco \(w(t)\) ha campioni incorrelati

  \begin{itemize}
  \item
    \(R_{ww}(\tau) = \frac{N_0}{2} \delta(\tau), \ \tau \neq 0\)
  \item
    \(E[w(t)] = 0\)
  \item
    \(S_{ww}(f) = \mathcal{F} \{\frac{N_0}{2}\} = \frac{N_0}{2} \to\)'\,'
    Potenza \textbf{infinita} \(\to\) densità spettrale di potenza
  \item
    Nota: si chiama bianco in quanto possedendo nello spettro
    \emph{tutte} componenti non nulle, si trova una similitudine con il
    colore bianco nello spettro dei colori
  \end{itemize}
\item
  Sapendo che \(y(t) = x(t) \otimes h(t)\) \begin{align*}
  E[y(t)] &= E[x(t)\otimes h(t)] = E[\int_{-\infty}^{\infty} h(\alpha)x(t - \alpha) d\alpha]
  = \int_{-\infty}^{\infty} h(\alpha)E[x(t - \alpha)] d\alpha =
     m_x \int_{-\infty}^{\infty} h(\alpha) d\alpha =\\&= m_x \cdot H(0)
  \\ R_y(t_1, t_2) &= E[y(t_1), y(t_2)] \to R_y(t, t-\tau) =
    E[y(t)y(t-\tau)] \to R_y(\tau) = R_x(\tau) \otimes (\tau) h(\tau) h(-\tau)=
  \ S_y(\tau) \\ &= S_x(f) \cdot H(f) \cdot H(-f) = S_x(f) \cdot H(f) \cdot H^*(-f) = S_x(f) \cdot |H(f)|^2
  \end{align*} Quindi è nuovamente un processo WSS
\end{enumerate}

\subparagraph{Risposte tdf sequenze
aperiodiche}\label{risposte-tdf-sequenze-aperiodiche}

\begin{enumerate}
\def\labelenumi{\arabic{enumi}.}
\setcounter{enumi}{5}
\item
  Il segnale \(x[n]\) viene espresso mediante la somma di molti termini
  \(X(f)\) i quali, al variare della frequenza \(\overline{X}(F)\) hanno
  un peso diverso (in ampiezza e in fase). A differenza di un segnale
  analogico, per esprimere una sequenza in ambito frequenziale sono
  \textbf{sufficienti} le sole componenti con frequenze comprese tra
  \([\frac{-1}{2T}, \frac{1}{2T}]\) (ciò è giustificato dalla
  periodicità della trasformata di Fourier per sequenze)
\item
  La condizione sufficiente per l'esistenza della trasformata per
  sequenze è che la sequenza sia \textbf{\emph{assolutamente
  sommabile}}, ovvero la somma dei valori assoluti dei suoi elementi sia
  finita.
  \[\displaystyle \sum_{n=-\infty}^{\infty} |x[n]| < + \infty \to |X(f)| < +\infty\]
  \[|\overline{X}(f)| = |\sum_{n=-\infty}^{\infty} x[n] e^{-j2\pi nfT}| \leq \sum_{n=-\infty}^{\infty} | x[n] e^{-j2\pi nfT}| \leq 
    \sum_{n=-\infty}^{\infty} |x[n]|\]
\item
  La denormalizzazione è necessaria quando la sequenza \(x[n]\) deriva
  da un'operazione di campionamento: in questo caso la frequenza \(F_0\)
  \((= \frac{F}{F_c}\) dove \(F\) è la frequenza e \(F_c\) è la
  frequenza di campionamento\()\) non permette di stabilire un legame
  \emph{immediato} con la frequenza espressa in Hz (e quindi
  \textbf{non} adimensionale) delle componenti nella trasformata del
  segnale analogico di partenza. Quindi se \(T\) = al periodo di
  campionamento e \(f=\frac{F}{T}=F\cdot F_c \to F=fT\) in Hz.
\item
  Il teorema dell'Incremento della Trasformata di Fourier per sequenze
  per sequenze reali afferma che l'incremento di una sequenza nel
  dominio del tempo corrisponde a una modulazione nel dominio della
  frequenza.
\item
  La condizione di convergenza della Trasformata di Fourier per sequenze
  è che la sequenza sia assolutamente sommabile, ovvero la somma dei
  valori assoluti dei suoi elementi sia finita.
\item
  \st{La trasformata di Fourier della sequenza...}
\item
  Se \(x[n]\) ha trasformata di Fourier per sequenze \(X(F)\), allora la
  sequenza \(y[n]\) con TDF \(Y(F)=X(F-F_0)\) è data da
  \(y[n]=x[n]e^{j2\pi F_0n}\). Questo è dovuto alla proprietà di
  traslazione in frequenza della trasformata di Fourier, che afferma che
  la traslazione di una funzione nel dominio della frequenza corrisponde
  a una modulazione esponenziale nel dominio del tempo.
\item
  La trasformata di Fourier della sequenza \(x[n]=u[n]-u[n-N]\), dove
  \(u[n]\) è la funzione gradino unitario, è data da: \begin{align*}
  \displaystyle \overline{X}(f) &= \sum_{n = -\infty}^{+\infty} (u[n]-u[n-N])e^{-2\pi nfT} = \sum_{n = 0}^{N-1}e^{-2\pi nfT} = \sum_{n = 0}^{N-1}(e^{-2\pi fT})^n = \\
  &\text{si applica la serie geometrica, con} \ q = e^{-j2\alpha}& \\
  &= \frac{1-e^{-2\pi N\alpha}}{1-e^{-2\pi \alpha}} = \frac{e^{j \alpha N}}{e^{\alpha}} \cdot \frac{e^{j\alpha N} - e^{-j\alpha N}}{e^{j\alpha } - e^{-j\alpha }} = e^{j \alpha (N-1)} \cdot \frac{sen(\alpha N)}{sen(\alpha)} = e^{-j\pi (N-1)fT} \cdot \frac{sen(N\pi fT)}{sen(\pi fT)}
  \end{align*}
\end{enumerate}

\subparagraph{Risposte campionamento e
interpolazione}\label{risposte-campionamento-e-interpolazione}

\begin{enumerate}
\def\labelenumi{\arabic{enumi}.}
\setcounter{enumi}{13}
\item
  ``Un segnale il cui spettro è \textbf{limitato} nella banda \(B\) può
  essere ricostruito esattamente dai propri campioni, purché la
  frequenza di campionamento non sia inferiore a \(2B\)''. La condizione
  di Nyquist: ``Una volta fissata \(B\) la frequenza di campionamento
  \(f_c = \frac{1}{T} \geq 2B, \to T \leq \frac{1}{2B}\)
\item
  L'aliasing è un effetto indesiderato che si verifica quando \emph{non
  viene rispettata} la frequenza di Nyquist (\(f_c < 2B\)). Può essere
  evitato campionando il segnale a una frequenza almeno doppia della
  banda. L'aliasing va evitato perché introduce distorsioni nel segnale
  ricostruito dal momento che le repliche, a causa della loro
  sovrapposizione si sommano, rendendo impossibile una ricostruzione
  \textbf{fedele} del segnale originale .
\item
  Il segnale \(x(t)=e^{-t}u(t)\) può essere campionato senza aliasing
  solo se la frequenza di campionamento è infinita. Questo perché il
  segnale ha componenti di frequenza che si estendono all'infinito,
  quindi non esiste una frequenza di Nyquist finita che possa essere
  utilizzata per campionare il segnale senza aliasing.
\item
  \(\displaystyle x(nT) = x[n] \text{ ottenuti da:} \ \hat{x}(t)= \sum_{n= -\infty}^{\infty} x[n] \cdot tri(\frac{t-nT}{T})= \sum_{n= -\infty}^{\infty}x[n] (1-\frac{|t|}{T})rect(\frac{t}{2T})\)
\item
  \(\overline{X}(f) = \frac{1}{T}\sum_{k=-\infty}^{\infty}X(f-\frac{k}{T})\)
\item
  \begin{itemize}
  \item
    L'interpolazione a mantenimento, o interpolazione a gradino, è un
    metodo di interpolazione che mantiene l'n-esimo valore della
    sequenza \(x[n]\) a partire dall'istante \(nT\) fino a \(nT+1\).
    Successivamente sarà mantenuto \(x[n+1]\) e così via\\
    \(\hat{x}(t)=\sum_{n=-\infty}^{+\infty}x[n]\cdot p(t-nT)\text{, con } p(t)= rect(\frac{t-\frac{T}{2}}{T})\).
  \item
    L'interpolazione cardinale, o sinc, utilizza come impulso \(p(t)\)
    una funzione sinc per interpolare tra i campioni:
    \(\to p(t)=sinc(\frac{t}{T}) \to \hat{x}(t)=\sum_{n=-\infty}^{+\infty} x[n]\cdot sinc(\frac{t-nT}{T})\).
    Rispetto alla interpolazione a mantenimento introduce meno
    discontinuità, ma nella realtà non è applicabile, dal momento che
    sono necessari un numero infinito di termini, dal fatto che non è
    \textbf{causale} e a causa della impossibilità di ricostruire un
    segnale rect in frequenza;
  \item
    L'interpolazione lineare utilizza un'impulso triangolare, in modo
    tale che la risultante sia una spezzata che unisce i vertici dei
    triangoli:
    \(p(t) = tri(\frac{t}{T}) \to \hat{x}(t) = \sum_n x[n]tri(\frac{t-nT}{T})\)

    \begin{itemize}
    \tightlist
    \item
      L'interpolazione a mantenimento è la più semplice, ma può
      introdurre distorsioni significative. L'interpolazione cardinale
      può fornire risultati più accurati, ma è computazionalmente più
      intensiva.
    \end{itemize}
  \end{itemize}
\item
  L'interpolazione a mantenimento ha il vantaggio di essere semplice da
  implementare e computazionalmente efficiente. Tuttavia, il segnale
  interpolato \textbf{non è limitato in banda}: in questo modo sono
  introdotte delle componenti frequenziali non presenti nel segnale
  analogico \(x(t)\). Derivano dalla presenza di repliche dello spettro
  del segnale a cavallo dei multipli delle frequenze di campionamento
  (sono dette \emph{immagini}). Inoltre anche all'interno della
  cosiddetta banda ``utile'' (intervallo base
  \(f \in[-\frac{1}{2T}, \frac{1}{2T}]\)) il segnale \(x(t)\) viene
  distorto in ampiezza (i due spettri sono legati dalla seguente
  relazione:
  \(\hat{X}(f)=\frac{P(f)X(f)}{T}=X(f)sinc(fT)e^{-j\pi fT}\)). Bisogna
  notare come sia possibile ridurre la presenza di immagini utilizzando
  un filtro ``\emph{anti-immagine}'' all'uscita dell'interpolatore:
  corrisponde ad un filtro passa-basso di banda B, che elimina le
  immagini dallo spettro del segnale interpolato.
\item
  \(\hat{x}(t)=\sum_{n=-\infty}^{+\infty}x[n]\cdot p(t-nT)\text{, con } p(t)= rect(\frac{t-\frac{T}{2}}{T})\)
\end{enumerate}

\subparagraph{Risposte TDF discreta}\label{risposte-tdf-discreta}

\begin{enumerate}
\def\labelenumi{\arabic{enumi}.}
\setcounter{enumi}{21}
\item
  La differenza tra una trasformata discreta e una per sequenze risiede
  nel fatto che la trasformata discreta è definita solo per un numero
  finito di punti, mentre la trasformata per sequenze può essere
  definita per sequenze infinite. Inoltre, la trasformata discreta
  assume che il segnale sia periodico, mentre la trasformata per
  sequenze non fa questa assunzione (?).
\item
  \begin{align*} 
  x[n] &= \cos^2(\frac{2\pi n}{10}), \ n= 0, \cdots, 19 \\
  x[n] &= \frac{1}{2}[1+\cos(2\frac{2\pi n}{10})] = \frac{1}{2} + \frac{1}{4}e^{\frac{j2\pi 4n}{20}} + \frac{1}{4}e^{\frac{-j2\pi 4n}{20}} \\
  x[k] &= \frac{1}{2}\sum_{k=0}^{19} e^{\frac{-j2\pi kn}{20}} + \frac{1}{4}\sum_{k=0}^{19} e^{\frac{-j2\pi (k-4)n}{20}} +\frac{1}{4}\sum_{k=0}^{19} e^{\frac{-j2\pi (k+4)n}{20}} = x_1[k]+x_2[k]+x_3[k]
  \end{align*} \[
  x_{1}[k] = \left\{ \begin{array}{cl}
  \frac{1}{2} \cdot 20 = 10 & \text{per } m=0 \\
  0 & \text{altrimenti}
  \end{array} \right., \
  x_{2}[k] = \left\{ \begin{array}{cl}
  \frac{1}{4} \cdot 20 = 5& \text{per } m=4 \\
  0 & \text{altrimenti}
  \end{array} \right.,
  \] \[
  x_{3}[k] = \left\{ \begin{array}{cl}
  \frac{1}{4} \cdot 20 = 5& \text{per } m=16 \\
  0 & \text{altrimenti}
  \end{array} \right.,\
  x[k] = \left\{ \begin{array}{cl}
  10& \text{per } m=0 \\
  5 & \text{per }m=4, 16\\
  0 & \text{altrimenti}
  \end{array} \right.
  \]
\item
  Il Teorema del prodotto per la DFT afferma che la DFT del prodotto di
  due sequenze è uguale alla convoluzione circolare delle DFT delle due
  sequenze. Il Teorema della convoluzione per la DFT afferma che la DFT
  della convoluzione di due sequenze è uguale al prodotto delle DFT
  delle due sequenze.
\item
  La Proprietà di Simmetria della Trasformata di Fourier per sequenze
  reali afferma che lo spettro di una sequenza reale è simmetrico
  rispetto all'origine. Questo significa che la parte positiva dello
  spettro è l'immagine speculare della parte negativa.
\item
  La proprietà di traslazione in frequenza della DFT afferma che la DFT
  di una sequenza moltiplicata per un'onda esponenziale complessa è
  uguale alla DFT della sequenza originale spostata in frequenza.
\item
  La proprietà di traslazione circolare della Trasformata Discreta di
  Fourier afferma che la DFT di una sequenza traslata circolarmente è
  uguale alla DFT della sequenza originale moltiplicata per un'onda
  esponenziale complessa.
\item
  Per una sequenza reale \(x[n]\) abbiamo: \[
  \operatorname{DFT}_{N_0}\Big\{ x[n] \Big\} = \operatorname{DFT}_{N_0}\Big\{x^{*}[n] \Big\} \to \overline{X}_k = \overline{X}^{*}_{-k} = \overline{X}^{*}_{N_0 -k}
  \] da cui derivano le proprietà di simmetria per il modulo e per la
  fase: \begin{gather*}
  \Big|\overline{X}_k \Big| = \Big| \overline{X}_{N_0 - k} \Big| \\
  \phase{\overline{X}_k} = -\phase{\overline{X}_{-k}}
  \end{gather*} Tali relazioni implicano che il modulo della sequenza
  \(X[k]\) è simmetrico rispetto al valore \(k = \frac{N_0}{2}\), mentre
  la fase è antisimmetrica rispetto a tale valore.

  \begin{itemize}
  \tightlist
  \item
    per sequenze di lunghezza \textbf{pari}, il centro di simmetria
    coincide con un campione della sequenza;
  \item
    per sequenze di lunghezza \textbf{dispari}, invece, il centro di
    simmetria coincide con un punto equidistante tra due campioni.
  \end{itemize}
\item
  \(x[n]\) sequenza di 10 campioni (per esempio posti ad 1) \newline
  \(X_{10}[k] \rightarrow\) DFT con periodicità 10 \newline
  \(X_{10}[k] \rightarrow\) DFT con periodicità 20 \newline
  \begin{align*}
  x_{10}[n]&=x[n], \quad 
  x_{20}[n] = \begin{cases}x[n] & n=0,\dots,9 \\
  0 & n=10,\dots,19\end{cases} \\
  X_{10}[k]&=\sum_{n=0}^9 x[n]  \\
  X_{20}[k]&=\sum_{n=0}^{19} x[n] \cdot e^{\frac{-j2\pi nk}{20}}= \\
  & =\sum_{n=0}^9 x[n] e^{\frac{-j2\pi nk}{20}} \\
  X_{20}[2k]&=\sum_{n=0}^9 x[n] e^{\frac{-2 \pi n \cdot \cancel{2}k}{\cancel{20}_{10}}} =
   =\sum_{n=0}^{9} x[n] e^{\frac{-j2\pi nk}{10}}= \\
  & =X_{10}[K] \\
  \end{align*} Quindi corrispondono i campioni \textbf{pari} delle
  sequenze!
\end{enumerate}

\subparagraph{Risposte sistemi}\label{risposte-sistemi}

\begin{enumerate}
\def\labelenumi{\arabic{enumi}.}
\setcounter{enumi}{29}
\item
  \(h(n) = \frac{1}{N}(u[n]-u[n-N])\)
\item
  Ha una risposta impulsiva uguale alla funzione gradino unitario, cioè
  \(h(n) = u(n)\)
\item
  Un filtro a media mobile prende in ingresso un numero \(N\) di
  campioni della sequenza d'ingresso, calcolandone la media aritmetica
  degli ultimi \(N\) valori, a partire dall'istante \(n^{*}\) verso
  tempi decrescenti, producendo un valore \(y[n^{*}]\) in uscita. Il
  nome media mobile deriva dalla traslazione in avanti di un ``passo'',
  al fine di ottenere il valore \(y[n^{*}+1]\) (la media viene
  ricalcolata!)
\item
  Il filtro accumulatore numerico è un sistema che somma di tutti i
  campioni arrivati al suo ingresso fino all'istante \(n\). Ha un
  risposta impulsiva uguale alla funzione gradino unitario, cioè
  \(h(n)=u(n)\). Quindi: \begin{align*}
  y[n] &=\sum_{k=-\infty}^{\infty}x[k] \ h[n-k] = \sum_{k=-\infty}^{\infty}x[k] \ u[n-k] = \\
  & =\sum_{k=-\infty}^{n}x[k] \\ &\ u[n-k] = 1 \text{ sse } n\geq k
  \end{align*} Il sistema è \emph{instabile}, ma l'uscita non è sempre
  illimitata: ad esempio esistono ingressi limitati (ad esempio segnali
  sinusoidali) per i quali l'uscita sarà comunque limitata.
\item
  Il filtro derivatore numerico o operatore differenza, e opera la
  differenza tra due campioni adiacenti. La sua risposta impulsiva è
  pari a: \begin{align*}
  h[n] &= \delta[n] - \delta[n-1], \text{ oppure } h[n]= \delta[n+1]-\delta[n]& \\
  y[n] &= \sum_{k=-\infty}^{\infty}x[k] \ h[n-k] = x[n] - x[n-1], \ y[n] = \sum_{k=-\infty}^{\infty}x[k] \ h[n-k] = x[n+1] - x[n] 
  \end{align*} Il primo è causale, il secondo no!
\item
  Domanda sulla trasformata z, da non considerare
\item
  \st{In uno schema di convoluzione veloce, il numero di moltiplicazioni
  reali per campione di uscita dipende dalla lunghezza del segnale e
  dalla lunghezza della risposta impulsiva del sistema. In generale, la
  convoluzione veloce richiede meno moltiplicazioni rispetto alla
  convoluzione diretta, ma il numero esatto dipende dai dettagli
  dell'implementazione dell'algoritmo FFT utilizzato per calcolare la
  convoluzione(?).}
\item
\item
  Un sistema SLS FIR (Finite Impulse Response) ha la sua risposta
  impulsiva costituita da un numero finito di campioni, mentre un
  sistema SLS è IIR (Infinite Impulse Response) se la sua risposta
  impulsiva è costituita da un numero infinito di campioni.

  Inoltre, condizione necessaria e sufficiente per la stabilità in senso
  BIBO di un SLS è la assoluta sommabilità della sua risposta impulsiva:
  \[
  \sum_{k=-\infty}^{\infty}\Big|h[k]\Big| < +\infty
  \] I sistemi di tipo IIR, invece, non sono sempre stabili: per essi, è
  necessario controllare la validità o meno della condizione:

  un SLS è causale se e solo se la sua risposta impulsiva è una sequenza
  causale: \[
  h[n] = 0 \text{ se } n < 0, \text{ ovvero } h[n] = h[n] \cdot u[n]
  \]
\item
  Un quantizzatore uniforme è ottenuto imponendo una distanza costante
  tra le soglie e i livelli di quantizzazione (quindi il \emph{passo}
  \(\Delta\) è costante
  \(\to x_{i+1}-x_i=\Delta; \ \hat{x}_{i+1}-\hat{x}_i = \Delta\)). Per
  quanto riguarda la ``scelta'' dei livelli di quantizzazione nei
  quantizzatori uniformi si distinguono i:

  \begin{itemize}
  \item
    Midtread: i livelli di quantizzazione si estendono su un intervallo
    approssimativamente \emph{simmetrico}, dal momento che i livelli
    sono un numero \emph{pari} ed è incluso lo 0 \[
    \to \{\hat{x}_i : i=0,\_,2^B -1\}= \{-2^{B-1}\Delta,\cdots,-\Delta,0,\Delta,\cdots,(2^{B-1}-1)\Delta \}
    \]
  \item
    Midrise: in questo caso i livelli coprono un intervallo
    \textbf{esattamente simmetrico} rispetto all'origine, tuttavia il
    valore 0 \textbf{non} è compreso nell'insieme (il passo è \(\Delta\)
    e parto da \(\frac{\Delta}{2}\)): \[
    \to \{\hat{x}_i : i=0,\_,2^B -1\}= \{-(2^{B-1}-\frac{1}{2})\Delta,\cdots,-\frac{\Delta}{2},\frac{\Delta}{2},\cdots,(2^{B-1}-\frac{1}{2})\Delta \}
    \]
  \end{itemize}
\item
  In un quantizzatore uniforme, in base alla scelta della regola di
  associazione tra \(x(nT)\leftrightarrow \hat{x}(nT)\) si può usare:

  \begin{itemize}
  \item
    arrotondamento: dove ad \(x(nT)\) viene associato il livello
    \(\hat{x}_i\) più vicino. Inoltre le soglie di quantizzazione
    risultano essere posizionate nel \emph{punto medio} tra i due
    livelli di quantizzazione. La relazione che permette di trovare il
    valore quantizzato \(\hat{x}(nT)\) a partire dal campione \(x(nT)\)
    è: \[
    \hat{x}(nT) = \{\hat{x}_i : i=arg \ min_k (|x(nT)=\hat{x}_k| \}
    \] L'errore è \(0 \leq |e(nT)| \leq \frac{\Delta}{2}\)
  \item
    troncamento: ad \(x(nT)\) viene associato il livello \(\hat{x}_i\)
    più vicino tra tutti quelli minori o uguali a \(x(nT)\). Le soglie
    di quantizzazione coincidono con i livelli di quantizzazione. La
    relazione che permette di trovare il valore quantizzato
    \(\hat{x}(nT)\) a partire dal campione \(x(nT)\): \[
    \hat{x}(nT) = \{\hat{x}_i : i=arg \ max_k (\hat{x}_k \text{ con } \hat{x}_k \leq x(nT)) \}
    \] L'errore è \(0 \leq |e(nT)| < \Delta\)
  \end{itemize}
\item
  Il rumore di quantizzazione \(e(nT)\) è l'errore introdotto dal
  processo di quantizzazione, modellato come processo aleatorio
  \emph{additivo}: \(\hat{x}(nT)=x(nT)+e(nT)\). Le ipotesi usualmente
  assunte per il rumore di quantizzazione sono:

  \begin{itemize}
  \item
    \(e(nT)\) sia un processo stazionario in senso lato: quindi media,
    potenza e varianza \emph{costanti} e non dipendono da n;
  \item
    che la densità di probabilità di \(e(nT)\) sia di tipo
    \textbf{uniforme}, permettendo di valutare tali costanti.
  \item
    \(\{e(nT)\}\) incorrelato con processo \(\{x(nT)\}\)
  \item
    I campioni del processo \(\{e(nT)\}\) sono \textbf{incorrelati} tra
    loro
  \end{itemize}
\item
  Dal momento che l'escursione (pari a due volte l'ampiezza della
  sinusoide fratto la dinamica del quantizzatore = \(\frac{1}{\Delta}\))
  è pari ad \(1\) (viene quindi occupata tutta la dinamica), la dinamica
  \(D=2\) e la potenza sarà \(S=\frac{1}{2}\) allora
  \(\sigma^2_e = \frac{\Delta^2}{12}=\frac{1}{12} = \frac{4}{2^{2B}}=\frac{1}{3} 2^{-2B} \to SNR_q=\frac{S}{\sigma^2_e}=\frac{3}{2}2^{2B} \to SNR=6.02B+1.76dB\)
\end{enumerate}

\subparagraph{Risposte sistemi
digitali}\label{risposte-sistemi-digitali}

\begin{enumerate}
\def\labelenumi{\arabic{enumi}.}
\setcounter{enumi}{42}
\item
  Il codificatore di canale in un sistema di comunicazione digitale
  introduce (in modo \emph{controllato}) una \textbf{ridondanza} nella
  sequenza di informazioni binarie. In questo modo il ricevitore può
  utilizzare questa ridondanza per \textbf{attenuare e/o correggere} gli
  effetti negativi dovuti al rumore o alle interferenze incontrate nella
  trasmissione del segnale, riducendo quindi la probabilità d'errore dei
  bit di ``informazione''. In questo modo viene aumentata l'affidabilità
  e migliorata la fedeltà del segnale. \newline Ci sono due strategie
  per impiegare i codici a controllo di errore: - rivelare gli errori,
  per poi richiedere una ritrasmissione - correggere gli errori, non
  richiedendo eventualmente una ritrasmissione Può essere implementata
  con una:

  \begin{itemize}
  \tightlist
  \item
    codici a blocchi (n/k): Il messaggio è diviso in blocchi di \(k\)
    simboli, a cui vengono associate parole di codice formate da
    \(n=k+q\) simboli. Si introducono quindi \(q\) bit di ridondanza.

    \begin{itemize}
    \item
      codice a ripetizione: consiste nel ripetere una cifra binaria per
      \(m\) volte (\(m>0\)). Se viene riscontrato un errore vi è una
      decisione ``a maggioranza'' (se errori
      \(\displaystyle < \frac{N}{2}\)). Tuttavia vi è un numero elevato
      di bit da trasmettere.
    \item
      codifica ``non banale'' (controllo di parità): vengono mappati
      \(k\) bit (alla volta) di una sequenza di \(n\) bit (\(k<n\))
      andando a creare una \textbf{parola in codice}. Il suo
      funzionamento si basa su una combinazione di k bit di dati e
      \((n-k\)) bit di parità, che vengono calcolati in modo da rendere
      rilevabili e correggibili determinati tipi di errori. In questo
      caso vi è una ridondanza pari a \(\frac{n}{k}\)
    \end{itemize}
  \end{itemize}
\item
  Il codificatore di sorgente in un sistema di comunicazione digitale è
  il primo blocco del trasmettitore. Si occupa del processo di
  conversione di una sequenza analogica o digitale in una sequenza di
  bit (questo processo è detto \textbf{codifica di sorgente o
  compressione dei dati}). Volendo \emph{minimizzare} il numero di bit
  utilizzati, con la compressione è possibile rappresentare una data
  quantità di informazioni con un numero di bit \emph{minore} rispetto
  alla rappresentazione originale.
\item
  Un modulatore digitale in un sistema di comunicazione digitale è
  l'interfaccia per il canale di comunicazione. Sceglie la forma d'onda
  più adatta alla trasmissione sul canale selezionato, in quanto è
  necessario spostare l'intervallo di frequenze in banda base (ovvero
  dalle frequenze ``originali'') in altri intervalli più adatti.
  \newline Per modulazione s'intende il processo mediante il quale
  alcune caratteristiche di una portante vengono variate in accordo con
  un'onda modulante (segnale).

  La modulazione può essere:

  \begin{itemize}
  \item
    binaria: la sequenza viene trasmessa un bit alla volta, con un
    bit-rate costante e pari a R bit al secondo;
    \(\left\{ \begin{array}{cl}0 \to \ \delta_0(t)\\1 \to \ \delta_1(t)\end{array} \right.\to \text{durata della forma d'onda}: \frac{1}{R}s\)\\
  \item
    in alternativa il modulatore può trasmettere \(b\) bits alla volta,
    utilizzando \(M=2^b\) forme d'onda \textbf{distinte}
    \(s_i(t), i:0, \_, M-1\). Ovvero, viene utilizzata una forma d'onda
    per ognuna delle \(2^b\) possibili sequenze di bit possibili.
  \end{itemize}
\item
  Gli effetti del rumore possono essere ridotti aumentando la potenza
  del segnale trasmesso (limite nell'attrezzatura) (?)
\item
  Vedi sottopunti 45.
\item
  L'operazione con cui diversi segnali sono trasmessi sullo stesso
  canale (cavo, fibra, radio,\ldots) senza interferenze è chiamata
  multiplexing dei segnali e l'apparato che effettua tale operazione
  multiplex. Un multiplex riceve N segnali distinti
  \(s_1 (t), s_2 (t), …, s_ N(t)\) e li invia su un unico canale
  utilizzando opportune regole (o tecniche multiplex). Il segnale in
  uscita è una combinazione degli \(N\) segnali in ingresso, da cui
  possiamo ricavare nuovamente i segnali in ingresso con un'operazione
  di multiplexing

  \begin{itemize}
  \item
    multiplex a \textbf{divisione di frequenze} (o FDM - Frequency
    Division Multiplex):

    Se in ingresso al sistema abbiamo \(N\) segnali con lo spettro
    diverso da zero nell'intervallo \((0, B)\), in frequenza l'i-esimo
    segnale viene shiftato di una frequenza \(f_i = f_1 + iB\).

    Il canale di comunicazione in uscita, utilizzando per trasmettere il
    segnale \(y(t)\), deve avere una larghezza di banda \(NB\). I
    singoli segnali sono recuperabili senza distorsioni da \(y(t)\)
    poiché occupano zone di frequenza diverse!
  \item
    multiplex a \textbf{divisione di tempo} (o TDM - Time Division
    Multiplex):

    I diversi segnali si differenziano sostanzialmente per l'intervallo
    di tempo utilizzato per la trasmissione. Se abbiamo \(N\) segnali da
    trasmettere in un intervallo di tempo \(\tau\), ad ognuno di essi
    viene assegnato un proprio sotto intervallo: ciò vale anche negli
    intervalli successivi.

    L'intervallo \(\tau\) è denominato \emph{frame} ed il suo valore
    varia nell'ordine di centinaia-decine di millisecondi.
  \end{itemize}
\end{enumerate}

\end{document}
